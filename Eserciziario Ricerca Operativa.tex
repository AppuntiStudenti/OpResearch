\documentclass[a4paper]{report}
\usepackage{amssymb, amsfonts, amsmath, eurosym, cancel}
\usepackage{graphicx, import, wrapfig}
\usepackage{fixltx2e}
\usepackage[T1]{fontenc}
\usepackage[latin1]{inputenc}
\usepackage[italian]{babel}
\usepackage{vmargin}
\usepackage[usenames,dvipsnames]{color}
\usepackage[usenames,dvipsnames,svgnames,table]{xcolor}
\usepackage[italian]{varioref}
\usepackage{array}
%\usepackage{booktabs}
\usepackage{tikz,pgfplots,fp,ifthen}
\usepgfplotslibrary{fillbetween}
\usetikzlibrary{shapes,arrows,intersections, patterns}


\usepackage{hyperref}

\renewcommand*\arraystretch{1.5}

\newcommand{\OL}[1]{\overline{#1}}
\newcommand{\UL}[1]{\underline{#1}}
\newcommand{\st}{\mathrm{s.t.}}
\newcommand{\Sc}[1]{\multicolumn{1}{c|}{#1}}
\newcommand{\Hr}[1]{%
  \colorbox{red!50}{$\displaystyle#1$}}
\newcommand*\C[1]{\tikz[baseline=(char.base)]{
    \node[shape=circle,draw,inner sep=2pt] (char) {#1};}}
%\newcommand{\,,}{,\,}
    
\newcommand{\intne}[4]{\node [label={[above right=-4pt]45:#3}, name intersections={of=#1 and #2, by=#4}] at (#4) {$\bullet$}}
\newcommand{\intse}[4]{\node [label={[below right=-4pt]-45:#3}, name intersections={of=#1 and #2, by=#4}] at (#4) {$\bullet$}}
\newcommand{\intnw}[4]{\node [label={[above left=-4pt]135:#3}, name intersections={of=#1 and #2, by=#4}] at (#4) {$\bullet$}}
\newcommand{\intsw}[4]{\node [label={[below left=-4pt]-135:#3}, name intersections={of=#1 and #2, by=#4}] at (#4) {$\bullet$}}
\newcommand{\intn}[4]{\node [label={[above=-4pt]90:#3}, name intersections={of=#1 and #2, by=#4}] at (#4) {$\bullet$}}
\newcommand{\ints}[4]{\node [label={[below=-4pt]-90:#3}, name intersections={of=#1 and #2, by=#4}] at (#4) {$\bullet$}}
\newcommand{\inte}[4]{\node [label={[right=-4pt]0:#3}, name intersections={of=#1 and #2, by=#4}] at (#4) {$\bullet$}}
\newcommand{\intw}[4]{\node [label={[left=-4pt]180:#3}, name intersections={of=#1 and #2, by=#4}] at (#4) {$\bullet$}}

\newcommand{\dotgrid}[5]{%
    \foreach \x in {#1,...,#2} {%
    \foreach \y in {#3,...,#4}{%
    \edef\temp{%
    \noexpand\node [#5, draw, circle, inner sep=1pt] at (axis cs:\x,\y) {};%
    }\temp}%
    }}

%\newcolumntype{C}{>{\centering\arraybackslash$}p{\linewidth}<{$}}

\newcommand*{\mathcolor}{}
\def\mathcolor#1#{\mathcoloraux{#1}}
\newcommand*{\mathcoloraux}[3]{%
  \protect\leavevmode
  \begingroup
    \color#1{#2}#3%
  \endgroup
}

\newcommand{\CG}[1]{\color{green}{#1}}

% Comando che restituisce velocemente il troncamento di #1*(#2/#3) in ambiente matematico
\newcommand{\rest}[3]{\left\lfloor #1 \frac{#2}{#3} \right\rfloor}

% Comandi per stati dominati e dominanti.
% Uno stato nel comando Dx domina i corrispettivi stati in Sx
% D=Dominatore S=Schiavo (circa. non mi veniva in mente una lettera migliore)
\newcommand{\DA}[1]{\mathcolor{red}{\pmb{#1}}}
\newcommand{\SA}[1]{\mathcolor{red}{\cancel{#1}}}
\newcommand{\DB}[1]{\mathcolor{green}{\pmb{#1}}}
\newcommand{\SB}[1]{\mathcolor{green}{\cancel{#1}}}
\newcommand{\DC}[1]{\mathcolor{blue}{\pmb{#1}}}
\newcommand{\SC}[1]{\mathcolor{blue}{\cancel{#1}}}

% Comando che sottolinea, o comunque enfatizza, lo stato con la soluzione ottima
\newcommand{\opt}[1]{\UL{#1}}

%\includeonly{capitolo4}

\begin{document}
\setpapersize{A4}
\title{Esercizi di Ricerca Operativa}
\author{Simone Laierno}

\maketitle
\tableofcontents
\chapter*{Introduzione}
Questa �, o almeno si propone di essere, una raccolta degli esercizi proposti a lezione del corso Ricerca Operativa M tenuto dal prof. Silvano Martello all'interno del CdL di Ingegneria Informatica M dell'Universit� di Bologna.

Non ha pretese di esattezza, tutt'altro, ma spero sia d'aiuto a chi segue o seguir� il corso. \textbf{Qualsiasi errore, dubbio, correzione, ecc.} � pi� che bene accetto e pu� essere comunicato privatamente al mio contatto Facebook o al mio indirizzo \textbf{e-mail}: \href{mailto:simonelaierno@gmail.com}{simonelaierno@gmail.com}

\chapter{18/03/2014}

\section{Esercizio 1}

Sia dato - in linguaggio naturale - il seguente problema di ottimizzazione:
\begin{enumerate}
\item Un'azienda realizza due tipi di prodotti X e Y; 
\item Il profitto di 1T di prodotto Y � doppio di quello di 1T di prodotto X; 
\item La produzione di 1T di qualsiasi prodotto richiede 2 ore; 
\item Non si pu� comunque produrre per pi� di 9 ore;
\item Non si possono produrre pi� di 2T di Y; 
\item La produzione di X non pu� superare di pi� di una tonnellata la produzione di Y;
\end{enumerate}
Si modelli il problema come un problema di programmazione lineare, lo si porti in forma standard, si realizzi una rappresentazione grafica del problema e si ottimizzi la funzione di profitto attraverso il \textbf{metodo del simplesso} affinch� \textbf{si ottenga il massimo profitto dalla produzione dei prodotti} nel rispetto dei vincoli assegnati. Si utilizzi la \textit{regola di Dantzig} per scegliere le basi su cui fare pivot.

\subsection{Modellizzazione}
Si indichi con:
\begin{itemize}
\item $x_1$ il numero di tonnellate di prodotto X;
\item $x_2$ il numero di tonnellate di prodotto Y.
\end{itemize}
Lo scopo del nostro problema � di massimizzare i profitti ottenuti dalla produzione. Anche se non siamo a conoscenza degli esatti profitti dati da ogni prodotto, abbiamo comunque a disposizione la relazione data dalla proposizione 2, cio� la variabile $x_2$ rende il doppio della variabile $x_1$. Possiamo quindi esprimere cos� la funzione di profitto:
$$
\max z = x_1 + 2x_2
$$

Modelliamo ora i vincoli espressi dal problema.

Le relazione 3 e 4 ci impongono di non produrre per pi� di 9 ore, considerando che ogni tonnellata di prodotto richiede 2 ore per essere prodotta. Perci� il vincolo sar� espresso come:
$$
2x_1 + 2x_2 \leq 9
$$
La relazione 5 � molto semplice, indica semplicemente che non potremo produrre pi� di due tonnellate di prodotto Y:
$$
x_2 \leq 2
$$
L'ultima relazione (6) ci impone un limite superiore alla produzione del prodotto X, che non deve superare di pi� di una tonnellata la produzione del prodotto Y. Perci�:
$$
x_1 \leq x_2 + 1
$$
Infine imponiamo il vincolo, implicito, che la produzione non pu� essere negativa:
$$
x_1,x_2 \geq0
$$
Il modello matematico pu� essere quindi cos� riassunto:
\begin{align*}
\max z	&= x_1+2x_2 \\
\st\;\;2& x_1+2x_2 \leq 9\\
	  	& x_2 \leq 2\\
	  	& x_1 \leq x_2 + 1\\
	  	& x_1,x_2 \geq 0
\end{align*}

\subsection{Problema in forma grafica}

In figura \vref{fig:graph1} � rappresentato graficamente il problema presentato. In giallo � rappresentato il politopo $P$ e sono stati chiamati $\alpha,\beta,\gamma,\delta,\varepsilon$ i suoi cinque vertici, i quali sappiamo corrispondere ognuno ad una BFS.
Inoltre in figura � riportato il verso del gradiente della funzione obiettivo. Ricordiamo che � necessario che il politopo $P$ sia limitato nella direzione del gradiente o, pi� precisamente, che sia limitato nella direzione opposta al gradiente dopo aver trasformato la funzione obiettivo in una funzione di minimo. Ovviamente i due casi sono gli stessi, basti osservare che il problema � espresso equivalentemente dalle equazioni:
\begin{align*}
\max z&=x_1+2x_2 \\
\min \varphi=-z&=-x_1-2x_2
\end{align*}
I gradienti delle due funzioni $z$ e $\varphi$ sono perci�:
\begin{align*}
\nabla(z)&=\left(\frac{\partial z}{\partial x},\frac{\partial z}{\partial y}\right) = \left(1,2\right) \\
\nabla(\varphi)&=\left(\frac{\partial\varphi}{\partial x},\frac{\partial\varphi}{\partial y}\right) = -\nabla(z) = \left(-1,-2\right)
\end{align*}
I due vettori sono ovviamente uno l'opposto dell'altro e di conseguenza il gradiente di $z$ cresce dove decresce quello di $\varphi$. I problemi sono quindi equivalenti.

\begin{figure}[htbp]
\centering
\begin{tikzpicture}
\begin{axis}
[axis lines=middle, axis equal, enlargelimits, xlabel=$x_1$, ylabel=$x_2$,
 every axis x label/.style={
    at={(ticklabel* cs:1.01)},
    anchor=west,
 },
 every axis y label/.style={
    at={(ticklabel* cs:1.01)},
    anchor=south,
 },]
    \path[name path=AX] 
        (axis cs:\pgfkeysvalueof{/pgfplots/xmin},0)--
        (axis cs:\pgfkeysvalueof{/pgfplots/xmax},0);
    \path[name path=AY] 
        (axis cs:0,\pgfkeysvalueof{/pgfplots/ymin})--
        (axis cs:0,\pgfkeysvalueof{/pgfplots/ymax});
    \path[name path=UP]
    	(axis cs:\pgfkeysvalueof{/pgfplots/xmin},\pgfkeysvalueof{/pgfplots/ymax})--
    	(axis cs:\pgfkeysvalueof{/pgfplots/xmax},\pgfkeysvalueof{/pgfplots/ymax});
\addplot
[domain=0:4.5, samples=10, thick, blue, name path=2x2y9]
{-x+(9/2)} node [pos=0.15,pin={75:{\color{blue}$2x_1+2x_2=9$}}, inner sep=0pt] {};
\addplot
[domain=1:5, samples=10, thick, red, name path=yx-1]
{x-1} node [pos=0.7, anchor=east, pin={165:{\color{red}$x_1=x_2+1$}}, inner sep=0pt] {};
\addplot
[domain=0:5, samples = 10, thick, purple, name path=y2]
{2} node [pos=0.1, anchor=north, pin={90:{\color{purple}$x_2=2$}}, inner sep= 0pt] {};
\addplot[thick, fill=yellow, fill opacity=0.5] fill between [of=2x2y9 and AX, soft clip={domain=0:11/4},];
\addplot[white] fill between [of=2x2y9 and y2];
\addplot[pattern=north west lines, pattern color=red!10] fill between [reverse=true, of=AX and UP, soft clip={domain=0:5}];
\addplot[white] fill between [of=yx-1 and AX];
\addplot[pattern=north east lines, pattern color=blue!10] fill between [of=AX and 2x2y9, soft clip={domain=0:5}];
\addplot[pattern=vertical lines, pattern color=purple!10] fill between [of=AX and y2];
%\node [label={[above right=-6pt]45:$\alpha$}, name intersections={of=AX and AY, by=alp}] at (alp) {$\bullet$};
%\node [label={[below right=-6pt]-45:$\beta$}, name intersections={of=AY and y2, by=bet}] at (bet) {$\bullet$};
%\node [label={[above right=-6pt]45:$\gamma$}, name intersections={of=y2 and 2x2y9, by=gam}] at (gam) {$\bullet$};
%\node [name intersections={of=2x2y9 and yx-1, by=del}, label={[right=-4pt]0:$\delta$}] at (del) {$\bullet$};
%\node [name intersections={of=yx-1 and AX, by=eps}, label={[above=-6pt]90:$\epsilon$}] at (eps) {$\bullet$};
\intse{AX}{AY}{$\alpha$}{alp};
\intse{AY}{y2}{$\beta$}{bet};
\intne{y2}{2x2y9}{$\gamma$}{gam};
\inte{2x2y9}{yx-1}{$\delta$}{del};
\intn{yx-1}{AX}{$\varepsilon$}{eps};
\node at (axis cs:1,1) {$P$};
\addplot[-latex, thick] coordinates
           {(0,0) (1/2.24,2/2.24)} node [pos=1, anchor=north, label={90:{\small $\nabla z$}}] {};
\end{axis}
\end{tikzpicture}
\caption{Rappresentazione cartesiana del problema di programmazione lineare}
\label{fig:graph1}
\end{figure}

\subsection{Forma standard}
Ricordiamo che un problema di \textbf{programmazione lineare in forma standard} � nella forma (matriciale):
\begin{align*}
\min c'x& \\
Ax& = b \\
x& \geq 0
\end{align*}
Cio�, la funzione obiettivo deve essere sotto forma di minimo (il che � molto semplice, dato che basta moltiplicarla per $-1$), i vincoli devono essere tutti espressi sotto forma di \textit{equazioni} e tutte le variabili devono essere positive.
A tal scopo introduciamo una \textbf{variabile slack} per ogni disequazione con simbolo $\leq$.
\begin{alignat*}{7}
&\min \varphi = \quad && -x_1 \quad && -2x_2\\
&\;\st  &&+2x_1		&&+2x_2 		&&+\pmb{x_3}	&&		 		&&\qquad\qquad		&&=9\\
&	 	&&\qquad\qquad &&+x_2		&&\qquad\qquad	&& +\pmb{x_4}	&&					&&=2\\
&	 	&&+x_1		&&-x_2 \qquad	&&				&&\qquad\qquad	&&+\pmb{x_5}		&&=1\\
&		&&\quad\; x_1,	&&\quad\; x_2,		&&\quad\; x_3,		&&\quad\; x_4,		&&\quad\; x_5		&&\geq 0
\end{alignat*}

\subsection{Risoluzione tramite tableau}

\subsubsection{Richiami (molto blandi) di teoria}
Una \textbf{base} � determinata da una sottomatrice della matrice A dei vincoli, di dimensione $m\times n$, di $n$ colonne linearmente indipendenti (nel \textbf{tableau} la matrice A � quella delimitata dalle due righe disegnate). Spesso l'individuazione delle colonne della prima base � semplice perch� l'introduzione di variabili slack o di variabili surplus crea nella nostra matrice delle colonne con tutti 0 e solo un 1, il che rende probabile la formazione di una sottomatrice \textbf{identit�}.
Ricordiamo che scelta una base $\mathcal{B}$ tale che:
$$
\mathcal{B}=A_{\beta(i)}; \quad i=1,\ldots,m
$$
ad essa � associata una \textbf{soluzione base} $x$ tale che:
$$
x=x_j; \quad j=1,\ldots,n \\
x_j = 0 \quad \forall j : A_j\not\in \mathcal{B}
$$
cio� il valore di una variabile non in base � 0. Questa � inoltre detta una soluzione \textbf{ammissibile} (\textbf{BFS}) se si trova nelle regione ammissibile determinata dai vincoli.

Al tableau aggiungeremo in alto una riga che indicher� il \textbf{costo relativo} $\OL{c_j}$ della colonna $j$-esima. Basti sapere che se facciamo in modo che $\forall A_j \in \mathcal{B} : \OL{c_j} = 0$, avremo nella prima colonna il guadagno $-\varphi$ della funzione obiettivo e in tutti gli altri avremo effettivamente i costi relativi. Per una spiegazione del perch� di questo fenomeno magico, si rimanda al testo o alle slide del docente.

Si ricorda, infine, che la colonna $b$ dei termini noti verr� inserita a sinistra. Non � indispensabile, ma una semplice convenzione.

\subsubsection{Risoluzione}

Per realizzare il \textbf{tableau} � sufficiente ricordare le regole base. La matrice $A$ e la colonna $b$ si riportano fedelmente sotto la loro consueta forma di matrice. Le variabili $x_j$ sono i coefficienti della rispettiva variabile nella funzione obiettivo. Ovviamente per tutte le variabili slack e surplus, che sono state aggiunte artificialmente da noi, il loro valore � 0. 

La prima colonna, una volta scelta una base che ha la forma di una matrice identit� (e per ora assumeremo che sia sempre gi� pronta o facilmente costruibile) rappresenta banalmente la soluzione del sistema $Ix=b$ che altro non � che $b$ stesso.

L'ultimo valore da inserire � quello di $-\varphi$, che varie a seconda delle colonne che assumeremo inizialmente come base. Se, come spesso accadr�, scegliamo tutte colonne associate a variabili slack o surplus, il loro valore non influir� sulla funzione obiettivo ed essendo tutte le altre variabili automaticamente nulle perch� non sono in base, sar� nullo anche $-\varphi$. Il nostro caso attuale ricade in quest'ultimo descritto, ma se fosse stato altrimenti, avremmo semplicemente dovuto calcolare il valore di $-\varphi$ in base al valore delle variabili $x_1$ e $x_2$.

\begin{table}[htbp]
\centering
\begin{tabular}{rcccccc}
 & $-\varphi$ & $x_1$ & $x_2$ & $x_3$ & $x_4$ & $x_5$ \\
$\OL{c_j}$ & \Sc{0} & -1 & -2 & 0 & 0 & 0 \\
\cline{2-7}
$x_3$ & \Sc{9} & 2 & 2 & 1 & 0 & 0 \\
$x_4$ & \Sc{2} & 0 & 1 & 0 & 1 & 0 \\
$x_5$ & \Sc{1} & 1 & -1 & 0 & 0 & 1 \\
\end{tabular}
\caption{Tableau iniziale. Vertice $\alpha(0,0)$}
\label{tab:tab1}
\end{table}

In tabella \vref{tab:tab1} il tableau definitivo ricavato dal nostro problema. Si noti che le ultime 3 colonne formano gi� una matrice identit�, perci� le assumeremo come base.
\begin{align*}
\mathcal{B}&=\{A_3,A_4,A_5\}\\
x&=(0,0,9,2,1)
\end{align*}

Per sapere in che punto dello spazio originale a due dimensioni ci troviamo, � sufficiente guardare le variabili $x_1$ e $x_2$. \'E evidente che ci troviamo nell'origine, che appartiene al politopo $P$ trovato in precedenza e che in particolare � il \textbf{vertice} $\pmb{\alpha}$.
Poich� non tutti i $\OL{c_j}$ sono non negativi, la nostra non � la BFS ottima e dobbiamo muoverci in una BFS migliore. Applicando la \textbf{regola di Dantzig}, facciamo entrare in base la colonna con il costo relativo maggiore in valore assoluto (cio� il "pi� negativo"). Nel nostro caso, prenderemo in considerazione quindi la colonna $\pmb{A_2}$.
Per scegliere su quale elemento fare \textbf{pivoting}, dobbiamo ottenere il valore di $y_{\ell 2}$ tale che:
$$
\vartheta_{\max}=\min_{i:y_{i2}>0}\frac{y_{i0}}{y_{i2}}=\frac{y_{i0}}{y_{\ell 2}}
$$
Perci�, operando con gli elementi nel tableau:
\begin{align*}
\vartheta_{\max}=\min\left(\frac{9}{2},\frac{2}{1}\right)=\frac{2}{1}=\frac{y_{20}}{\pmb{y_{22}}}
\end{align*}
Faremo pivoting sull'elemento $y_{22}$ (cerchiato in tabella \vref{tab:tab2}). Il nostro scopo � ora far comparire uno 0 nella colonna dell'elemento pivot in tutte le righe tranne quella in cui si trova l'elemento pivot e far comparire un 1 in quest'ultima.

\begin{table}[htbp]
\centering
\begin{tabular}{rrcccccc}
 & & $-\varphi$ & $x_1$ & $x_2$ & $x_3$ & $x_4$ & $x_5$ \\
$R_0$ & $\OL{c_j}$ & \Sc{0} & -1 & -2 & 0 & 0 & 0 \\
\cline{3-8}
$R_1$ & $x_3$ & \Sc{9} & 2 & 2 & 1 & 0 & 0 \\
$R_2$ & $x_4$ & \Sc{2} & 0 & \C{1} & 0 & 1 & 0 \\
$R_3$ & $x_5$ & \Sc{1} & 1 & -1 & 0 & 0 & 1 \\
\end{tabular}
\caption{Pivoting su $y_{22}$. $A_2$ entra in base e $A_4$ esce.}
\label{tab:tab2}
\end{table}

Possiamo felicemente notare che $y_{22}=1$, perci� nulla da fare su $R_2$. Se cos� non fosse stato sarebbe bastato moltiplicare $R_2R$ per un coefficiente $h$. Algebricamente, per far comparire uno 0 in tutti gli altri elementi della colonna $A_2$, possiamo sostituire ad ogni riga la riga stessa sommata ad un'altra qualsiasi riga moltiplicata per un coefficiente $k$. Ovviamente la riga pi� comoda da sommare � la riga su cui stiamo facendo pivot $R_\ell$, avendo un comodissimo 1 nella colonna interessata. Perci� possiamo riassumere che l'operazione consentita su ogni riga $R_i$ e sulla riga di pivot $R_\ell$�:
\begin{align*}
R_\ell&\leftarrow hR_l \\
R_i&\leftarrow R_i+kR_\ell
\end{align*}
Queste sono dette \textbf{operazioni elementari di riga}. Applicando le regole al nostra tableau, operiamo:
\begin{align*}
R_0&\leftarrow R_0 + 2R_2; \\
R_1&\leftarrow R_1 - 2R_2; \\
R_3&\leftarrow R_3 + R_2
\end{align*}
Il nostro nuovo tableau diventa quindi quello in tabella \vref{tab:tab3}.

\begin{table}[htbp]
\centering
\begin{tabular}{rrcccccc}
 & & $-\varphi$ & $x_1$ & $x_2$ & $x_3$ & $x_4$ & $x_5$ \\
$R_0$ & $\OL{c_j}$ & \Sc{4} & -1 & 0 & 0 & 2 & 0 \\
\cline{3-8}
$R_1$ & $x_3$ & \Sc{5} & 2 & 0 & 1 & -2 & 0 \\
$R_2$ & $x_2$ & \Sc{2} & 0 & 1 & 0 & 1 & 0 \\
$R_3$ & $x_5$ & \Sc{3} & 1 & 0 & 0 & 1 & 1 \\
\end{tabular}
\caption{Secondo tableau. Vertice $\beta(0,2)$}
\label{tab:tab3}
\end{table}

Ora che $A_4$ � entrato in base e $A_2$ ne � uscito, abbiamo una nuova base $\mathcal{B}$ e una nuova BFS $x$:
\begin{align*}
\mathcal{B}&=\{A_3,A_2,A_5\} \\
x&=(0,2,5,0,3)
\end{align*}
Ci troviamo nel \textbf{vertice} $\pmb{\beta}$, ma questa non � ancora la BFS ottima. Possiamo osservare, infatti, che la colonna $A_1$ presenta un $\OL{c_j}$ negativo e sar� necessario fare ulteriormente pivoting su un elemento di questa. Otteniamo quindi il valore di $y_{\ell 1}$ tale che:
\begin{align*}
\vartheta_{\max}&=\min_{i:y_{i1}>0}\frac{y_{i0}}{y_{i1}}=\frac{y_{i0}}{y_{\ell 1}} \\
\vartheta_{\max}&=\min\left(\frac{5}{2},\frac{3}{1}\right)=\frac{5}{2}=\frac{y_{10}}{\pmb{y_{11}}}
\end{align*}
Faremo pivoting sull'elemento $y_{11}$ (cerchiato in tabella \vref{tab:tab4}). 
\begin{table}[htbp]
\centering
\begin{tabular}{rrcccccc}
 & & $-\varphi$ & $x_1$ & $x_2$ & $x_3$ & $x_4$ & $x_5$ \\
$R_0$ & $\OL{c_j}$ & \Sc{4} & -1 & 0 & 0 & 2 & 0 \\
\cline{3-8}
$R_1$ & $x_3$ & \Sc{5} & \C{2} & 0 & 1 & -2 & 0 \\
$R_2$ & $x_2$ & \Sc{2} & 0 & 1 & 0 & 1 & 0 \\
$R_3$ & $x_5$ & \Sc{3} & 1 & 0 & 0 & 1 & 1 \\
\end{tabular}
\caption{Pivoting su $y_{11}$. $A_1$ entra in base e $A_3$ esce.}
\label{tab:tab4}
\end{table}

Questa volta dobbiamo lavorare anche sulla riga dell'elemento pivot, dividendola per 2:
$$
R_1\rightarrow \frac{1}{2}R_1
$$
Partendo dal nuovo tableau in tabella \vref{tab:tab5}, facciamo pivoting sulle restanti righe in questo modo:
\begin{align*}
R_0&\rightarrow R_0 + R_1 \\
R_3&\rightarrow R_3 - R_1
\end{align*}
\begin{table}[htbp]
\centering
\begin{tabular}{rrcccccc}
 & & $-\varphi$ & $x_1$ & $x_2$ & $x_3$ & $x_4$ & $x_5$ \\
$R_0$ & $\OL{c_j}$ & \Sc{4} & -1 & 0 & 0 & 2 & 0 \\
\cline{3-8}
$R_1$ & $x_1$ & \Sc{$\frac{5}{2}$} & \C{1} & 0 & $\frac{1}{2}$ & -1 & 0 \\
$R_2$ & $x_2$ & \Sc{2} & 0 & 1 & 0 & 1 & 0 \\
$R_3$ & $x_5$ & \Sc{3} & 1 & 0 & 0 & 1 & 1 \\
\end{tabular}
\caption{Pivoting su $y_{11}$. $R_1$ divisa per 2.}
\label{tab:tab5}
\end{table}
Il nostro nuovo tableau, quindi, � quello in tabella \vref{tab:tab6}. 
\begin{table}[htbp]
\centering
\begin{tabular}{rrcccccc}
 & & $-\varphi$ & $x_1$ & $x_2$ & $x_3$ & $x_4$ & $x_5$ \\
$R_0$ & $\OL{c_j}$ & \Sc{$\frac{13}{2}$} & 0 & 0 & $\frac{1}{2}$ & 1 & 0 \\
\cline{3-8}
%\hline
$R_1$ & $x_1$ & \Sc{$\frac{5}{2}$} & 1 & 0 & $\frac{1}{2}$ & -1 & 0 \\
$R_2$ & $x_2$ & \Sc{2} & 0 & 1 & 0 & 1 & 0 \\
$R_3$ & $x_5$ & \Sc{$\frac{1}{2}$} & 0 & 0 & $-\frac{1}{2}$ & 2 & 1 \\
\end{tabular}
\caption{Terzo tableau. Vertice $\gamma\left(\frac{5}{2},2\right)$}
\label{tab:tab6}
\end{table}
Notiamo che tutti i $\OL{c_j}$ sono non negativi, perci� ci troviamo nella \textbf{BFS ottima}. La base $\mathcal{B}$ e la soluzione $x$ sono quindi:
\begin{align*}
\mathcal{B}&={A_1,A_2,A_5} \\
x&=\left(\frac{5}{2},2,0,0,\frac{1}{2}\right)
\end{align*}
La soluzione ottima � quella del vertice $\pmb{\gamma\left(\frac{5}{2},2\right)}$. Riassumendo, tutti i valori delle variabili in gioco sono i seguenti:
\begin{align*}
z&=-\varphi=\frac{13}{2}=6.5 \\
x_1&=\frac{5}{2}=2.5 \\
x_2&=2
\end{align*}
\subsection{Conclusione}
La soluzione ottima consiste nel produrre $2.5$T di prodotto X e $2$T di prodotto Y, ottenendo un \textbf{profitto} pari a $6.5$ volte quello di $1$T di prodotto X.

\section{Esercizio 2}

Sia dato - in linguaggio naturale - il seguente problema di ottimizzazione:
\begin{enumerate}
\item Un'azienda chimica produce due composti, 1 e 2, composti da due sostanze chimiche A e B;
\item Un lotto di composto 1 richiede $3$T di sostanza A e $3$T di sostanza B;
\item Un lotto di composto 2 richiede $6$T di sostanza A e $3$T di sostanza B;
\item Per motivi di mercato non si possono produrre pi� di 3 lotti di composto 1;
\item Si hanno a disposizione $12$T di composto A e $9$T di composto B;
\item Il profitto di un lotto di composto A � di $12\,000$\euro ;
\item Il profitto di un lotto di composto B � di $15\,000$\euro .
\end{enumerate}
Si modelli il problema come un problema di programmazione lineare, lo si porti in forma standard, si realizzi una rappresentazione grafica del problema e si ottimizzi la funzione di profitto attraverso il \textbf{metodo del simplesso} affinch� \textbf{si ottenga il massimo profitto dalla produzione dei prodotti} nel rispetto dei vincoli assegnati. Si utilizzi la \textit{regola di Bland} per scegliere le basi su cui fare pivot.

\subsection{Modellizzazione}
Si indichi con:
\begin{itemize}
\item $x_1$ il numero di lotti di composto 1;
\item $x_2$ il numero di lotti di composto 2.
\end{itemize}
Lo scopo del nostro problema � di massimizzare i profitti ottenuti dalla produzione. Per comodit� di rappresentazione, stabiliamo che la funzione di profitto $z$ esprima il profitto in \textbf{migliaia di euro}:
$$
\max z = 12x_1 + 15x_2
$$

Modelliamo ora i vincoli espressi dal problema.

Dalle relazioni 3, 4 e 5 possiamo dedurre i seguenti vincoli:
\begin{itemize}
\item Servono $3$T di prodotto A per produrre un lotto di composto 1 e $6$T per produrre un lotto di composto 2. In tutto non possiamo utilizzare pi� di $12$T di prodotto A.
\item Servono $3$T di prodotto B per produrre un lotto di composto 1 e $3$T per produrre un lotto di composto 2. In tutto non possiamo utilizzare pi� di $9$T di prodotto B.
\end{itemize}
\begin{align*}
3x_1 + 6x_2 &\leq 12 \\
3x_1 + 3x_2 &\leq 9
\end{align*}
La relazione 4 � cos� facilmente esprimibile:
$$
x_1 \leq 3
$$
Infine imponiamo il vincolo, implicito, che la produzione non pu� essere negativa:
$$
x_1,x_2 \geq0
$$
Il modello matematico pu� essere quindi cos� riassunto (sono state apportate semplificazione algebriche):
\begin{align*}
\max z	&= 12x_1+15x_2 \\
\st\;\; & x_1+2x_2 \leq 4\\
	  	& x_1+x_2 \leq 3\\
	  	& x_1 \leq 3\\
	  	& x_1,x_2 \geq 0
\end{align*}

\subsection{Problema in forma grafica}

In figura \vref{fig:graph2} � rappresentato graficamente il problema presentato. In giallo � rappresentato il politopo $P$ e sono stati chiamati $\alpha,\beta,\gamma,\delta$ i suoi quattro vertici, i quali sappiamo corrispondere ognuno ad una BFS.
Il gradiente della funzione obiettivo vale
\begin{equation*}
\nabla(z)=\left(\frac{\partial z}{\partial x},\frac{\partial z}{\partial y}\right) = \left(12,15\right) \\
\end{equation*}
Il politopo $P$ �, ovviamente, limitato nella direzione del gradiente (si fa notare che finora $P$ � sempre limitato in ogni direzione, quindi qualsiasi direzione avesse il gradiente non ci sarebbero problemi).

\begin{figure}[htbp]
\centering
\begin{tikzpicture}
\begin{axis}
[axis lines=middle, axis equal, enlargelimits, xlabel=$x_1$, ylabel=$x_2$,
 every axis x label/.style={
    at={(ticklabel* cs:1.01)},
    anchor=west,
 },
 every axis y label/.style={
    at={(ticklabel* cs:1.01)},
    anchor=south,
 },]
    \path[name path=AX] 
        (axis cs:\pgfkeysvalueof{/pgfplots/xmin},0)--
        (axis cs:\pgfkeysvalueof{/pgfplots/xmax},0);
    \path[name path=AY] 
        (axis cs:0,\pgfkeysvalueof{/pgfplots/ymin})--
        (axis cs:0,\pgfkeysvalueof{/pgfplots/ymax});
    \path[name path=UP]
    	(axis cs:\pgfkeysvalueof{/pgfplots/xmin},\pgfkeysvalueof{/pgfplots/ymax})--
    	(axis cs:\pgfkeysvalueof{/pgfplots/xmax},\pgfkeysvalueof{/pgfplots/ymax});
\addplot
[domain=0:4, samples=10, thick, blue, name path=x2y4]
{-.5*x+2} node [pos=0.15,pin={75:{\color{blue}$x_1+2x_2=4$}}, inner sep=0pt] {};
\addplot
[domain=0:3, samples=10, thick, red, name path=xy3]
{-x+3} node [pos=0.1, pin={75:{\color{red}$x_1+x_2=3$}}, inner sep=0pt] {};
\addplot
[domain=0:5, samples = 10, thick, purple, name path=x3]
(3,x) node [pos=0.5, anchor=north, pin={0:{\color{purple}$x_1=3$}}, inner sep= 0pt] {};
\addplot[thick, fill=yellow, fill opacity=0.5] fill between [of=x2y4 and AX, soft clip={domain=0:3}];
\addplot[white] fill between [of=xy3 and UP];
%\addplot[pattern=north east lines, pattern color=red!10] fill between [reverse=true, of=AX and UP, soft clip={domain=0:5}];
%\addplot[white] fill between [of=xy3 and AX];
\addplot[pattern=north east lines, pattern color=red!10] fill between [of=xy3 and AX];
\addplot[pattern=vertical lines, pattern color=blue!10] fill between [of=x2y4 and AX];
%\addplot[pattern=north east lines, pattern color=blue!10] fill between [of=AX and 2x2y9, soft clip={domain=0:5}];
\addplot[pattern=horizontal lines, pattern color=purple!10] fill between [of=AY and x3];
\intse{AX}{AY}{$\alpha$}{alp};
\intne{AY}{x2y4}{$\beta$}{bet};
\intne{x2y4}{xy3}{$\gamma$}{gam};
\intne{xy3}{AX}{$\delta$}{del};
\node at (axis cs:1.5,.5) {$P$};
\addplot[-latex, thick] coordinates
           {(0,0) (4/6.4,5/6.4)} node [pos=1, anchor=north, label={90:{\small $\nabla z$}}] {};
\end{axis}
\end{tikzpicture}
\caption{Rappresentazione cartesiana del problema di programmazione lineare}
\label{fig:graph2}
\end{figure}

\subsection{Forma standard}
Ricordiamo che un problema di \textbf{programmazione lineare in forma standard} � nella forma (matriciale):
\begin{align*}
\min c'x& \\
Ax& = b \\
x& \geq 0
\end{align*}
Trasformiamo la funzione obiettivo $z$ in $\varphi$ tale che:
\begin{equation*}
\varphi=-\frac{z}{3}=-4x_1-5x_2
\end{equation*}
Quindi introduciamo una \textbf{variabile slack} per ogni disequazione con simbolo $\leq$. Otterremo infine:
\begin{alignat*}{7}
&\min \varphi = \quad && -4x_1 \quad && -5x_2\\
&\;\st  &&+x_1		&&+2x_2 		&&+\pmb{x_3}	&&		 		&&\qquad\qquad		&&=4\\
&	 	&&+x_1		&&+x_2			&&\qquad\qquad	&& +\pmb{x_4}	&&					&&=3\\
&	 	&&+x_1		&&\qquad\qquad	&&				&&\qquad\qquad	&&+\pmb{x_5}		&&=3\\
&		&&\quad\; x_1,	&&\quad\; x_2,		&&\quad\; x_3,		&&\quad\; x_4,		&&\quad\; x_5		&&\geq 0
\end{alignat*}

\subsection{Risoluzione tramite tableau}

\begin{table}[htbp]
\centering
\begin{tabular}{rcccccc}
			&$-\varphi$ & $x_1$ & $x_2$ & $x_3$ & $x_4$ & $x_5$ \\
$\OL{c_j}$ 	& \Sc{0} 	& -4 	& -5 	& 0 	& 0 	& 0 \\
\cline{2-7}
$x_3$ 		& \Sc{4} 	& 1 	& 2 	& 1 	& 0 	& 0 \\
$x_4$ 		& \Sc{3} 	& 1 	& 1		& 0 	& 1 	& 0 \\
$x_5$ 		& \Sc{3} 	& 1 	& 0 	& 0 	& 0 	& 1 \\
\end{tabular}
\caption{Tableau iniziale. Vertice $\alpha(0,0)$}
\label{tab:tab21}
\end{table}

In tabella \vref{tab:tab21} il tableau ricavato dal nostro problema. Si noti che le ultime 3 colonne formano gi� una matrice identit�, perci� le assumeremo come base.
\begin{align*}
\mathcal{B}&=\{A_3,A_4,A_5\}\\
x&=(0,0,4,3,3)
\end{align*}
Ci troviamo nell'origine, che appartiene al politopo $P$ trovato in precedenza e che in particolare � il \textbf{vertice} $\pmb{\alpha}$.
Poich� non tutti i $\OL{c_j}$ sono non negativi, la nostra non � la BFS ottima e dobbiamo muoverci in una BFS migliore. Applicando la \textbf{regola di Bland}, facciamo entrare in base la colonna con l'indice minore. Nel nostro caso, prenderemo in considerazione quindi la colonna $\pmb{A_1}$.
Per scegliere su quale elemento fare \textbf{pivoting}, dobbiamo ottenere il valore di $y_{\ell 1}$ tale che:
$$
\vartheta_{\max}=\min_{i:y_{i1}>0}\frac{y_{i0}}{y_{i1}}=\frac{y_{i0}}{y_{\ell 1}}
$$
Perci�, operando con gli elementi nel tableau:
\begin{align*}
\vartheta_{\max}=\min\left(\frac{4}{1},\frac{3}{1},\frac{3}{1}\right)=\frac{3}{1}=\frac{y_{20}}{\pmb{y_{21}}}=\frac{y_{30}}{\pmb{y_{31}}}
\end{align*}
Abbiamo un \textbf{pareggio} tra gli elementi $y_{21}$ e $y_{31}$.Seguendo la \textbf{regola di Bland}, sceglieremo come pivot l'elemento che far� uscire dalla base la variabile con l'\textbf{indice minore}%
\footnote{Si fa notare che l'utilizzo della \textbf{regola di Bland} evita i casi di \textbf{loop} in presenza di basi degeneri durante l'algoritmo del simplesso. Questa propriet� � dimostrabile ma la dimostrazione esula dai nostri scopi.}.
Faremo quindi pivoting sull'elemento $y_{21}$ (cerchiato in tabella \vref{tab:tab22}) poich� far� uscire dalla base la variabile $x_4$. Il nostro scopo � ora far comparire uno 0 nella colonna dell'elemento pivot in tutte le righe tranne quella in cui si trova l'elemento pivot e far comparire un 1 in quest'ultima.
\begin{table}[htbp]
\centering
\begin{tabular}{rrcccccc}
 	  & 			&$-\varphi$ & $x_1$ & $x_2$ & $x_3$ & $x_4$ & $x_5$ \\
$R_0$ & $\OL{c_j}$ 	& \Sc{0} 	& -4 	& -5 	& 0 	& 0 	& 0 \\
\cline{3-8}
$R_1$ & $x_3$ 		& \Sc{4} 	& 1 	& 2 	& 1 	& 0 	& 0 \\
$R_2$ & $x_4$ 		& \Sc{3} 	& \C{1}	& 1		& 0 	& 1 	& 0 \\
$R_3$ & $x_5$ 		& \Sc{3} 	& 1 	& 0 	& 0 	& 0 	& 1 \\
\end{tabular}
\caption{Pivoting su $y_{21}$. $A_1$ entra in base e $A_4$ esce.}
\label{tab:tab22}
\end{table}
Poich� $y_{21}=1$ non c'� nulla da fare su $R_2$. Applichiamo le operazioni elementari di riga al nostro tableau come segue:
\begin{align*}
R_0&\leftarrow R_0 + 4R_2; \\
R_1&\leftarrow R_1 - R_2; \\
R_3&\leftarrow R_3 - R_2
\end{align*}
Il nostro nuovo tableau diventa quindi quello in tabella \vref{tab:tab23}.
\begin{table}[htbp]
\centering
\begin{tabular}{rrcccccc}
 	  & 			&$-\varphi$ & $x_1$ & $x_2$ & $x_3$ & $x_4$ & $x_5$ \\
$R_0$ & $\OL{c_j}$ 	& \Sc{12} 	& 0 	& -1 	& 0 	& 4 	& 0 \\
\cline{3-8}
$R_1$ & $x_3$ 		& \Sc{1} 	& 0 	& 1 	& 1 	& -1 	& 0 \\
$R_2$ & $x_1$ 		& \Sc{3} 	& 1		& 1		& 0 	& 1 	& 0 \\
$R_3$ & $x_5$ 		& \Sc{0} 	& 0 	& -1 	& 0 	& -1 	& 1 \\
\end{tabular}
\caption{Secondo tableau. Vertice $\delta(3,0)$}
\label{tab:tab23}
\end{table}

Ora che $A_1$ � entrato in base e $A_4$ ne � uscito, abbiamo una nuova base $\mathcal{B}$ e una nuova BFS $x$:
\begin{align*}
\mathcal{B}&=\{A_3,A_1,A_5\} \\
x&=(3,0,1,0,0)
\end{align*}
Ci troviamo nel \textbf{vertice} $\pmb{\delta}$, ma questa non � ancora la BFS ottima. Inoltre, ci troviamo nel caso di una \textbf{base degenere}: la variabile $x_5$, che � in base, ha valore nullo. Questo non dovrebbe comunque crearci problemi in quanto stiamo applicando la regola di Bland.
L'unica colonna ad avere un $\OL{c_j}$ negativo � $A_2$ ed � su questa che cercheremo l'elemento di pivot $y_{\ell 2}$:
\begin{align*}
\vartheta_{\max}&=\min_{i:y_{i2}>0}\frac{y_{i0}}{y_{i2}}=\frac{y_{i0}}{y_{\ell 2}} \\
\vartheta_{\max}&=\min\left(\frac{1}{1},\frac{3}{1}\right)=\frac{1}{1}=\frac{y_{10}}{\pmb{y_{12}}}
\end{align*}
Faremo pivoting sull'elemento $y_{12}$ (cerchiato in tabella \vref{tab:tab24}). 
\begin{table}[htbp]
\centering
\begin{tabular}{rrcccccc}
 	  & 			&$-\varphi$ & $x_1$ & $x_2$ & $x_3$ & $x_4$ & $x_5$ \\
$R_0$ & $\OL{c_j}$ 	& \Sc{12} 	& 0 	& -1 	& 0 	& 4 	& 0 \\
\cline{3-8}
$R_1$ & $x_3$ 		& \Sc{1} 	& 0 	& \C{1}	& 1 	& -1 	& 0 \\
$R_2$ & $x_1$ 		& \Sc{3} 	& 1		& 1		& 0 	& 1 	& 0 \\
$R_3$ & $x_5$ 		& \Sc{0} 	& 0 	& -1 	& 0 	& -1 	& 1 \\
\end{tabular}
\caption{Pivoting su $y_{12}$. $A_2$ entra in base e $A_3$ esce.}
\label{tab:tab24}
\end{table}
Le operazione di pivoting saranno:
\begin{align*}
R_0&\rightarrow R_0 + R_1 \\
R_2&\rightarrow R_2 - R_1 \\
R_3&\rightarrow R_3 + R_1
\end{align*}
Il nostro nuovo tableau, quindi, � quello in tabella \vref{tab:tab25}. 
\begin{table}[htbp]
\centering
\begin{tabular}{rrcccccc}
 	  & 			&$-\varphi$ & $x_1$ & $x_2$ & $x_3$ & $x_4$ & $x_5$ \\
$R_0$ & $\OL{c_j}$ 	& \Sc{13} 	& 0 	& 0 	& 1 	& 3 	& 0 \\
\cline{3-8}
$R_1$ & $x_2$ 		& \Sc{1} 	& 0 	& 1 	& 1 	& -1 	& 0 \\
$R_2$ & $x_1$ 		& \Sc{2} 	& 1		& 0		& -1 	& 2 	& 0 \\
$R_3$ & $x_5$ 		& \Sc{1} 	& 0 	& 0 	& 1 	& -2 	& 1 \\
\end{tabular}
\caption{Terzo tableau. Vertice $\gamma(2,1)$}
\label{tab:tab25}
\end{table}
Notiamo che tutti i $\OL{c_j}$ sono non negativi, perci� ci troviamo nella \textbf{BFS ottima}. La base $\mathcal{B}$ e la soluzione $x$ sono quindi:
\begin{align*}
\mathcal{B}&={A_2,A_1,A_5} \\
x&=(2,1,0,0,1)
\end{align*}
La soluzione ottima � quella del vertice $\pmb{\gamma(2,1)}$. Riassumendo, tutti i valori delle variabili in gioco sono i seguenti:
\begin{align*}
z&=-3\varphi=-3(-13)=39 \\
x_1&=2 \\
x_2&=1
\end{align*}
\subsection{Conclusione}
La soluzione ottima consiste nel produrre $2$ lotti di composto 1 e $1$ lotto di composto 2, ottenendo un \textbf{profitto} pari a $39\,000$\euro .

\section{Esercizio 3}

Sia dato - in linguaggio naturale - il seguente problema di ottimizzazione:
\begin{enumerate}
\item Un'azienda produce due tipi di composto A e B;
\item Il profitto del composto A � il doppio di quello del composto B;
\item Per motivi di mercato non si possono produrre pi� di $2$T di composto A;
\item Ogni tonnellata di ogni composto contiene $1$Q di sostanza base;
\item Ho a disposizione $3$Q di sostanza base;
\item $1$T di composto A contiene $1$Q di sostanza chimica;
\item $1$T di composto B contiene $2$Q di sostanza chimica;
\item Ho a disposizione $5$Q di sostanza chimica.
\end{enumerate}
Si modelli il problema come un problema di programmazione lineare, lo si porti in forma standard, si realizzi una rappresentazione grafica del problema e si ottimizzi la funzione di profitto attraverso il \textbf{metodo del simplesso} affinch� \textbf{si ottenga il massimo profitto dalla produzione dei prodotti} nel rispetto dei vincoli assegnati. Si utilizzi la \textit{regola di Dantzig} per scegliere le basi su cui fare pivot.

\subsection{Modellizzazione}
Si indichi con:
\begin{itemize}
\item $x_1$ il numero di tonnellate di composto 1;
\item $x_2$ il numero di tonnellate di composto 2.
\end{itemize}
Lo scopo del nostro problema � di massimizzare i profitti ottenuti dalla produzione. Anche se non siamo a conoscenza degli esatti profitti dati da ogni prodotto, abbiamo comunque a disposizione la relazione data dalla proposizione 2, cio� la variabile $x_1$ rende il doppio della variabile $x_2$. Possiamo quindi esprimere cos� la funzione di profitto:
$$
\max z = 2x_1 + x_2
$$

Modelliamo ora i vincoli espressi dal problema.

La relazione 3 � cos� facilmente esprimibile:
$$
x_1 \leq 2
$$
Dalle relazioni 4 e 5 possiamo dedurre il seguente vincolo:
\begin{equation*}
x_1 + x_2 \leq 3
\end{equation*}
Dalle relazioni 6, 7 e 8 possiamo dedurre il seguente vincolo:
\begin{equation*}
x_1 + 2x_2 \leq 5
\end{equation*}
Infine imponiamo il vincolo, implicito, che la produzione non pu� essere negativa:
$$
x_1,x_2 \geq0
$$
Il modello matematico pu� essere quindi cos� riassunto (sono state apportate semplificazione algebriche):
\begin{align*}
\max z	&= 2x_1+x_2 \\
\st\;\; & x_1 \leq 2\\
	  	& x_1+x_2 \leq 3\\
		& x_1+2x_2 \leq 5\\
	  	& x_1,x_2 \geq 0
\end{align*}

\subsection{Problema in forma grafica}

In figura \vref{fig:graph3} � rappresentato graficamente il problema presentato. In giallo � rappresentato il politopo $P$ e sono stati chiamati $\alpha,\beta,\gamma,\delta,\varepsilon$ i suoi cinque vertici, i quali sappiamo corrispondere ognuno ad una BFS.
Il gradiente della funzione obiettivo vale
\begin{equation*}
\nabla(z)=\left(\frac{\partial z}{\partial x},\frac{\partial z}{\partial y}\right) = \left(2,1\right) \\
\end{equation*}
Il politopo $P$ �, ovviamente, limitato nella direzione del gradiente (si fa notare che finora $P$ � sempre limitato in ogni direzione, quindi qualsiasi direzione avesse il gradiente non ci sarebbero problemi).

\begin{figure}[htbp]
\centering
\begin{tikzpicture}
\begin{axis}
[axis lines=middle, axis equal, enlargelimits, xlabel=$x_1$, ylabel=$x_2$,
 every axis x label/.style={
    at={(ticklabel* cs:1.01)},
    anchor=west,
 },
 every axis y label/.style={
    at={(ticklabel* cs:1.01)},
    anchor=south,
 },]
    \path[name path=AX] 
        (axis cs:\pgfkeysvalueof{/pgfplots/xmin},0)--
        (axis cs:\pgfkeysvalueof{/pgfplots/xmax},0);
    \path[name path=AY] 
        (axis cs:0,\pgfkeysvalueof{/pgfplots/ymin})--
        (axis cs:0,\pgfkeysvalueof{/pgfplots/ymax});
    \path[name path=UP]
    	(axis cs:\pgfkeysvalueof{/pgfplots/xmin},\pgfkeysvalueof{/pgfplots/ymax})--
    	(axis cs:\pgfkeysvalueof{/pgfplots/xmax},\pgfkeysvalueof{/pgfplots/ymax});
\addplot
[domain=0:4, samples=10, thick, blue, name path=x2y5]
{-.5*x+2.5} node [pos=0.8,pin={75:{\color{blue}$x_1+2x_2=5$}}, inner sep=0pt] {};
\addplot
[domain=0:3, samples=10, thick, red, name path=xy3]
{-x+3} node [pos=0.2, pin={85:{\color{red}$x_1+x_2=3$}}, inner sep=0pt] {};
\addplot
[domain=0:5, samples = 10, thick, purple, name path=x2]
(2,x) node [pos=0.5, anchor=north, pin={0:{\color{purple}$x_1=2$}}, inner sep= 0pt] {};
\addplot[thick, fill=yellow, fill opacity=0.5] fill between [of=x2y5 and AX, soft clip={domain=0:3}];
\addplot[white] fill between [of=xy3 and UP];
%\addplot[pattern=north east lines, pattern color=red!10] fill between [reverse=true, of=AX and UP, soft clip={domain=0:5}];
\addplot[white] fill between [of=x2 and AX];
\addplot[pattern=north east lines, pattern color=red!10] fill between [of=xy3 and AX];
\addplot[pattern=vertical lines, pattern color=blue!10] fill between [of=x2y5 and AX];
%\addplot[pattern=north east lines, pattern color=blue!10] fill between [of=AX and 2x2y9, soft clip={domain=0:5}];
\addplot[pattern=horizontal lines, pattern color=purple!10] fill between [of=AY and x2];
\intse{AX}{AY}{$\alpha$}{alp};
\intse{AY}{x2y5}{$\beta$}{bet};
\intne{x2y5}{xy3}{$\gamma$}{gam};
\inte{xy3}{x2}{$\delta$}{del};
\intne{x2}{AX}{$\varepsilon$}{eps};
\node at (axis cs:1,1) {$P$};
\addplot[-latex, thick] coordinates
           {(0,0) (2/2.24,1/2.24)} node [pos=.7, anchor=south, label={0:{\small $\nabla z$}}] {};
\end{axis}
\end{tikzpicture}
\caption{Rappresentazione cartesiana del problema di programmazione lineare}
\label{fig:graph3}
\end{figure}

\subsection{Forma standard}
Ricordiamo che un problema di \textbf{programmazione lineare in forma standard} � nella forma (matriciale):
\begin{align*}
\min c'x& \\
Ax& = b \\
x& \geq 0
\end{align*}
Trasformiamo la funzione obiettivo $z$ in $\varphi$ tale che:
\begin{equation*}
\varphi=-z=-2x_1-x_2
\end{equation*}
Quindi introduciamo una \textbf{variabile slack} per ogni disequazione con simbolo $\leq$. Otterremo infine:
\begin{alignat*}{7}
&\min \varphi = \quad && -2x_1 \quad\; && -x_2 \quad\;\; && \qquad\qquad && \qquad\qquad && \qquad\qquad && \\
&\;\st  &&+x_1			&&		 		&&+\pmb{x_3}	&&		 		&&					&&=2\\
&	 	&&+x_1			&&+x_2			&&				&& +\pmb{x_4}	&&					&&=3\\
&	 	&&+x_1			&&+2x_2			&&				&&				&&+\pmb{x_5}		&&=5\\
&		&&\quad\; x_1,	&&\quad\; x_2,	&&\quad\; x_3,	&&\quad\; x_4,	&&\quad\; x_5		&&\geq 0
\end{alignat*}

\subsection{Risoluzione tramite tableau}

\begin{table}[htbp]
\centering
\begin{tabular}{rcccccc}
			&$-\varphi$ & $x_1$ & $x_2$ & $x_3$ & $x_4$ & $x_5$ \\
$\OL{c_j}$ 	& \Sc{0} 	& -2 	& -1 	& 0 	& 0 	& 0 \\
\cline{2-7}
$x_3$ 		& \Sc{2} 	& 1 	& 0 	& 1 	& 0 	& 0 \\
$x_4$ 		& \Sc{3} 	& 1 	& 1		& 0 	& 1 	& 0 \\
$x_5$ 		& \Sc{5} 	& 1 	& 2 	& 0 	& 0 	& 1 \\
\end{tabular}
\caption{Tableau iniziale. Vertice $\alpha(0,0)$}
\label{tab:tab31}
\end{table}

In tabella \vref{tab:tab31} il tableau ricavato dal nostro problema. Si noti che le ultime 3 colonne formano gi� una matrice identit�, perci� le assumeremo come base.
\begin{align*}
\mathcal{B}&=\{A_3,A_4,A_5\}\\
x&=(0,0,2,3,5)
\end{align*}
Ci troviamo nell'origine, che appartiene al politopo $P$ trovato in precedenza e che in particolare � il \textbf{vertice} $\pmb{\alpha}$.
Poich� non tutti i $\OL{c_j}$ sono non negativi, la nostra non � la BFS ottima e dobbiamo muoverci in una BFS migliore. Applicando la \textbf{regola di Dantzig}, facciamo entrare in base la colonna il cui $\OL{c_j}$ � maggiore in valore assoluto. Nel nostro caso, prenderemo in considerazione quindi la colonna $\pmb{A_1}$.
Per scegliere su quale elemento fare \textbf{pivoting}, dobbiamo ottenere il valore di $y_{\ell 1}$ tale che:
$$
\vartheta_{\max}=\min_{i:y_{i1}>0}\frac{y_{i0}}{y_{i1}}=\frac{y_{i0}}{y_{\ell 1}}
$$
Perci�, operando con gli elementi nel tableau:
\begin{align*}
\vartheta_{\max}=\min\left(\frac{2}{1},\frac{3}{1},\frac{5}{1}\right)=\frac{2}{1}=\frac{y_{10}}{\pmb{y_{11}}}
\end{align*}
Faremo pivoting sull'elemento $y_{11}$ (cerchiato in tabella \vref{tab:tab32}). Il nostro scopo � ora far comparire uno 0 nella colonna dell'elemento pivot in tutte le righe tranne quella in cui si trova l'elemento pivot e far comparire un 1 in quest'ultima.
\begin{table}[htbp]
\centering
\begin{tabular}{rrcccccc}
 	  & 			&$-\varphi$ & $x_1$ & $x_2$ & $x_3$ & $x_4$ & $x_5$ \\
$R_0$ & $\OL{c_j}$ 	& \Sc{0} 	& -2 	& -1 	& 0 	& 0 	& 0 \\
\cline{3-8}
$R_1$ & $x_3$ 		& \Sc{2} 	& \C{1}	& 0 	& 1 	& 0 	& 0 \\
$R_2$ & $x_4$ 		& \Sc{3} 	& 1		& 1		& 0 	& 1 	& 0 \\
$R_3$ & $x_5$ 		& \Sc{5} 	& 1 	& 2 	& 0 	& 0 	& 1 \\
\end{tabular}
\caption{Pivoting su $y_{11}$. $A_1$ entra in base e $A_3$ esce.}
\label{tab:tab32}
\end{table}
Poich� $y_{11}=1$ non c'� nulla da fare su $R_1$. Applichiamo le operazioni elementari di riga al nostro tableau come segue:
\begin{align*}
R_0&\leftarrow R_0 + 2R_2; \\
R_2&\leftarrow R_2 - R_1; \\
R_3&\leftarrow R_3 - R_1.
\end{align*}
Il nostro nuovo tableau diventa quindi quello in tabella \vref{tab:tab33}.
\begin{table}[htbp]
\centering
\begin{tabular}{rrcccccc}
 	  & 			&$-\varphi$ & $x_1$ & $x_2$ & $x_3$ & $x_4$ & $x_5$ \\
$R_0$ & $\OL{c_j}$ 	& \Sc{4} 	& 0 	& -1 	& 2 	& 0 	& 0 \\
\cline{3-8}
$R_1$ & $x_1$ 		& \Sc{2} 	& 1 	& 0 	& 1 	& 0 	& 0 \\
$R_2$ & $x_4$ 		& \Sc{1} 	& 0		& 1		& -1 	& 1 	& 0 \\
$R_3$ & $x_5$ 		& \Sc{3} 	& 0 	& 2 	& -1 	& 0 	& 1 \\
\end{tabular}
\caption{Secondo tableau. Vertice $\varepsilon(2,0)$}
\label{tab:tab33}
\end{table}

Ora che $A_1$ � entrato in base e $A_3$ ne � uscito, abbiamo una nuova base $\mathcal{B}$ e una nuova BFS $x$:
\begin{align*}
\mathcal{B}&=\{A_1,A_4,A_5\} \\
x&=(2,0,0,1,3)
\end{align*}
Ci troviamo nel \textbf{vertice} $\pmb{\varepsilon}$, ma questa non � ancora la BFS ottima.
L'unica colonna ad avere un $\OL{c_j}$ negativo � $A_2$ ed � su questa che cercheremo l'elemento di pivot $y_{\ell 2}$:
\begin{align*}
\vartheta_{\max}&=\min_{i:y_{i2}>0}\frac{y_{i0}}{y_{i2}}=\frac{y_{i0}}{y_{\ell 2}} \\
\vartheta_{\max}&=\min\left(\frac{1}{1},\frac{3}{2}\right)=\frac{1}{1}=\frac{y_{20}}{\pmb{y_{22}}}
\end{align*}
Faremo pivoting sull'elemento $y_{22}$ (cerchiato in tabella \vref{tab:tab34}). 
\begin{table}[htbp]
\centering
\begin{tabular}{rrcccccc}
 	  & 			&$-\varphi$ & $x_1$ & $x_2$ & $x_3$ & $x_4$ & $x_5$ \\
$R_0$ & $\OL{c_j}$ 	& \Sc{4} 	& 0 	& -1 	& 2 	& 0 	& 0 \\
\cline{3-8}
$R_1$ & $x_1$ 		& \Sc{2} 	& 1 	& 0 	& 1 	& 0 	& 0 \\
$R_2$ & $x_4$ 		& \Sc{1} 	& 0		& \C{1}	& -1 	& 1 	& 0 \\
$R_3$ & $x_5$ 		& \Sc{3} 	& 0 	& 2 	& -1 	& 0 	& 1 \\
\end{tabular}
\caption{Pivoting su $y_{22}$. $A_2$ entra in base e $A_4$ esce.}
\label{tab:tab34}
\end{table}
Le operazioni di pivoting saranno:
\begin{align*}
R_0&\rightarrow R_0 + R_2 \\
R_3&\rightarrow R_3 - 2R_2
\end{align*}
Il nostro nuovo tableau, quindi, � quello in tabella \vref{tab:tab35}. 
\begin{table}[htbp]
\centering
\begin{tabular}{rrcccccc}
 	  & 			&$-\varphi$ & $x_1$ & $x_2$ & $x_3$ & $x_4$ & $x_5$ \\
$R_0$ & $\OL{c_j}$ 	& \Sc{5} 	& 0 	& 0 	& 1 	& 1 	& 0 \\
\cline{3-8}
$R_1$ & $x_1$ 		& \Sc{2} 	& 1 	& 0 	& 1 	& 0 	& 0 \\
$R_2$ & $x_2$ 		& \Sc{1} 	& 0		& \C{1}	& -1 	& 1 	& 0 \\
$R_3$ & $x_5$ 		& \Sc{1} 	& 0 	& 0 	& 1 	& -2 	& 1 \\
\end{tabular}
\caption{Terzo tableau. Vertice $\delta(2,1)$}
\label{tab:tab35}
\end{table}
Notiamo che tutti i $\OL{c_j}$ sono non negativi, perci� ci troviamo nella \textbf{BFS ottima}. La base $\mathcal{B}$ e la soluzione $x$ sono quindi:
\begin{align*}
\mathcal{B}&={A_2,A_1,A_5} \\
x&=(2,1,0,0,1)
\end{align*}
La soluzione ottima � quella del vertice $\pmb{\delta(2,1)}$. Riassumendo, tutti i valori delle variabili in gioco sono i seguenti:
\begin{align*}
z&=-\varphi=5 \\
x_1&=2 \\
x_2&=1
\end{align*}
\subsection{Conclusione}
La soluzione ottima consiste nel produrre $2$T di composto A e $1$T di composto B ottenendo un \textbf{profitto} pari a 5 volte il profitto di $1$T di composto B.


\chapter{31/03/2014}

I problemi saranno posti in maniera leggermente diversa, cio� quella fornita sul pdf reperibile sul sito del docente al seguente link (se il testo � effettivamente disponibile, s'intende): \url{http://www.or.deis.unibo.it/staff_pages/martello/testi_esercizi_ottimizzazione.pdf}.
Inoltre, anche se durante l'esercitazione non � stata trovata la soluzione dei problemi duali, dato che il metodo per individuarli � stato spiegato dal prof. nella lezione subito successiva, ho ritenuto opportuno e interessante cercarle io stesso e inserirle in questo eserciziario. A maggior ragione, le soluzioni dei duali \textbf{potrebbero essere errate}, per cui chiedo ad ognuno di provare a rivederle e comunicarmi gli eventuali errori trovati.
Inoltre, ho deciso - in maniera del tutto personale e arbitraria - di preporre la rappresentazione grafica alla risoluzione con tableau negli esercizi di ottimizzazione. L'unico motivo � che mi piace avere un'idea un po' pi� concreta di quello che sta succedendo sul piano geometrico.

\section{Esercizio 1}
Un'azienda chimica produce due tipi di composto, A e B, che danno lo stesso profitto, utilizzando una sostanza base della quale sono disponibili 8 quintali. Ogni tonnellata di composto (indipendentemente dal tipo) contiene un quintale di sostanza base. Il numero di tonnellate di composto A prodotto deve superare di almeno una unit� il numero di tonnellate di composto B prodotto. Per problemi di stoccaggio non si possono produrre pi� di 6 tonnellate di composto A. Si associ la variabile $x_1$ al composto A e la variabile $x_2$ al composto B.
\begin{enumerate}
\item Definire il modello LP che determina la funzione di massimo profitto.
\item Porre il modello in forma standard e risolverlo con il metodo delle due fasi e la regola di Bland, introducendo il minimo numero di variabili artificiali. Dire esplicitamente qual � la soluzione trovata.
\item Disegnare con cura la regione ammissibile.
\item Costruire il duale del modello definito al punto 2 e ricavarne le soluzioni ottime.
\item Imporre il vincolo di interezza sulle variabili (supporre che non si possano produrre frazioni di tonnellate) e risolvere il problema con il metodo branch-and-bound. [\textit{Questo punto non sar� analizzato perch� in data di stesura del documento (04/04/2014) l'argomento non � ancora stato trattato dal prof}]
\end{enumerate}

\subsection{Modellizzazione}

Si indichi con:
\begin{itemize}
\item $x_1$ il numero di tonnellate di composto A;
\item $x_2$ il numero di tonnellate di composto B.
\end{itemize}
Lo scopo del nostro problema � di massimizzare i profitti ottenuti dalla produzione. Anche se non siamo a conoscenza degli esatti profitti dati da ogni prodotto, sappiamo che entrambi i composti portano allo stesso profitto. Possiamo quindi esprimere cos� la funzione di profitto:
$$
\max z = x_1 + x_2
$$

Modelliamo ora i vincoli espressi dal problema.
Il modello matematico pu� essere quindi cos� riassunto (sono state apportate semplificazione algebriche):
\begin{align*}
\max z	&= 2x_1+x_2 \\
\st\;\;	& x_1+x_2 \leq 8\\
		& x_1 \geq x_2 + 1\\
		& x_1 \leq 6 \\
	  	& x_1,x_2 \geq 0
\end{align*}

\subsection{Problema in forma grafica}

In figura \vref{fig:graph4} � rappresentato graficamente il problema presentato. In giallo � rappresentato il politopo $P$ e sono stati chiamati $\alpha,\beta,\gamma,\delta$ i suoi quattro vertici, i quali sappiamo corrispondere ognuno ad una BFS.
Il gradiente della funzione obiettivo vale
\begin{equation*}
\nabla(z)=\left(\frac{\partial z}{\partial x},\frac{\partial z}{\partial y}\right) = \left(1,1\right) \\
\end{equation*}
Il politopo $P$ �, ovviamente, limitato nella direzione del gradiente (si fa notare che finora $P$ � sempre limitato in ogni direzione, quindi qualsiasi direzione avesse il gradiente non ci sarebbero problemi).

\begin{figure}[htbp]
\centering
\begin{tikzpicture}
\begin{axis}
[axis lines=middle, axis equal, enlargelimits, xlabel=$x_1$, ylabel=$x_2$,
 every axis x label/.style={
    at={(ticklabel* cs:1.01)},
    anchor=west,
 },
 every axis y label/.style={
    at={(ticklabel* cs:1.01)},
    anchor=south,
 },]
    \path[name path=AX] 
        (axis cs:\pgfkeysvalueof{/pgfplots/xmin},0)--
        (axis cs:\pgfkeysvalueof{/pgfplots/xmax},0);
    \path[name path=AY] 
        (axis cs:0,\pgfkeysvalueof{/pgfplots/ymin})--
        (axis cs:0,\pgfkeysvalueof{/pgfplots/ymax});
    \path[name path=UP]
    	(axis cs:\pgfkeysvalueof{/pgfplots/xmin},\pgfkeysvalueof{/pgfplots/ymax})--
    	(axis cs:\pgfkeysvalueof{/pgfplots/xmax},\pgfkeysvalueof{/pgfplots/ymax});
\addplot
[domain=0:8, samples=10, thick, blue, name path=xy8]
{-x+8} node [pos=0.2,pin={75:{\color{blue}$x_1+x_2=8$}}, inner sep=0pt] {};
\addplot
[domain=0:8, samples=10, thick, red, name path=xy1]
{x-1} node [pos=0.8, pin={-85:{\color{red}$x_1=x_2+1$}}, inner sep=0pt] {};
\addplot
[domain=0:8, samples = 10, thick, purple, name path=x6]
(6,x) node [pos=0.3, anchor=north, pin={0:{\color{purple}$x_1=6$}}, inner sep= 0pt] {};
\addplot[thick, fill=yellow, fill opacity=0.5] fill between [of=xy1 and AX, soft clip={domain=1:6}];
\addplot[white] fill between [of=xy8 and UP];
%\addplot[pattern=north east lines, pattern color=red!10] fill between [reverse=true, of=AX and UP, soft clip={domain=0:5}];
\addplot[white] fill between [of=x6 and AX];
\addplot[pattern=north east lines, pattern color=blue!10] fill between [of=xy8 and AX];
\addplot[pattern=north west lines, pattern color=red!10] fill between [of=xy1 and AX];
%\addplot[pattern=north east lines, pattern color=blue!10] fill between [of=AX and 2x2y9, soft clip={domain=0:5}];
\addplot[pattern=horizontal lines, pattern color=purple!10] fill between [of=AY and x6];
\ints{AX}{xy1}{$\alpha$}{alp};
\ints{xy1}{xy8}{$\beta$}{bet};
\intw{xy8}{x6}{$\gamma$}{gam};
\intnw{x6}{AX}{$\delta$}{del};
\node at (axis cs:4.5,1.5) {$P$};
\addplot[-latex, thick] coordinates
           {(0,0) (1/1.414,1/1.414)} node [pos=.3, anchor=south, label={45:{\small $\nabla z$}}] {};
\end{axis}
\end{tikzpicture}
\caption{Rappresentazione cartesiana del problema di programmazione lineare}
\label{fig:graph4}
\end{figure}

Possiamo osservare che anche solo dal grafico � facilmente intuibile dove si trover� la soluzione ottima. Il gradiente $\nabla z$ � \textbf{perpendicolare} allo spigolo $\OL{\beta \gamma}$, da ci� potremmo dedurre che non esiste una soluzione ottima, ma che ve ne sono infinite e tutte posizionate su questo spigolo. Riprenderemo questa considerazione in seguito, dopo aver risolto il problema con il metodo del simplesso.

\subsection{Forma standard}
Ricordiamo che un problema di \textbf{programmazione lineare in forma standard} � nella forma (matriciale):
\begin{align*}
\min c'x& \\
Ax& = b \\
x& \geq 0
\end{align*}
Trasformiamo la funzione obiettivo $z$ in $\varphi$ tale che:
\begin{equation*}
\varphi=-z=-x_1-x_2
\end{equation*}
Quindi introduciamo una \textbf{variabile slack} per ogni disequazione con simbolo $\leq$ e una \textbf{variabile surplus} per ogni disequazione con simbolo $\geq$. Otterremo infine:
\begin{alignat*}{7}
&\min \varphi = \quad && -x_1 \quad\; && -x_2 \quad\;\; && \qquad\qquad && \qquad\qquad && \qquad\qquad && \\
&\;\st  &&+x_1			&&+x_2	 		&&+\pmb{x_3}	&&		 		&&					&&=8\\
&	 	&&+x_1			&&-x_2			&&				&& -\pmb{x_4}	&&					&&=1\\
&	 	&&+x_1			&&				&&				&&				&&+\pmb{x_5}		&&=6\\
&		&&\quad\; x_1,	&&\quad\; x_2,	&&\quad\; x_3,	&&\quad\; x_4,	&&\quad\; x_5		&&\geq 0
\end{alignat*}

\subsection{Risoluzione tramite tableau}

\begin{table}[htbp]
\centering
\begin{tabular}{rcccccc}
			&$-\varphi$ & $x_1$ & $x_2$ & $x_3$ & $x_4$ & $x_5$ \\
$\OL{c_j}$ 	& \Sc{0} 	& -1 	& -1 	& 0 	& 0 	& 0 \\
\cline{2-7}
$R_1$ 		& \Sc{8} 	& 1 	& 1 	& 1 	& 0 	& 0 \\
$R_2$		& \Sc{1} 	& 1 	& -1	& 0 	& -1 	& 0 \\
$R_3$		& \Sc{6} 	& 1 	& 0 	& 0 	& 0 	& 1 \\
\end{tabular}
\caption{Tableau iniziale.}
\label{tab:tab41}
\end{table}

In tabella \vref{tab:tab41} il tableau ricavato dal nostro problema. A differenza dei precedenti esercizi, la fortuna non � dalla nostra parte e non abbiamo nessuna sottomatrice identit� a disposizione da utilizzare come base ammissibile.
Si potrebbe \textit{erroneamente} pensare che per ottenere una BFS sia sufficiente operare $R_2\leftarrow -1\cdot R_2$. Ma si fa subito notare che cos� facendo otterremo come base:
\begin{align*}
\mathcal{B}&=\{A_3,A_4,A_5\}\\
x&=(0,0,8,1,6)
\end{align*}
Questa \textbf{non � una BFS} in quanto ricade \textit{all'esterno} del politopo $P$. Per ottenere una BFS di partenza, quindi, ricorriamo alla \textbf{fase 1 del metodo del simplesso}.

\subsubsection{Fase 1 - aggiunta di variabili artificiali}

Per ottenere una BFS aggiungiamo un numero $n'\leq m$ di variabili artificiali tali da riuscire ad ottenere una BFS nel nuovo problema con $m$ vincoli e $n+n'$ variabili. Ipoteticamente, potremmo aggiungere sempre $n'=m$ variabili artificiali tali da formare gi� loro una sottomatrice identit� nel tableau, ma tale metodo risulterebbe molto sconveniente nel caso in cui i vincoli e le variabili fossero centinaia o migliaia. Inoltre, ma non meno importante, la traccia dell'esercizio richiede esplicitamente di \textbf{introdurre il minore numero di variabili artificiali}.

Per ridurre al minimo le variabili artificiali $x_i^a,\quad i=1,\cdots,n'$ � sufficiente aggiungerne una per ogni colonna della matrice identit� mancante nel tableau originale. Nel nostro caso manca solo la seconda colonna e sar� quella che introdurremo con l'\textit{unica} variabile artificiale $x^a$, trasformando il secondo vincolo in:
$$
x_1 - x_2 - x_4 + x^a = 1
$$
Il nostro scopo, dopo l'introduzione di $x^a$, sar� quello di \textbf{eliminarla} dalla base. Per far ci� bisogna fare in modo che questa valga zero e quindi introduciamo, a tale scopo, una nuova funzione obiettivo da minimizzare $\psi$ tale che:
$$
\psi = \sum_{i=1}^{n'}x_i^a = x^a
$$
Scriviamo il nuovo tableau in tabella \vref{tab:tab42} e applichiamo il simplesso per ottimizzare la nostra funzione $\psi$.
\begin{table}[htbp]
\centering
\begin{tabular}{rrccccccc}
 	  & 			&$-\psi$	& $x_1$ & $x_2$ & $x_3$ & $x_4$ & $x_5$	& $x^a$\\
$R_0$ & $\OL{c_j}$ 	& \Sc{0} 	& 0 	& 0 	& 0 	& 0 	& \Sc{0}& 1\\
\cline{3-9}
$R_1$ & $x_3$ 		& \Sc{8} 	& 1		& 1 	& 1 	& 0 	& \Sc{0}& 0 \\
$R_2$ & $x^a$ 		& \Sc{1} 	& 1		& -1	& 0 	& -1 	& \Sc{0}& 1 \\
$R_3$ & $x_5$ 		& \Sc{6} 	& 1 	& 0 	& 0 	& 0 	& \Sc{1}& 0 \\
\end{tabular}
\caption{Nuovo tableau con la variabile artificiale $x^a$.}
\label{tab:tab42}
\end{table}
Abbiamo una sottomatrice identit� formata dalla base:
$$
\mathcal{B}=\{A_3,A_6,A_5\}
$$
Per avere a avere a disposizione i valori delle coordinate della BFS del nuovo problema, � necessario che:
$$
y_{ij}=0 \quad \forall i,j:A_j\in\mathcal{B},i\neq j
$$
Condizione vera per ogni valore tranne $y_{06}$ che provvediamo ad annullare tramite l'operazione elementare di riga:
$$
R_0\leftarrow R_0 - R_2
$$
Nel nuovo tableau in figura \vref{tab:tab43} faremo pivoting sull'unica colonna con $\OL{c_j}<0$, cio� su $A_1$.
Per scegliere su quale elemento fare \textbf{pivoting}, dobbiamo ottenere il valore di $y_{\ell 1}$ tale che:
$$
\vartheta_{\max}=\min_{i:y_{i1}>0}\frac{y_{i0}}{y_{i1}}=\frac{y_{i0}}{y_{\ell 1}}
$$
Perci�, operando con gli elementi nel tableau:
$$
\vartheta_{\max}=\min\left(\frac{8}{1},\frac{1}{1},\frac{6}{1}\right)=\frac{1}{1}=\frac{y_{20}}{\pmb{y_{21}}}
$$
Faremo pivoting sull'elemento $y_{21}$ (cerchiato in tabella). Il nostro scopo � ora far comparire uno 0 nella colonna dell'elemento pivot in tutte le righe tranne quella in cui si trova l'elemento pivot e far comparire un 1 in quest'ultima.
\begin{table}[htbp]
\centering
\begin{tabular}{rrccccccc}
 	  & 			&$-\psi$	& $x_1$ & $x_2$ & $x_3$ & $x_4$ & $x_5$	& $x^a$\\
$R_0$ & $\OL{c_j}$ 	& \Sc{-1} 	& -1 	& 1 	& 0 	& 1 	& \Sc{0}& 0\\
\cline{3-9}
$R_1$ & $x_3$ 		& \Sc{8} 	& 1		& 1 	& 1 	& 0 	& \Sc{0}& 0 \\
$R_2$ & $x^a$ 		& \Sc{1} 	& \C{1} & -1	& 0 	& -1 	& \Sc{0}& 1 \\
$R_3$ & $x_5$ 		& \Sc{6} 	& 1 	& 0 	& 0 	& 0 	& \Sc{1}& 0 \\
\end{tabular}
\caption{Pivoting su $y_{21}$. $A_1$ entra in base e $A_6$ esce.}
\label{tab:tab43}
\end{table}
Poich� $y_{21}=1$ non c'� nulla da fare su $R_2$. Applichiamo le operazioni elementari di riga al nostro tableau come segue:
\begin{align*}
R_0&\leftarrow R_0 + R_2; \\
R_1&\leftarrow R_1 - R_2; \\
R_3&\leftarrow R_3 - R_2.
\end{align*}
Il nostro nuovo tableau diventa quindi quello in tabella \vref{tab:tab44}.
\begin{table}[htbp]
\centering
\begin{tabular}{rrccccccc}
 	  & 			&$-\psi$	& $x_1$ & $x_2$ & $x_3$ & $x_4$ & $x_5$	& $x^a$\\
$R_0$ & $\OL{c_j}$ 	& \Sc{0} 	& 0 	& 0 	& 0 	& 0 	& \Sc{0}& 1\\
\cline{3-9}
$R_1$ & $x_3$ 		& \Sc{7} 	& 0		& 2 	& 1 	& 1 	& \Sc{0}& -1 \\
$R_2$ & $x_1$ 		& \Sc{1} 	& 1		& -1	& 0 	& -1 	& \Sc{0}& 1 \\
$R_3$ & $x_5$ 		& \Sc{5} 	& 0 	& 1 	& 0 	& 1 	& \Sc{1}& -1 \\
\end{tabular}
\caption{Secondo tableau. Vertice $\alpha(1,0)$}
\label{tab:tab44}
\end{table}
Siamo giunti alla soluzione ottima, essendo $\OL{c_j}>0 \quad\forall j$. Inoltre la variabile artificiale $x^a$ non � pi� in base. La nuova base e la nuova soluzione sono:
\begin{align*}
\mathcal{B}&=\{A_3,A_1,A_5\} \\
x&=(1,0,7,0,5,0)
\end{align*}
Siamo nel vertice $\alpha(1,0)$ e quindi in una BFS da cui possiamo partire per la \textbf{fase 2} del metodo del simplesso.
\subsubsection{Fase 2 - Simplesso}
Per questa fase useremo come tableau di partenza quello in tabella \vref{tab:tab44} sostituendo la funzione obiettivo fittizia $\psi$ utilizzata in precedenza con la nostra vera funzione obiettivo $\varphi$. Manterremo la variabile artificiale (che si fa notare non cambia in alcun modo il nostro problema in quanto non faremo mai entrare in base) perch�, come vedremo poi, il suo costo relativo finale sar� utile ai fini della soluzione del problema duale.
Il tableau cos� ottenuto � quello in tabella \vref{tab:tab45}
\begin{table}[htbp]
\centering
\begin{tabular}{rrccccccc}
 	  & 			&$-\varphi$	& $x_1$ & $x_2$ & $x_3$ & $x_4$ & $x_5$	& $x^a$\\
$R_0$ & $\OL{c_j}$ 	& \Sc{0} 	& -1 	& -1 	& 0 	& 0 	& \Sc{0}& 0\\
\cline{3-9}
$R_1$ & $x_3$ 		& \Sc{7} 	& 0		& 2 	& 1 	& 1 	& \Sc{0}& -1 \\
$R_2$ & $x_1$ 		& \Sc{1} 	& 1		& -1	& 0 	& -1 	& \Sc{0}& 1 \\
$R_3$ & $x_5$ 		& \Sc{5} 	& 0 	& 1 	& 0 	& 1 	& \Sc{1}& -1 \\
\end{tabular}
\caption{Secondo tableau. Vertice $\alpha(1,0)$ e funzione obiettivo $\varphi$.}
\label{tab:tab45}
\end{table}
Per applicare il simplesso, dobbiamo fare in modo che:
$$
y_{ij}=0 \quad\forall i,j:j\in\mathcal{B}, i\neq j
$$
L'elemento $y_{01}$ � l'unico a non essere nullo. Ovviamo al problema con l'operazione di riga:
$$
R_0\leftarrow R_0 + R_1
$$
Otteniamo quindi il tableau in tabella \vref{tab:tab46}.
\begin{table}[htbp]
\centering
\begin{tabular}{rrccccccc}
 	  & 			&$-\varphi$	& $x_1$ & $x_2$ & $x_3$ & $x_4$ & $x_5$	& $x^a$\\
$R_0$ & $\OL{c_j}$ 	& \Sc{1} 	& 0 	& -2 	& 0 	& -1 	& \Sc{0}& 1\\
\cline{3-9}
$R_1$ & $x_3$ 		& \Sc{7} 	& 0		& 2 	& 1 	& 1 	& \Sc{0}& -1 \\
$R_2$ & $x_1$ 		& \Sc{1} 	& 1		& -1	& 0 	& -1 	& \Sc{0}& 1 \\
$R_3$ & $x_5$ 		& \Sc{5} 	& 0 	& 1 	& 0 	& 1 	& \Sc{1}& -1 \\
\end{tabular}
\caption{Secondo tableau. Vertice $\alpha(1,0)$}
\label{tab:tab46}
\end{table}
Per fare pivoting sceglieremo la colonna $A_2$ in base alla regola di Bland (avremmo scelto la stessa colonna anche con la regola di Dantzig). Cerchiamo quindi l'elemento pivot $y_{\ell 2}$.
\begin{align*}
\vartheta_{\max}&=\min_{i:y_{i2}>0}\frac{y_{i0}}{y_{i2}}=\frac{y_{i0}}{y_{\ell 2}} \\
\vartheta_{\max}&=\min\left(\frac{7}{2},\frac{5}{1}\right)=\frac{7}{2}=\frac{y_{10}}{\pmb{y_{12}}}
\end{align*}
Faremo pivoting sull'elemento $y_{12}$ (cerchiato in tabella \vref{tab:tab47}). 
\begin{table}[htbp]
\centering
\begin{tabular}{rrccccccc}
 	  & 			&$-\varphi$	& $x_1$ & $x_2$ & $x_3$ & $x_4$ & $x_5$	& $x^a$\\
$R_0$ & $\OL{c_j}$ 	& \Sc{1} 	& 0 	& -2 	& 0 	& -1 	& \Sc{0}& 1\\
\cline{3-9}
$R_1$ & $x_3$ 		& \Sc{7} 	& 0		& \C{2}	& 1 	& 1 	& \Sc{0}& -1 \\
$R_2$ & $x_1$ 		& \Sc{1} 	& 1		& -1	& 0 	& -1 	& \Sc{0}& 1 \\
$R_3$ & $x_5$ 		& \Sc{5} 	& 0 	& 1 	& 0 	& 1 	& \Sc{1}& -1 \\
\end{tabular}
\caption{Terzo tableau. Vertice $\alpha(1,0)$}
\label{tab:tab47}
\end{table}
Le operazioni elementari di riga, \textbf{in ordine}, sono:
\begin{align*}
R_0&\leftarrow R_0 + R_1 \\
R_1&\leftarrow \frac{R_1}{2} \\
R_2&\leftarrow R_2 + R_1 \\
R_3&\leftarrow R_3 - R_1
\end{align*}
Otterremo il tableau in tabella \vref{tab:tab48}.
\begin{table}[htbp]
\centering
{
	\newcommand{\sm}{$\frac{7}{2}$}
	\newcommand{\nm}{$\frac{9}{2}$}
	\newcommand{\um}{$\frac{1}{2}$}
	\newcommand{\tm}{$\frac{3}{2}$}
\begin{tabular}{rrccccccc}
 	  & 			&$-\varphi$	& $x_1$ & $x_2$ & $x_3$ & $x_4$ & $x_5$	& $x^a$\\
$R_0$ & $\OL{c_j}$ 	& \Sc{8} 	& 0 	& 0 	& 1 	& 0 	& \Sc{0}& 0\\
\cline{3-9}
$R_1$ & $x_2$ 		& \Sc{\sm}	& 0		& 1		& \um 	& \um 	& \Sc{0}& -\um \\
$R_2$ & $x_1$ 		& \Sc{\nm} 	& 1		& 0		& \um	& -\um 	& \Sc{0}& \um \\
$R_3$ & $x_5$ 		& \Sc{-\tm}	& 0 	& 0 	& -\um 	& \um 	& \Sc{1}& -\um \\
\end{tabular}
}
\caption{Quarto tableau. Vertice $\beta(\frac{9}{2},\frac{7}{2})$. $A_2$ entra in base al posto di $A_3$, che esce.}
\label{tab:tab48}
\end{table}

Siamo giunti alla soluzione ottima, essendo $\OL{c_j}>0 \quad\forall j$. La nuova base e la nuova soluzione sono:
\begin{align*}
\mathcal{B}&=\{A_2,A_1,A_5\} \\
x&=(1,0,7,0,5,0)
\end{align*}
Siamo nel vertice $\beta(\frac{9}{2},\frac{7}{2})$ ed appartiene, come previsto durante l'analisi geometrica, allo spigolo $\OL{\beta\gamma}$. Il valore della soluzione ottima � $\varphi=-8$, proviamo ora a calcolare il valore della soluzione con il vertice $\gamma(6,2)$:
$$
\varphi(6,2)=-6-2=-8
$$
Anche il vertice $\gamma$ � una soluzione ottima del nostro problema. Da ci� possiamo desumere che l'intero spigolo $\OL{\beta\gamma}$ � composto da infinite soluzioni ottime. D'altronde, spostandoci lungo $\OL{\beta\gamma}$ avanzeremo in direzione perpendicolare al gradiente della funzione obiettivo e il valore della soluzione non pu� cambiare.
Si fa notare infine che la nostra funzione obiettivo iniziale �:
$$
z=-\varphi=8
$$

\subsection{Soluzione del problema primale}
La soluzione ottima consiste nel produrre $4.5$T di composto A e $3.5$T di composto B ottenendo un profitto pari a 8 volte quello di $1$T di composto A (o di composto B, equivalentemente).

\subsection{Costruzione del problema duale}
Riportiamo, per comodit�, il problema primale espresso in forma standard.
\begin{alignat*}{7}
&\min \varphi = \quad && -x_1 \quad\; && -x_2 \quad\;\; && \qquad\qquad && \qquad\qquad && \qquad\qquad && \\
&\;\st  &&+x_1			&&+x_2	 		&&+\pmb{x_3}	&&		 		&&					&&=8\\
&	 	&&+x_1			&&-x_2			&&				&& -\pmb{x_4}	&&					&&=1\\
&	 	&&+x_1			&&				&&				&&				&&+\pmb{x_5}		&&=6\\
&		&&\quad\; x_1,	&&\quad\; x_2,	&&\quad\; x_3,	&&\quad\; x_4,	&&\quad\; x_5		&&\geq 0
\end{alignat*}
Ricordiamo che le regole base per la creazione del problema duale (considereremo solo quelle in grassetto nel caso di problemi primali in forma standard):
\begin{itemize}
\item \textbf{Ad ogni vincolo corrisponde una variabile duale};
\item \textbf{Ad ogni vincolo di uguaglianza, la rispettiva variabile duale � una variabile libera};
\item Ad ogni vincolo di non minoranza corrisponde una variabile duale non negativa;
\item \textbf{Ad ogni variabile non negativa nel primale corrisponde un vincolo con relazione di non maggioranza nel duale};
\item Ad ogni variabile libera nel primale corrisponde un vincolo di uguaglianza nel duale.
\end{itemize}
In dettaglio, ridefiniamo in questo modo il generico problema primale in forma standard:
\begin{align*}
\min c'x& \\
Ax& = b \\
x& \geq 0
\end{align*}
Sia $\pi$ il vettore delle variabili duali, il problema duale � il seguente:
\begin{align*}
\max \pi'b& \\
\pi'A& \leq c' \\
\pi'&\gtreqless 0
\end{align*}
Ove, i vettori $x,\pi,b,c$ e la matrice $A$ sono:
\begin{align*}
x'&=
\begin{bmatrix}
x_1 & x_2 & x_3 & x_4 & x_5
\end{bmatrix} \\
\pi'&=
\begin{bmatrix}
\pi_1 & \pi_2 & \pi_3
\end{bmatrix} \\
b'&=
\begin{bmatrix}
8 & 1 & 6
\end{bmatrix} \\
c'&=
\begin{bmatrix}
-1 & -1 & 0 & 0 & 0
\end{bmatrix} \\
A&=
\begin{bmatrix}
1  & 1  & 1  & 0  & 0 \\
1  & -1 & 0  & -1 & 0 \\
1  & 0  & 0  & 0  & 1 \\
\end{bmatrix}
\end{align*}
Da ci�, il corrispondente problema duale con la sua funzione obiettivo $\xi$:
\begin{alignat*}{7}
&\max \xi = \quad && +8\pi_1 \quad\; && +\pi_2 \quad\;\; && +6\pi_3 \quad\;\; && \\
&\;\st  &&+\pi_1		&&+\pi_2 		&&+\pi_3		&& \leq -1\\
&	 	&&+\pi_1		&&-\pi_2 		&&				&& \leq -1\\
&	 	&&+\pi_1		&&				&&				&& \leq 0 \\
&	 	&&				&&-\pi_2		&&				&& \leq 0 \\
&	 	&&				&&				&&+\pi_3		&& \leq 0 \\
&		&&\quad\;\pi_1,	&&\quad\;\pi_2,	&&\quad\;\pi_3,	&& \gtreqless 0
\end{alignat*}
Per trovare la soluzione del problema duale non � necessario trasformarlo in forma standard e applicare il metodo del simplesso. Il tableau del problema primale sul quale abbiamo applicato il metodo del simplesso contiene tutte le informazioni per avere la soluzione del problema duale.
\subsubsection{Richiami (sempre molto blandi) di teoria}
Per ottenere dal tableau del problema primale la soluzione del problema duale, � sufficiente ricordare che il problema duale � ottenuto a partire dal \textbf{criterio di ottimalit�}.
Per questo motivo, il costo relativo nel tableau finale - corrispondente alla soluzione ottima - � cos� esprimibile:
$$
\OL{c_j}=c_j-z_j=c_j-\pi'A_j \quad \forall j
$$
Se consideriamo le colonne $A_j$ corrispondenti alla base iniziale $\mathcal{B}_0$ di partenza del primo tableau - ricordando che � una matrice identit� - possiamo ottenere:
$$
\OL{c_j}=c_j-\pi_j \quad \forall j:A_j\in\mathcal{B}_0
$$
Applicando un semplice passaggio algebrico:
$$
\pi_j=c_j-\OL{c_j}
$$
Ove $c_j$ � il costo iniziale nel primo tableau e $\OL{c_j}$ il costo relativo nel tableau finale.
Nel caso in cui abbiamo fatto uso di variabili artificiali e della fase 1 del metodo del simplesso, allora per tale variabile - il cui costo � $c_j=0$ - vale:
$$
\pi_j=-\OL{c_j}
$$
\textit{� importante ricordare che bisogna utilizzare il primo tableau con le variabili artificiali ma con il vettore dei costi originario in cui le variabili artificiali hanno costo nullo.}
\subsection{Soluzione del problema duale}
Riportiamo nuovamente i tableau iniziale e finale rispettivamente nelle tabella \vref{tab:tab49} e \vref{tab:tab410}.
\begin{table}[htbp]
\centering
\begin{tabular}{rrccccccc}
 	  & 			&$-\varphi$	& $x_1$ & $x_2$ & $x_3$ & $x_4$ & $x_5$	& $x^a$\\
$R_0$ & $\OL{c_j}$ 	& \Sc{0} 	& -1 	& -1 	& 0 	& 0 	& \Sc{0}& 0\\
\cline{3-9}
$R_1$ & $x_3$ 		& \Sc{8} 	& 1		& 1 	& 1 	& 0 	& \Sc{0}& 0 \\
$R_2$ & $x^a$ 		& \Sc{1} 	& 1		& -1	& 0 	& -1 	& \Sc{0}& 1 \\
$R_3$ & $x_5$ 		& \Sc{6} 	& 1 	& 0 	& 0 	& 0 	& \Sc{1}& 0 \\
\end{tabular}
\caption{Tableau iniziale.}
\label{tab:tab49}
\end{table}
\begin{table}[htbp]
\centering
{
	\newcommand{\sm}{$\frac{7}{2}$}
	\newcommand{\nm}{$\frac{9}{2}$}
	\newcommand{\um}{$\frac{1}{2}$}
	\newcommand{\tm}{$\frac{3}{2}$}
\begin{tabular}{rrccccccc}
 	  & 			&$-\varphi$	& $x_1$ & $x_2$ & $x_3$ & $x_4$ & $x_5$	& $x^a$\\
$R_0$ & $\OL{c_j}$ 	& \Sc{8} 	& 0 	& 0 	& 1 	& 0 	& \Sc{0}& 0\\
\cline{3-9}
$R_1$ & $x_2$ 		& \Sc{\sm}	& 0		& 1		& \um 	& \um 	& \Sc{0}& -\um \\
$R_2$ & $x_1$ 		& \Sc{\nm} 	& 1		& 0		& \um	& -\um 	& \Sc{0}& \um \\
$R_3$ & $x_5$ 		& \Sc{-\tm}	& 0 	& 0 	& -\um 	& \um 	& \Sc{1}& -\um \\
\end{tabular}
}
\caption{Tableau finale.}
\label{tab:tab410}
\end{table}
La base iniziale � $\mathcal{B}_1={A_3,A_6,A_5}$. Applicando delle semplici sottrazioni, ricaviamo la soluzione del problema duale:
\begin{align*}
\pi_1&=c_3-\OL{c_3}=-1 \\
\pi_2&=c_6-\OL{c_6}=0 \\
\pi_3&=c_5-\OL{c_5}=0
\end{align*}
Perci�, la soluzione del problema duale � il vettore:
$$
\pi'=
\begin{bmatrix}
-1 & 0 & 0
\end{bmatrix}
$$
Per verificare la correttezza dei calcoli, applichiamo la soluzione alla funzione obiettivo del problema duale:
$$
\xi(-1,0,0)=8(-1)+0+6(0)=-8
$$
Il risultato �, come atteso, lo stesso del problema primale.

\subsubsection{Vincolo di interezza}

\textit{Sezione in fase di allestimento...ci rivediamo appena il prof. spiegher� i metodi per la ILP}.

\section{Esercizio 2}

Sia dato il seguente modello matematico di un problema di LP:
\begin{align*}
\min z	&= 2x_1+x_2 \\
\st\;\;	& x_1+x_2 \leq 2\\
		& 2x_1+3x_2\geq 6\\
	  	& x_1,x_2 \geq 0
\end{align*}
\begin{itemize}
\item Si rappresenti accuratamente il problema in forma grafica;
\item Si ricavi la forma standard;
\item Si risolva il problema tramite il metodo del simplesso a due fasi utilizzando il minor numero di variabili artificiali.
\end{itemize}

\subsection{Problema in forma grafica}

In figura \vref{fig:graph5} � rappresentato graficamente il problema presentato. In giallo � rappresentato il politopo $P$ e sono stati chiamati $\alpha,\beta,\gamma$ i suoi tre vertici, i quali sappiamo corrispondere ognuno ad una BFS.
Il gradiente della funzione obiettivo vale
\begin{equation*}
\nabla(z)=\left(\frac{\partial z}{\partial x},\frac{\partial z}{\partial y}\right) = \left(2,1\right) \\
\end{equation*}
Poich� questa volta il problema � rappresentato sotto forma di minimo, saremo interessati alla limitazione del politopo nella direzione \textbf{opposta} al gradiente. A tal fine utilizziamo una funzione obiettivo ausiliaria $\varphi$ tale che:
\begin{equation*}
\varphi = -z \Rightarrow \nabla\varphi = -\nabla z = (-2,-1)
\end{equation*}

\begin{figure}[htbp]
\centering
\begin{tikzpicture}
\begin{axis}
[axis lines=middle, axis equal, enlargelimits, xlabel=$x_1$, ylabel=$x_2$,
 every axis x label/.style={
    at={(ticklabel* cs:1.01)},
    anchor=west,
 },
 every axis y label/.style={
    at={(ticklabel* cs:1.01)},
    anchor=south,
 },xtick={1,2,3}]
    \path[name path=AX] 
        (axis cs:\pgfkeysvalueof{/pgfplots/xmin},0)--
        (axis cs:\pgfkeysvalueof{/pgfplots/xmax},0);
    \path[name path=AY] 
        (axis cs:0,\pgfkeysvalueof{/pgfplots/ymin})--
        (axis cs:0,\pgfkeysvalueof{/pgfplots/ymax});
    \path[name path=UP]
    	(axis cs:\pgfkeysvalueof{/pgfplots/xmin},\pgfkeysvalueof{/pgfplots/ymax})--
    	(axis cs:\pgfkeysvalueof{/pgfplots/xmax},\pgfkeysvalueof{/pgfplots/ymax});

\foreach \q in {1,...,6} {
	\ifthenelse{\q < 2}{\newcommand{\x}{2}}{\newcommand{\x}{\q}}
	\addplot [domain=(\x/2)-1:(\q/2), samples=10, ultra thin, purple] {-2*x+\q};
	}

\addplot
[domain=0:2.01, samples=10, thick, blue, name path=xy2]
{-x+2} node [pos=0.2,pin={75:{\color{blue}$x_1+x_2=2$}}, inner sep=0pt] {};

\addplot
[domain=0:3.01, samples=10, thick, red, name path=2x3y6]
{-(2/3)*x+2} node [pos=0.5, pin={75:{\color{red}$2x_1+3x_2=6$}}, inner sep=0pt] {};
\addplot[thick, fill=yellow, fill opacity=0.5] fill between [of=2x3y6 and AX];
\addplot[white] fill between [of=xy2 and AX];
%\addplot[pattern=north east lines, pattern color=red!10] fill between [reverse=true, of=AX and UP, soft clip={domain=0:5}];
\addplot[pattern=north east lines, pattern color=blue!10] fill between [of=xy2 and UP];
\addplot[pattern=vertical lines, pattern color=red!10] fill between [of=2x3y6 and AX];
\intne{AY}{xy2}{$\alpha$}{alp};
\intne{AX}{2x3y6}{$\beta$}{bet};
\intne{xy2}{AX}{$\gamma$}{gam};
\node at (axis cs:1.8,0.5) {$P$};
\addplot[-latex, thick] coordinates
           {(2/2.236,1/2.236) (0,0)} node [pos=.3, anchor=south, label={45:{\small $\nabla\varphi$}}] {};
\end{axis}
\end{tikzpicture}
\caption{Rappresentazione cartesiana del problema di programmazione lineare}
\label{fig:graph5}
\end{figure}

Osservando il fascio di rette perpendicolari al gradiente si pu� intuire che la nostra soluzione ottima si trover� nel vertice $\alpha$.

\subsection{Forma standard}

Aggiungendo una variabile slack e una variabile surplus, il problema in forma standard si presenta cos�:
\begin{alignat*}{6}
&\min z = \quad && +2x_1 \quad\; && +x_2 \quad\;\; && \qquad\qquad && \qquad\qquad && \\
&\;\st  &&+x_1			&&+x_2	 		&&+\pmb{x_3}	&&		 		&&=2\\
&	 	&&+2x_1			&&+3x_2			&&				&& -\pmb{x_4}	&&=6\\
&		&&\quad\; x_1,	&&\quad\; x_2,	&&\quad\; x_3,	&&\quad\; x_4,	&&\geq 0
\end{alignat*}

\subsection{Risoluzione tramite tableau}

\begin{table}[htbp]
\centering
\begin{tabular}{rccccc}
			&$-z$ & $x_1$ & $x_2$ & $x_3$ & $x_4$ \\
$\OL{c_j}$ 	& \Sc{0} 	& 2 	& 1 	& 0 	& 0 \\
\cline{2-6}
$R_1$ 		& \Sc{2} 	& 1 	& 1 	& 1 	& 0\\
$R_2$		& \Sc{6} 	& 2 	& 3		& 0 	& -1 \\
\end{tabular}
\caption{Tableau iniziale.}
\label{tab:tab411}
\end{table}

In tabella \vref{tab:tab411} il tableau ricavato dal nostro problema. Non abbiamo nessuna sottomatrice identit� a disposizione da utilizzare come base ammissibile quindi ricorriamo alla \textbf{fase 1 del metodo del simplesso} per ottenere una BFS di partenza.

\subsubsection{Fase 1 - aggiunta di variabili artificiali}

Manca solo la seconda colonna della matrice identit� con cui formare la BFS di partenza e la che introdurremo con l'\textit{unica} variabile artificiale $x^a$, trasformando il secondo vincolo in:
$$
2x_1 + 3x_2 - x_4 + x^a = 6
$$
Il nostro scopo, dopo l'introduzione di $x^a$, sar� quello di \textbf{eliminarla} dalla base. Per far ci� bisogna fare in modo che questa valga zero e quindi introduciamo, a tale scopo, una nuova funzione obiettivo da minimizzare $\psi$ tale che:
$$
\psi = \sum_{i=1}^{n'}x_i^a = x^a
$$
Scriviamo il nuovo tableau in tabella \vref{tab:tab412} e applichiamo il simplesso per ottimizzare la nostra funzione $\psi$.
\begin{table}[htbp]
\centering
\begin{tabular}{rrcccccc}
 	  & 			&$-\psi$	& $x_1$ & $x_2$ & $x_3$ & $x_4$	& $x^a$\\
$R_0$ & $\OL{c_j}$ 	& \Sc{0} 	& 0 	& 0 	& 0 	& \Sc{0}& 1\\
\cline{3-8}
$R_1$ & $x_3$ 		& \Sc{2} 	& 1		& 1 	& 1 	& \Sc{0}& 0 \\
$R_2$ & $x^a$ 		& \Sc{6} 	& 2		& 3		& 0 	&\Sc{-1}& 1 \\
\end{tabular}
\caption{Nuovo tableau con la variabile artificiale $x^a$.}
\label{tab:tab42}
\end{table}
Abbiamo una sottomatrice identit� formata dalla base:
$$
\mathcal{B}=\{A_3,A_5\}
$$
Per avere a avere a disposizione i valori delle coordinate della BFS del nuovo problema, � necessario che:
$$
y_{ij}=0 \quad \forall i,j:A_j\in\mathcal{B},i\neq j
$$
Condizione vera per ogni valore tranne $y_{05}$ che provvediamo ad annullare tramite l'operazione elementare di riga:
$$
R_0\leftarrow R_0 - R_2
$$
Nel nuovo tableau in figura \vref{tab:tab413} dobbiamo scegliere su quale colonna fare pivoting. Dato che la traccia non ci specifica nulla sulla regola da utilizzare \footnote{Oppure non ricordo se era stato specificato dal tutor all'inizio dell'esercizio durante la lezione [NdA]} applicheremo la regola di Dantzig e sceglieremo la colonna con il $\OL{c_j}$ pi� negativo, cio� $A_2$.
Per scegliere su quale elemento fare \textbf{pivoting}, dobbiamo ottenere il valore di $y_{\ell 2}$ tale che:
$$
\vartheta_{\max}=\min_{i:y_{i2}>0}\frac{y_{i0}}{y_{i2}}=\frac{y_{i0}}{y_{\ell 2}}
$$
Perci�, operando con gli elementi nel tableau:
$$
\vartheta_{\max}=\min\left(\frac{2}{1},\frac{6}{3}\right)
$$
Abbiamo un pareggio. Applichiamo ora la regola di Bland per risolvere il pareggio, scegliendo tra gli elementi su cui fare pivot quello con l'indice di riga minore:
$$
\vartheta_{\max}=\frac{2}{1}=\frac{y_{10}}{\pmb{y_{12}}}
$$
Faremo pivoting sull'elemento $y_{12}$ (cerchiato in tabella). Il nostro scopo � ora far comparire uno 0 nella colonna dell'elemento pivot in tutte le righe tranne quella in cui si trova l'elemento pivot e far comparire un 1 in quest'ultima.
\begin{table}[htbp]
\centering
\begin{tabular}{rrcccccc}
 	  & 			&$-\psi$	& $x_1$ & $x_2$ & $x_3$ & $x_4$	& $x^a$\\
$R_0$ & $\OL{c_j}$ 	& \Sc{-6} 	& -2 	& -3 	& 0 	& \Sc{1}& 0\\
\cline{3-8}
$R_1$ & $x_2$ 		& \Sc{2} 	& \C{1}	& 1 	& 1 	& \Sc{0}& 0 \\
$R_2$ & $x^a$ 		& \Sc{6} 	& 2		& 3		& 0 	&\Sc{-1}& 1 \\
\end{tabular}
\caption{Pivoting su $y_{12}$. $A_2$ entra in base e $A_3$ esce.}
\label{tab:tab413}
\end{table}
Poich� $y_{12}=1$ non c'� nulla da fare su $R_1$. Applichiamo le operazioni elementari di riga al nostro tableau come segue:
\begin{align*}
R_0&\leftarrow R_0 + 3R_1; \\
R_2&\leftarrow R_2 - 3R_1.
\end{align*}
Il nostro nuovo tableau diventa quindi quello in tabella \vref{tab:tab414}.
\begin{table}[htbp]
\centering
\begin{tabular}{rrcccccc}
 	  & 			&$-\psi$	& $x_1$ & $x_2$ & $x_3$ & $x_4$	& $x^a$\\
$R_0$ & $\OL{c_j}$ 	& \Sc{0} 	& 1 	& 0 	& 3 	& \Sc{1}& 0\\
\cline{3-8}
$R_1$ & $x_2$ 		& \Sc{2} 	& 1		& 1 	& 1 	& \Sc{0}& 0 \\
$R_2$ & $x^a$ 		& \Sc{0} 	& -1	& 0		&\C{-3}	&\Sc{-1}& 1 \\
\end{tabular}
\caption{Secondo tableau. $x^a$ ancora in base.}
\label{tab:tab414}
\end{table}
Siamo giunti alla soluzione ottima, ma non � ancora sufficiente. Possiamo osservare, infatti, che $x_3$ � subentrata in base al posto di $x_2$ e che quindi $x^a$ � ancora in base e noi non lo vogliamo. Se il problema fosse risolvibile dovrebbe esserlo \textit{a prescindere} dalla variabile artificiale, cio� dovremmo essere in grado di trovare una soluzione a questo tableau con $x^a$ fuori base.
Non tutto � ancora perduto. Osserviamo che la base in cui ci troviamo ora � \textbf{degenere}, e possiamo fare entrare al posto di $x^a$ una qualsiasi altra variabile senza creare problemi. Qualsiasi operazione elementare di riga, in tal caso, non apporterebbe modifiche al valore di $-\psi$ e rimarremmo comunque in basi ottime.
Possiamo fare pivot su qualsiasi elemento di $R_2$ purch� non sia nullo (e purch� non sia la stessa variabile artificiale, ovviamente). Questa volta sceglieremo $y_{23}$, consapevoli che sarebbero andati bene anche $y_{21}$ e $y_{24}$.
Applichiamo le operazioni elementari di riga, \textbf{nell'ordine}, per completare l'operazione di pivoting:
\begin{align*}
R_0&\leftarrow R_0 + R_2 \\
R_2&\leftarrow \frac{1}{3}R_2 \\
R_1&\leftarrow R_1 + R_2
\end{align*}
Otteniamo il tableau in tabella \vref{tab:tab415}. � ancora un tableau ottimo (non poteva essere diversamente) e questa volta nessuna fastidiosa variabile artificiale � in base.
\begin{table}[htbp]
\centering
{
	\newcommand{\ut}{$\frac{1}{3}$}
	\newcommand{\dt}{$\frac{2}{3}$}
\begin{tabular}{rrcccccc}
 	  & 			&$-\psi$	& $x_1$ & $x_2$ & $x_3$ & $x_4$	& $x^a$\\
$R_0$ & $\OL{c_j}$ 	& \Sc{0} 	& 0 	& 0 	& 0 	& \Sc{0}& 1\\
\cline{3-8}
$R_1$ & $x_2$ 		& \Sc{2} 	& \dt	& 1 	& 0 	&\Sc{-\ut}& \ut \\
$R_2$ & $x^3$ 		& \Sc{0} 	& -\ut	& 0		& 1 	&\Sc{-\ut}& 0 \\
\end{tabular}\caption{Terzo tableau. $A_3$ entra in base al posto di $A_5$. Vertice $\alpha(0,2)$}
}
\label{tab:tab415}
\end{table}
La nuova base e la nuova soluzione sono:
\begin{align*}
\mathcal{B}&=\{A_2,A_3\} \\
x&=(0,2,0,0,0)
\end{align*}
Siamo nel vertice $\alpha(0,2)$ e quindi in una BFS da cui possiamo partire per la \textbf{fase 2} del metodo del simplesso.
\subsubsection{Fase 2 - Simplesso}
Per questa fase useremo come tableau di partenza quello in tabella \vref{tab:tab415} sostituendo la funzione obiettivo fittizia $\psi$ utilizzata in precedenza con la nostra vera funzione obiettivo $z$. Manterremo la variabile artificiale (che si fa notare non cambia in alcun modo il nostro problema in quanto non faremo mai entrare in base) perch�, come vedremo poi, il suo costo relativo finale sar� utile ai fini della soluzione del problema duale.
Il tableau cos� ottenuto � quello in tabella \vref{tab:tab416}
\begin{table}[htbp]
\centering
{
	\newcommand{\ut}{$\frac{1}{3}$}
	\newcommand{\dt}{$\frac{2}{3}$}
\begin{tabular}{rrcccccc}
 	  & 			&$-z$		& $x_1$ & $x_2$ & $x_3$ & $x_4$	& $x^a$\\
$R_0$ & $\OL{c_j}$ 	& \Sc{0} 	& 2 	& 1 	& 0 	& \Sc{0}& 0\\
\cline{3-8}
$R_1$ & $x_2$ 		& \Sc{2} 	& \dt	& 1 	& 0 	&\Sc{-\ut}& \ut \\
$R_2$ & $x^3$ 		& \Sc{0} 	& -\ut	& 0		& 1 	&\Sc{-\ut}& 0 \\
\end{tabular}\caption{Quarto tableau. Vertice $\alpha(0,2)$}
}
\label{tab:tab416}
\end{table}
Per applicare il simplesso, dobbiamo fare in modo che:
$$
y_{ij}=0 \quad\forall i,j:j\in\mathcal{B}, i\neq j
$$
L'elemento $y_{02}$ � l'unico a non essere nullo. Ovviamo al problema con l'operazione di riga:
$$
R_0\leftarrow R_0 - R_1
$$
Otteniamo quindi il tableau in tabella \vref{tab:tab417}.
\begin{table}[htbp]
\centering
{
	\newcommand{\ut}{$\frac{1}{3}$}
	\newcommand{\dt}{$\frac{2}{3}$}
	\newcommand{\qt}{$\frac{4}{3}$}
\begin{tabular}{rrcccccc}
 	  & 			&$-z$		& $x_1$ & $x_2$ & $x_3$ & $x_4$	& $x^a$\\
$R_0$ & $\OL{c_j}$ 	& \Sc{-2} 	& \qt 	& 0 	& 0 	&\Sc{\ut} & -\ut\\
\cline{3-8}
$R_1$ & $x_2$ 		& \Sc{2} 	& \dt	& 1 	& 0 	&\Sc{-\ut}& \ut \\
$R_2$ & $x^3$ 		& \Sc{0} 	& -\ut	& 0		& 1 	&\Sc{-\ut}& 0 \\
\end{tabular}\caption{Quarto tableau. Vertice $\alpha(0,2)$}
}
\label{tab:tab417}
\end{table}
Il nostro lavoro si conclude qui in quanto $\OL{c_j}>0\quad\forall j>0$ e il vertice $\alpha$ � gi� quello della soluzione ottima, come d'altronde previsto durante la soluzione per via grafica. La base e la soluzione sono quelle gi� espresse in precedenza:
\begin{align*}
\mathcal{B}&=\{A_2,A_3\} \\
x&=(0,2,0,0,0)
\end{align*}

\subsection{Soluzione del problema}
La soluzione del problema �:
$$
z(\alpha)=z(0,2)=2
$$

\subsection{Extra - Costruzione del problema duale}
Anche se nessuno ce l'ha chiesto, proviamo a costruire e risolvere il problema duale a quello dato.
Riportiamo ora il problema primale e, per motivi di sintesi, poniamo anche una notazione pi� breve che ci permetter� di costruire il problema duale:
$$
{
	\renewcommand*\arraystretch{.7}
\begin{array}{r| @{}c@{} |c| @{}c@{} |c| @{}c@{}|c| @{}c@{} |c| c|}
\cline{2-2}\cline{4-4}\cline{6-6}\cline{8-8}\cline{10-10}
	 &\qquad\qquad& &\qquad\qquad& &\qquad\qquad& &\qquad\qquad& & \\
\min		& 2x_1 	&+& x_2 	& &   		& &   		& & \CG{\max}\\
	 		&\CG{=}	& &\CG{=}	& &\CG{=}	& &\CG{=}	& & 	\\
\CG{\pi_1}	& x_1	&+& x_2 	&+& x_3 	& &			&=& 2 	\\
	 		&\CG{+}	& &\CG{+}	& &\CG{+}	& &\CG{+}	& & 	\\
\CG{\pi_2}	& 2x_1 	&+&3x_2 	& &			&-& x_4		&=& 6	\\
	 		&  		& &  		& &	  		& &   		& & 	\\
\cline{2-2}\cline{4-4}\cline{6-6}\cline{8-8}\cline{10-10}
	 		&  		& &  		& &	  		& &   		& & 	\\
	 		& x_1	&,& x_2 	&,& x_3 	&,& x_4 	&\geq& 0\\
	 		&  		& &  		& &	  		& &   		& & 	\\
\cline{2-2}\cline{4-4}\cline{6-6}\cline{8-8}\cline{10-10}
\end{array}
}
$$
Ricordando che \textit{ad ogni variabile non negativa corrisponde un vincolo duale di non maggioranza}, il problema duale, perci�, � il seguente:
\begin{alignat*}{6}
&\max \xi = \quad && +2\pi_1 \quad\; && +6\pi_2 \quad\;\; && \\
&\;\st  &&+\pi_1		&&+2\pi_2 		&& \leq 2 \\
&	 	&&+\pi_1		&&+3\pi_2 		&& \leq 1\\
&	 	&&+\pi_1		&&				&& \leq 0 \\
&	 	&&				&&-\pi_2		&& \leq 0 \\
&		&&\quad\;\pi_1,	&&\quad\;\pi_2,	&& \gtreqless 0
\end{alignat*}

\subsection{Extra - Soluzione del problema duale}
Riportiamo i tableau iniziale e finale con i quali abbiamo applicato l'algoritmo del simplesso.
\begin{table}[htbp]
\centering
\begin{tabular}{rrcccccc}
 	  & 			&$-\psi$	& $x_1$ & $x_2$ & $x_3$ & $x_4$	& $x^a$\\
$R_0$ & $\OL{c_j}$ 	& \Sc{0} 	& 2 	& 1 	& 0 	& \Sc{0}& 0\\
\cline{3-8}
$R_1$ & $x_3$ 		& \Sc{2} 	& 1		& 1 	& 1 	& \Sc{0}& 0 \\
$R_2$ & $x^a$ 		& \Sc{6} 	& 2		& 3		& 0 	&\Sc{-1}& 1 \\
\end{tabular}
\caption{Tableau iniziale con la \textit{funzione obiettivo originaria}.}
\label{tab:tab418}
\end{table}
\begin{table}[htbp]
\centering
{
	\newcommand{\ut}{$\frac{1}{3}$}
	\newcommand{\dt}{$\frac{2}{3}$}
	\newcommand{\qt}{$\frac{4}{3}$}
\begin{tabular}{rrcccccc}
 	  & 			&$-z$		& $x_1$ & $x_2$ & $x_3$ & $x_4$	& $x^a$\\
$R_0$ & $\OL{c_j}$ 	& \Sc{-2} 	& \qt 	& 0 	& 0 	&\Sc{\ut} & -\ut\\
\cline{3-8}
$R_1$ & $x_2$ 		& \Sc{2} 	& \dt	& 1 	& 0 	&\Sc{-\ut}& \ut \\
$R_2$ & $x^3$ 		& \Sc{0} 	& -\ut	& 0		& 1 	&\Sc{-\ut}& 0 \\
\end{tabular}\caption{Tableau finale.}
}
\label{tab:tab419}
\end{table}
La base iniziale � $\mathcal{B}_1={A_3,A_5}$. Applicando delle semplici sottrazioni, ricaviamo la soluzione del problema duale:
\begin{align*}
\pi_1&=c_3-\OL{c_3}=0 \\
\pi_3&=c_5-\OL{c_5}=\frac{1}{3}
\end{align*}
Perci�, la soluzione del problema duale � il vettore:
$$
\pi'=
\begin{bmatrix}
0 & \frac{1}{3}
\end{bmatrix}
$$
Per verificare la correttezza dei calcoli, applichiamo la soluzione alla funzione obiettivo del problema duale:
$$
\xi(0,\frac{1}{3})=2(0)+0+6(\frac{1}{3})=2
$$
Il risultato �, come atteso, lo stesso del problema primale.

\section{Esercizio 3}

Sia dato il seguente modello matematico di un problema di LP:
\begin{align*}
\min z	&= -x_1-x_2 \\
\st\;\;	& x_2 \leq 1\\
		& -x_1+x_2\geq 2\\
	  	& x_1,x_2 \geq 0
\end{align*}
\begin{itemize}
\item Si rappresenti accuratamente il problema in forma grafica;
\item Si ricavi la forma standard;
\item Si risolva il problema tramite il metodo del simplesso a due fasi utilizzando il minor numero di variabili artificiali.
\end{itemize}

\subsection{Problema in forma grafica}

In figura \vref{fig:graph6} � rappresentato graficamente il problema presentato. Il gradiente della funzione obiettivo vale:
\begin{equation*}
\nabla(z)=\left(\frac{\partial z}{\partial x},\frac{\partial z}{\partial y}\right) = \left( -1,-1 \right) \\
\end{equation*}
Poich� questa volta il problema � rappresentato sotto forma di minimo, saremo interessati alla limitazione del politopo nella direzione \textbf{opposta} al gradiente. A tal fine utilizziamo una funzione obiettivo ausiliaria $\varphi$ tale che:
\begin{equation*}
\varphi = -z \Rightarrow \nabla\varphi = -\nabla z = (-2,-1)
\end{equation*}
In giallo � rappresentata l'area tra i due vincoli lineari di disuguaglianza. Notiamo che quest'area si estende nel secondo e nel terzo quadrante, perci� non rispetta i vincoli di non minoranza delle singole variabili. Ci aspettiamo una soluzione impossibile dal simplesso.
\begin{figure}[htbp]
\centering
\begin{tikzpicture}
\begin{axis}
[axis lines=middle, axis equal, enlargelimits, xlabel=$x_1$, ylabel=$x_2$,
 every axis x label/.style={
    at={(ticklabel* cs:1.01)},
    anchor=west,
 },
 every axis y label/.style={
    at={(ticklabel* cs:1.01)},
    anchor=south,
 },%xtick={1,2,3}
 ]
    \path[name path=AX] 
        (axis cs:\pgfkeysvalueof{/pgfplots/xmin},0)--
        (axis cs:\pgfkeysvalueof{/pgfplots/xmax},0);
    \path[name path=AY] 
        (axis cs:0,\pgfkeysvalueof{/pgfplots/ymin})--
        (axis cs:0,\pgfkeysvalueof{/pgfplots/ymax});
    \path[name path=UP]
    	(axis cs:\pgfkeysvalueof{/pgfplots/xmin},\pgfkeysvalueof{/pgfplots/ymax})--
    	(axis cs:\pgfkeysvalueof{/pgfplots/xmax},\pgfkeysvalueof{/pgfplots/ymax});

\addplot
[domain=-3.01:3.01, samples=10, thick, blue, name path=yx2]
{x+2} node [pos=0.6,pin={135:{\color{blue}$-x_1+x_2=2$}}, inner sep=0pt] {};
\addplot
[domain=-3.01:3.01, samples=10, thick, red, name path=y1]
{1} node [pos=0.5, pin={75:{\color{red}$x_2=1$}}, inner sep=0pt] {};
\addplot[thick, fill=yellow, fill opacity=0.5] fill between [of=yx2 and y1, soft clip={domain=-3:-1}];
%\addplot[white] fill between [of=xy2 and AX];
%\addplot[pattern=north east lines, pattern color=red!10] fill between [reverse=true, of=AX and UP, soft clip={domain=0:5}];
\addplot[pattern=north west lines, pattern color=blue!10] fill between [of=yx2 and UP];
\addplot[pattern=vertical lines, pattern color=red!10] fill between [of=y1 and AX];
\intne{y1}{yx2}{$\alpha$}{alp};
%\node at (axis cs:1.8,0.5) {$P$};
\addplot[-latex, thick] coordinates
           {(0,0) (1/1.414,1/1.414)} node [pos=.3, anchor=south, label={45:{\small $\nabla\varphi$}}] {};
\end{axis}
\end{tikzpicture}
\caption{Rappresentazione cartesiana del problema di programmazione lineare}
\label{fig:graph6}
\end{figure}

\subsection{Forma standard}

Aggiungendo una variabile slack e una variabile surplus, il problema in forma standard si presenta cos�:
\begin{alignat*}{6}
&\min z = \quad && -x_1 \quad\; && -x_2 \quad\;\; && \qquad\qquad && \qquad\qquad && \\
&\;\st  &&				&&+x_2	 		&&+\pmb{x_3}	&&		 		&&=1\\
&	 	&&-x_1			&&+x_2			&&				&& -\pmb{x_4}	&&=2\\
&		&&\quad\; x_1,	&&\quad\; x_2,	&&\quad\; x_3,	&&\quad\; x_4,	&&\geq 0
\end{alignat*}

\subsection{Risoluzione tramite tableau}

\begin{table}[htbp]
\centering
\begin{tabular}{rccccc}
			&$-z$ & $x_1$ & $x_2$ & $x_3$ & $x_4$ \\
$\OL{c_j}$ 	& \Sc{0} 	& -1 	& -1 	& 0 	& 0 \\
\cline{2-6}
$R_1$ 		& \Sc{1} 	& 0 	& 1 	& 1 	& 0\\
$R_2$		& \Sc{2} 	& -1 	& 1		& 0 	& -1 \\
\end{tabular}
\caption{Tableau iniziale.}
\label{tab:tab420}
\end{table}

In tabella \vref{tab:tab420} il tableau ricavato dal nostro problema. Non abbiamo nessuna sottomatrice identit� a disposizione da utilizzare come base ammissibile quindi ricorriamo alla \textbf{fase 1 del metodo del simplesso} per ottenere una BFS di partenza.

\subsubsection{Fase 1 - aggiunta di variabili artificiali}

Manca solo la seconda colonna della matrice identit� con cui formare la BFS di partenza e la introdurremo con l'\textit{unica} variabile artificiale $x^a$, trasformando il secondo vincolo in:
$$
-x_1 + x_2 - x_4 + x^a = 2
$$
Il nostro scopo, dopo l'introduzione di $x^a$, sar� quello di \textbf{eliminarla} dalla base. Per far ci� bisogna fare in modo che questa valga zero e quindi introduciamo, a tale scopo, una nuova funzione obiettivo da minimizzare $\psi$ tale che:
$$
\psi = \sum_{i=1}^{n'}x_i^a = x^a
$$
Scriviamo il nuovo tableau in tabella \vref{tab:tab421} e applichiamo il simplesso per ottimizzare la nostra funzione $\psi$.
\begin{table}[htbp]
\centering
\begin{tabular}{rrcccccc}
 	  & 			&$-\psi$	& $x_1$ & $x_2$ & $x_3$ & $x_4$	& $x^a$\\
$R_0$ & $\OL{c_j}$ 	& \Sc{0} 	& 0 	& 0 	& 0 	& \Sc{0}& 1\\
\cline{3-8}
$R_1$ & $x_3$ 		& \Sc{1} 	& 0		& 1 	& 1 	& \Sc{0}& 0 \\
$R_2$ & $x^a$ 		& \Sc{2} 	& -1	& 1		& 0 	&\Sc{-1}& 1 \\
\end{tabular}
\caption{Nuovo tableau con la variabile artificiale $x^a$.}
\label{tab:tab421}
\end{table}
Abbiamo una sottomatrice identit� formata dalla base:
$$
\mathcal{B}=\{A_3,A_5\}
$$
Per avere a avere a disposizione i valori delle coordinate della BFS del nuovo problema, � necessario che:
$$
y_{ij}=0 \quad \forall i,j:A_j\in\mathcal{B},i\neq j
$$
Condizione vera per ogni valore tranne $y_{05}$ che provvediamo ad annullare tramite l'operazione elementare di riga:
$$
R_0\leftarrow R_0 - R_2
$$
Nel nuovo tableau in figura \vref{tab:tab422} dobbiamo scegliere su quale colonna fare pivoting. L'unica colonna con $\OL{c_j}<0$ � $A_2$ e cercheremo qui l'elemento pivot.
Per scegliere su quale elemento fare \textbf{pivoting}, dobbiamo ottenere il valore di $y_{\ell 2}$ tale che:
$$
\vartheta_{\max}=\min_{i:y_{i2}>0}\frac{y_{i0}}{y_{i2}}=\frac{y_{i0}}{y_{\ell 2}}
$$
Perci�, operando con gli elementi nel tableau:
$$
\vartheta_{\max}=\min\left(\frac{1}{1},\frac{2}{1}\right)=\frac{1}{1}=\frac{y_{10}}{\pmb{y_{12}}}
$$
Faremo pivoting sull'elemento $y_{12}$ (cerchiato in tabella). Il nostro scopo � ora far comparire uno 0 nella colonna dell'elemento pivot in tutte le righe tranne quella in cui si trova l'elemento pivot e far comparire un 1 in quest'ultima.
\begin{table}[htbp]
\centering
\begin{tabular}{rrcccccc}
 	  & 			&$-\psi$	& $x_1$ & $x_2$ & $x_3$ & $x_4$	& $x^a$\\
$R_0$ & $\OL{c_j}$ 	& \Sc{-2} 	& 1 	& -1 	& 0 	& \Sc{1}& 0\\
\cline{3-8}
$R_1$ & $x_2$ 		& \Sc{1} 	& 0		& \C{1}	& 1 	& \Sc{0}& 0 \\
$R_2$ & $x^a$ 		& \Sc{2} 	& -1	& 1		& 0 	&\Sc{-1}& 1 \\
\end{tabular}
\caption{Pivoting su $y_{12}$. $A_2$ entra in base e $A_3$ esce.}
\label{tab:tab422}
\end{table}
Poich� $y_{12}=1$ non c'� nulla da fare su $R_1$. Applichiamo le operazioni elementari di riga al nostro tableau come segue:
\begin{align*}
R_0&\leftarrow R_0 + R_1; \\
R_2&\leftarrow R_2 - R_1.
\end{align*}
Il nostro nuovo tableau diventa quindi quello in tabella \vref{tab:tab423}.
\begin{table}[htbp]
\centering
\begin{tabular}{rrcccccc}
 	  & 			&$-\psi$	& $x_1$ & $x_2$ & $x_3$ & $x_4$	& $x^a$\\
$R_0$ & $\OL{c_j}$ 	& \Sc{-1} 	& 1 	& 0 	& 1 	& \Sc{1}& 0\\
\cline{3-8}
$R_1$ & $x_2$ 		& \Sc{1} 	& 0		& 1 	& 1 	& \Sc{0}& 0 \\
$R_2$ & $x^a$ 		& \Sc{1} 	& -1	& 0		& -1	&\Sc{-1}& 1 \\
\end{tabular}
\caption{Secondo tableau. $x^a$ ancora in base e $\psi\neq 0$.}
\label{tab:tab423}
\end{table}
Siamo nella soluzione ottima ma:
\begin{itemize}
\item la variabile artificiale $x^a$ � ancora fuori base;
\item la soluzione ottima non � nulla.
\end{itemize}
Non vale la pena sprecare energie per far uscire dalla base la variabile artificiale. Qualunque operazione facessimo, il valore della soluzione ottima non potrebbe migliorare e quindi non potrebbe annullarsi. Ci� significa che non esistono soluzioni ammissibili al nostro problema.
\subsection{Soluzione del problema}
Il problema non ha soluzione in quanto nella fase 1 del metodo del simplesso non siamo riusciti a trovare una BFS che non coinvolga variabili artificiali.

\chapter{15/04/2014}

In questo capitolo, riguardante la terza esercitazione, verranno svolti alcuni dei punti mancanti dell'esercitazione precedente. Per essere precisi, tutti quelli riguardanti il \textbf{vincolo di interezza} sulle variabili, cio� i problemi ILP. Non saranno riportati, ovviamente, gli interi esercizi ma solo i tableau finali con i quali partire per eseguire il simplesso duale. Verranno comunque creati dei riferimenti ipertestuali negli esercizi precedenti che rimanderanno alle soluzioni che di seguito apporr�.

\section{Esercizio 4}

In questa sezione riprenderemo da dove eravamo rimasti in \vref{sec:es4ILP}, risolvendo il problema ILP prima con il metodo dei \textbf{tagli di Gomory} e poi con il metodo \textbf{branch and bound} (il primo in realt� non � richiesto dalla traccia, ma lo proveremo ugualmente come utile esercitazione).
Prima di ci�, � bene ricordare alcuni concetti teorici.

\subsection{Richiami (blandi, sempre blandi) di teoria}
\subsubsection{Tagli di Gomory}
Ricordiamo dato un problema ILP con \textbf{soluzione del rilassamento continuo $x^*$}, \textbf{eseguire un taglio} significa applicare un \textbf{vincolo} tale che:
\begin{enumerate}
\item \textbf{elimini} una parte della regione ammissibile \textbf{contente $x^*$};
\item \textbf{non elimini} nessuna soluzione intera ammissibile.
\end{enumerate}
Definendo:
\begin{align*}
\mathcal{B} & \qquad\textbf{base ottima} \\
f_{ij}=y_{ij}-\left\lfloor y_{ij} \right\rfloor & \qquad\textbf{parte frazionaria di $y_{ij}$}
\end{align*}
Un \textbf{taglio di Gomory} � tale che:
$$
\sum_{A_j \notin \mathcal{B}} f_{ij}x_j \geq f_{i0}
$$
Ove la riga $i$ � scelta arbitrariamente tra quelle per cui $y_{i0}\notin\mathbb{N}$.
Pu� essere dimostrato (ma non lo faremo qui) che questo tipo di taglio rispetta i due punti citati in precedenza.
\subsubsection{Branch and bound (da rivedere)}
Come dal nome del metodo, questo metodo si compone di due parti: \textbf{branching} e \textbf{bounding}.
\paragraph{Branching}
L'operazione di \textbf{branching} consiste nell'imporre due \textbf{vincoli mutualmente esclusivi ed esaustivi} su una variabile $x_j$, data $x_j^*$ la componente $j$-esima della soluzione del rilassamento continuo:
$$
x_j \leq \left\lfloor x_j^* \right\rfloor \qquad\veebar\qquad x_j \geq \left\lceil x_j^* \right\rceil
$$
Unendo al problema originario alternativamente questi due vincoli, si ottengono due nuovi sottoproblemi. La soluzione migliore tra i due sar� anche la migliore in assoluto. Per risolvere i due nuovi sottoproblemi si pu� applicare ricorsivamente un nuovo branching.
\paragraph{Bounding}
Ogni volta che si ottiene un nuovo sottoproblema da quello originario, si risolve nuovamente il rilassamento continuo.
Il rilassamento continuo di un problema costituisce un \textbf{lower bound}, poich� nessuna soluzione pu� essere migliore di quella offerta dal rilassamento continuo. In particolare, si pu� considerare come lower bound anche la sua \textbf{parte intera}.
Nel momento in cui uno dei sottoproblemi offre una soluzione intera uguale a quella del rilassamento continuo, si pu� evitare di risolvere \textbf{tutti i sottoproblemi non ancora risolti il cui lower bound � maggiore o uguale alla soluzione appena ottenuto} in quanto nella migliore delle ipotesi potremmo trovare solo una soluzione uguale a quella appena trovata, ma non una migliore.
\paragraph{Rappresentazione}
Il metodo migliore per rappresentare la soluzione tramite branch and bound � un albero binario.

\begin{figure}[ht!]
\centering
\begin{tikzpicture}[edge from parent/.style={draw=black,-latex},
					level/.style={sibling distance=50mm, level distance=20mm},
					cross/.style={path picture={ 
 						\draw[black]
						(path picture bounding box.south east) -- (path picture bounding box.north west) (path picture bounding box.south west) -- (path picture bounding box.north east);}}
					]
\node [circle, draw, label=east:{$L_B=-9$}] (z) {$P_0$}
	child {node [circle, draw, label=east:{$L_B=-8$}] (a) {$P_1$}
			child {node [circle, draw, label=east:{$L_B=-8$}, label=south:{$-z=-8$}] (b) {$P_2$}
				   edge from parent node[above left] {$x_2\leq 3$}
				   }
			edge from parent node[left] {$x_1\leq 1$}
		   } 
	child {node [circle, draw, cross] (b) {$P_3$}
			edge from parent node[above right, label=east:{$L_B=-8$}] {$x_1\geq 2$}
			}
	;
\end{tikzpicture}
\end{figure}
Nell'esempio, il problema $P_2$ ha la soluzione del rilassamento continua uguale al lower bound del problema padre. Questo rende inutile esplorare l'altro figlio del problema $P_1$, in quanto non potremmo avere alcuna soluzione pi� bassa di $-8$, come indicato dal suo lower bound.
Poich� il lower bound del problema $P_0$ � migliore della soluzione attualmente disponibile, ha senso esplorare il suo altro figlio. Nel nostro esempio, per�, il figlio $P_3$ ha un lower bound uguale a $-8$ e non c'� pi� possibilit� di trovare una soluzione migliore di quella trovata in precedenza. Si dice, in questo caso, che il nodo $P_2$ \textbf{uccide} il nodo $P_3$ (sul quale abbiamo all'uopo apposto una croce).
\textit{Noto che � difficile spiegare in maniera generale il funzionamento del metodo branch and bound. Si spera perci� che risulti pi� chiaro applicandolo praticamente negli esercizi.}

\subsection{ILP - Tagli di Gomory}

Riportiamo in tabella \vref{tab:tab411} il tableau finale della soluzione del problema LP.
\begin{table}[htbp]
\centering
{
	\newcommand{\sm}{$\frac{7}{2}$}
	\newcommand{\nm}{$\frac{9}{2}$}
	\newcommand{\um}{$\frac{1}{2}$}
	\newcommand{\tm}{$\frac{3}{2}$}
\begin{tabular}{rrccccccc}
 	  & 			&$-\varphi$	& $x_1$ & $x_2$ & $x_3$ & $x_4$ & $x_5$	& $x^a$\\
$R_0$ & $\OL{c_j}$ 	& \Sc{8} 	& 0 	& 0 	& 1 	& 0 	& \Sc{0}& 0\\
\cline{3-9}
$R_1$ & $x_2$ 		& \Sc{\sm}	& 0		& 1		& \um 	& \um 	& \Sc{0}& -\um \\
$R_2$ & $x_1$ 		& \Sc{\nm} 	& 1		& 0		& \um	& -\um 	& \Sc{0}& \um \\
$R_3$ & $x_5$ 		& \Sc{\tm}	& 0 	& 0 	& -\um 	& \um 	& \Sc{1}& -\um \\
\end{tabular}
}
\caption{Tableau finale del problema LP. Rappresenta il rilassamento continuo del problema ILP.}
\label{tab:tab411}
\end{table}
Le righe $R_1$,$R_2$ ed $R_3$ sono valide candidate per  applicare il taglio di Gomory. Scegliamo la riga $R_1$:
$$
\frac{1}{2}x_3+\frac{1}{2}x_4\geq\frac{1}{2}
$$
Portiamo in forma standard il vincolo moltiplicando per $-1$ e aggiungendo una variabile slack $s$:
$$
-\frac{1}{2}x_3-\frac{1}{2}x_4+s=\frac{1}{2}
$$
Il vincolo � abbastanza agevole da inserire nel tableau come in tabella \vref{tab:tab412} (� stata eliminata la variabile artificiale, ormai inutile).
\begin{table}[htbp]
\centering
{
	\newcommand{\sm}{$\frac{7}{2}$}
	\newcommand{\nm}{$\frac{9}{2}$}
	\newcommand{\um}{$\frac{1}{2}$}
	\newcommand{\mum}{$-\frac{1}{2}$}
	\newcommand{\tm}{$\frac{3}{2}$}
\begin{tabular}{rrccccccc}
 	  & 			&$-\varphi$	& $x_1$ & $x_2$ & $x_3$ & $x_4$ & $x_5$	& $s$\\
$R_0$ & $\OL{c_j}$ 	& \Sc{8} 	& 0 	& 0 	& 1 	& 0 	& \Sc{0}& 0\\
\cline{3-9}
$R_1$ & $x_2$ 		& \Sc{\sm}	& 0		& 1		& \um 	& \um 	& \Sc{0}& 0 \\
$R_2$ & $x_1$ 		& \Sc{\nm} 	& 1		& 0		& \um	& \mum 	& \Sc{0}& 0 \\
$R_3$ & $x_5$ 		& \Sc{\tm}	& 0 	& 0 	& \mum 	& \um 	& \Sc{1}& 0 \\
\cline{3-8}
$R_4$ & $s$			& \Sc{\mum}	& 0		& 0 	& \mum	&\C{\mum}& 0	& 1
\end{tabular}
}
\caption{Applicazione del primo taglio di Gomory.}
\label{tab:tab412}
\end{table}
� evidente che nel tableau � ancora presente la sottomatrice identit� e che la soluzione rispetta il \textit{criterio di ottimalit�} essendo tutti i $\OL{c_j}$ non negativi. L'unico intoppo � rappresentato dal valore della variabile $s$, che � negativo.
In altri termini, siamo in una soluzione ottima ma non ammissibile. Lo strumento che fa al caso nostro ora � il \textbf{simplesso duale}, che ci porter� da una base ottima ad una nuova base sempre ottima ma pi� vicina all'ammissibilit�.
Il simplesso duale opera come il simplesso primale ma invertendo righe e colonne. Perci�, dovremo scegliere tra tutti gli elementi della \textit{riga} $i$-esima con $y_{i0}<0$ quello su cui fare pivot. A tal scopo, ridefiniamo la grandezza $\vartheta$ in tal modo:
$$
\vartheta=\max_{j>0:y_{ij}<0}\left(\frac{y_{0j}}{y_{ij}}\right)=\frac{y_{0s}}{y_{is}}
$$
L'elemento $y_{is}$ sar� quello di pivoting.
Nel nostro caso:
$$
\vartheta=\max\left(\frac{1}{-\frac{1}{2}},\frac{0}{-\frac{1}{2}}\right)=\frac{0}{-\frac{1}{2}}=\frac{y_{04}}{\pmb{y_{44}}}
$$
Faremo pivoting sull'elemento $y_{44}$, cerchiato in tabella \vref{tab:tab412}. Le operazioni da attuare sono le consuete operazioni elementari di riga:
\begin{align*}
R_1&\leftarrow R_1+R_4 \\
R_2&\leftarrow R_1-R_4 \\
R_3&\leftarrow R_3+R_4 \\
R_4&\leftarrow -2R_4
\end{align*}
Ci� che ne risulta � il tableau in tabella \vref{tab:tab413}.
\begin{table}
\centering
{
	\newcommand{\sm}{$\frac{7}{2}$}
	\newcommand{\nm}{$\frac{9}{2}$}
	\newcommand{\um}{$\frac{1}{2}$}
	\newcommand{\mum}{$-\frac{1}{2}$}
	\newcommand{\tm}{$\frac{3}{2}$}
\begin{tabular}{rrccccccc}
 	  & 			&$-\varphi$	& $x_1$ & $x_2$ & $x_3$ & $x_4$ & $x_5$	& $s$\\
$R_0$ & $\OL{c_j}$ 	& \Sc{8} 	& 0 	& 0 	& 1 	& 0 	& \Sc{0}& 0\\
\cline{3-9}
$R_1$ & $x_2$ 		& \Sc{3}	& 0		& 1		& 0 	& 0 	& \Sc{0}& 1 \\
$R_2$ & $x_1$ 		& \Sc{5} 	& 1		& 0		& 1		& 0 	& \Sc{0}& -1 \\
$R_3$ & $x_5$ 		& \Sc{1}	& 0 	& 0 	& -1 	& 0 	& \Sc{1}& 1 \\
\cline{3-8}
$R_4$ & $x_4$		& \Sc{1}	& 0		& 0 	& 1		& 1		& 0		& -2
\end{tabular}
}
\caption{Dopo il taglio di Gomory siamo in una soluzione intera nel vertice $\varepsilon(5,3)$.}
\label{tab:tab413}
\end{table}
La nuova base e la nuova soluzione sono:
\begin{align*}
\mathcal{B}&=\{A_2,A_1,A_5,A_4\} \\
x&=(5,3,0,1,1)
\end{align*}
La soluzione � intera ed il nostro problema di ILP � risolto nel nuovo vertice $\varepsilon(5,3)$. Il valore della soluzione �:
$$
z(\varepsilon)=-\varphi=8
$$
\subsubsection{Rappresentazione grafica}
Per quanto riguarda i tagli di Gomory, la rappresentazione grafica non � agevole e immediata e se ne dar� una rappresentazione solo alla fine.
Il taglio, in forma di disequazione, � il seguente:
$$
\frac{1}{2}x_3+\frac{1}{2}x_4\geq\frac{1}{2}
$$
Ovviamente non � rappresentabile, in questa forma algebrica, sul piano cartesiano. Le variabili $x_3$ e $x_4$ sono rispettivamente una variabile slack e una variabile surplus, perci� possiamo fare in modo di sostituirle con le variabili reali $x_1$ e $x_2$ operando algebricamente sui vincoli in forma standard che le introducono:
$$
\begin{cases}
x_1 + x_2 + x_3 = 8 \\
x_1 - x_2 - x_4 = 1 
\end{cases}
\Rightarrow
\begin{cases}
x_3 = 8 - x_1 - x_2 \\
x_4 = -1 + x_1 - x_2
\end{cases}
$$
Sostituiamo le due variabili nella disequazione:
$$
\frac{1}{2}(8-x_1-x_2)+\frac{1}{2}(-1+x_1-x_2)\geq\frac{1}{2} \Rightarrow x_2 \leq 3
$$
Cos� espresso il vincolo � ora facilmente rappresentabile graficamente come in figura \vref{fig:graph42}.
\begin{figure}[htbp]
\centering
\begin{tikzpicture}
\begin{axis}
[axis lines=middle, axis equal, enlargelimits, xlabel=$x_1$, ylabel=$x_2$,
 every axis x label/.style={
    at={(ticklabel* cs:1.01)},
    anchor=west,
 },
 every axis y label/.style={
    at={(ticklabel* cs:1.01)},
    anchor=south,
 },]
    \path[name path=AX] 
        (axis cs:\pgfkeysvalueof{/pgfplots/xmin},0)--
        (axis cs:\pgfkeysvalueof{/pgfplots/xmax},0);
    \path[name path=AY] 
        (axis cs:0,\pgfkeysvalueof{/pgfplots/ymin})--
        (axis cs:0,\pgfkeysvalueof{/pgfplots/ymax});
    \path[name path=UP]
    	(axis cs:\pgfkeysvalueof{/pgfplots/xmin},\pgfkeysvalueof{/pgfplots/ymax})--
    	(axis cs:\pgfkeysvalueof{/pgfplots/xmax},\pgfkeysvalueof{/pgfplots/ymax});
\dotgrid{0}{8}{0}{4}{black!50}
\addplot
[domain=0:8, samples=10, thick, blue, name path=xy8] {-x+8};
\addplot
[domain=0:8, samples=10, thick, red, name path=xy1] {x-1};
\addplot
[domain=0:8, samples = 10, thick, purple, name path=x6] (6,x);
\addplot
[domain=0:8, samples=10, very thick, green, name path=gomory]
{3} node [pos=0.8, anchor=south, pin={75:{\color{green}$x_2\leq 3$}}, inner sep= 0pt] {};
\addplot[thick, fill=yellow, fill opacity=0.5] fill between [of=xy1 and AX, soft clip={domain=1:6}];
\addplot[white] fill between [of=xy8 and UP];
\addplot[white] fill between [of=x6 and AX];
\addplot[white] fill between [of=gomory and UP];
%\addplot[pattern=north east lines, pattern color=blue!10] fill between [of=xy8 and AX];
\ints{AX}{xy1}{$\alpha$}{alp};
\ints{xy1}{xy8}{$\beta$}{bet};
\intw{xy8}{x6}{$\gamma$}{gam};
\intnw{x6}{AX}{$\delta$}{del};
\intne{gomory}{xy8}{$\varepsilon$}{eps};
\node at (axis cs:4.5,1.5) {$P'$};
    \path[name path=GOM] 
        (axis cs:\pgfkeysvalueof{/pgfplots/xmin}, 2.5)--
        (axis cs:\pgfkeysvalueof{/pgfplots/xmax}, 2.5);
\addplot[pattern=north east lines, pattern color=green!30] fill between [of=gomory and GOM, soft clip={domain=0:8}];
\addplot[-latex, thick] coordinates
           {(0,0) (1/1.414,1/1.414)} node [pos=.3, anchor=south, label={45:{\small $\nabla z$}}] {};
\end{axis}
\end{tikzpicture}
\caption{Rappresentazione cartesiana del problema di programmazione lineare}
\label{fig:graph42}
\end{figure}
In giallo � rappresentato il nuovo politopo $P'$ risultato del vecchio politopo $P$ e del taglio di Gomory applicato.

\subsection{ILP - Branch and bound}
Il problema di LP iniziale verr� denominato $P_0$. La soluzione del rilassamento continuo $x^*=\left(\frac{9}{2},\frac{7}{2}\right)$ � $-z^*=-8$, il che imposter� il nostro lower bound a $L_B=\left\lceil -8 \right\rceil = -8$.
\begin{figure}[ht!]
\centering
\begin{tikzpicture}[edge from parent/.style={draw=black,-latex},
					level/.style={sibling distance=50mm, level distance=20mm},
					cross/.style={path picture={ 
 						\draw[black]
						(path picture bounding box.south east) -- (path picture bounding box.north west) (path picture bounding box.south west) -- (path picture bounding box.north east);}}
					]
\node [circle, draw, label=east:{$L_B=-8$}] (z) {$P_0$}
	;
\end{tikzpicture}
\end{figure}

Iniziando dalla variabile $x_1$ possiamo prendere in considerazione due diversi sottoproblemi derivati dall'aggiunta di uno dei due seguenti vincoli:
\begin{align*}
x_1&\geq\left\lceil x^*_1 \right\rceil &\Rightarrow x_1&\geq 5\\
x_1&\leq\left\lfloor x^*_1 \right\rfloor &\Rightarrow x_1&\leq 4
\end{align*}
Per convincerci che considerando questi due vincoli non si esclude nessuna soluzione e che sono mutuamente esclusivi, si faccia riferimento al grafico \vref{fig:graph43} e osservando che le due aree colorate in ciano e in verde non escludono nessuno dei punti interi.
\begin{figure}[htbp]
\centering
\begin{tikzpicture}
\begin{axis}
[axis lines=middle, axis equal, enlargelimits, xlabel=$x_1$, ylabel=$x_2$,
 every axis x label/.style={
    at={(ticklabel* cs:1.01)},
    anchor=west,
 },
 every axis y label/.style={
    at={(ticklabel* cs:1.01)},
    anchor=south,
 },]
    \path[name path=AX] 
        (axis cs:\pgfkeysvalueof{/pgfplots/xmin},0)--
        (axis cs:\pgfkeysvalueof{/pgfplots/xmax},0);
    \path[name path=AY] 
        (axis cs:0,\pgfkeysvalueof{/pgfplots/ymin})--
        (axis cs:0,\pgfkeysvalueof{/pgfplots/ymax});
    \path[name path=UP]
    	(axis cs:\pgfkeysvalueof{/pgfplots/xmin},\pgfkeysvalueof{/pgfplots/ymax})--
    	(axis cs:\pgfkeysvalueof{/pgfplots/xmax},\pgfkeysvalueof{/pgfplots/ymax});
\dotgrid{0}{8}{0}{4}{black!50}
\addplot
[domain=4:8, samples=10, thick, blue, name path=xy8] {-x+8};
\addplot
[domain=1:5, samples=10, thick, red, name path=xy1] {x-1};
\addplot
[domain=0:3, samples = 10, thick, purple, name path=x6] (6,x);
\addplot
[domain=0:4, samples=10, thick, cyan, name path=bb1]
(4,x) node [pos=0, anchor=south, pin={-135:{\color{cyan}$x_1\leq 4$}}, inner sep= 0pt] {};
    \path[name path=bb1i] (axis cs:3.75, 0)--(axis cs:3.75, 4);
\addplot
[domain=0:4, samples=10, thick, green, name path=bb2]
(5,x) node [pos=0, anchor=south, pin={-75:{\color{green}$x_2\geq 5$}}, inner sep= 0pt] {};
    \path[name path=bb2i] (axis cs:5.25, 0)--(axis cs:5.25, 4);
\addplot[thick, fill=cyan, fill opacity=0.2] fill between [of=bb1 and xy1];
\addplot[white] fill between [of=xy1 and UP];
\addplot[thick, fill=green, fill opacity=0.2] fill between [of=bb2 and x6];
\addplot[white] fill between [of=xy8 and UP];
%\addplot[pattern=north east lines, pattern color=blue!10] fill between [of=xy8 and AX];
\ints{AX}{xy1}{$\alpha$}{alp};
\inte{xy1}{xy8}{$\beta\equiv x^*$}{bet};
\intw{xy8}{x6}{$\gamma$}{gam};
\intnw{x6}{AX}{$\delta$}{del};
\inte{gomory}{xy8}{$\varepsilon$}{eps};
\node at (axis cs:3.5,1.5) {$P_2$};
\node at (axis cs:5.5,1.5) {$P_1$};
\addplot[pattern=north east lines, pattern color=cyan] fill between [of=bb1 and bb1i];
\addplot[pattern=north east lines, pattern color=green] fill between [of=bb2 and bb2i];
\addplot[-latex, thick] coordinates
           {(0,0) (1/1.414,1/1.414)} node [pos=.3, anchor=south, label={45:{\small $\nabla z$}}] {};
\end{axis}
\end{tikzpicture}
\caption{Rappresentazione cartesiana dei due sottoproblemi di $P_0$.}
\label{fig:graph43}
\end{figure}
Esploriamo ora il figlio $P_1$ del problema $P_0$, corrispondente al vincolo $x_1\geq\left\lceil x^*_1 \right\rceil = 5$.
\begin{figure}[ht!]
\centering
\begin{tikzpicture}[edge from parent/.style={draw=black,-latex},
					level/.style={sibling distance=50mm, level distance=20mm},
					cross/.style={path picture={ 
 						\draw[black]
						(path picture bounding box.south east) -- (path picture bounding box.north west) (path picture bounding box.south west) -- (path picture bounding box.north east);}}
					]
\node [circle, draw, label=east:{$L_B=-8$}] (z) {$P_0$}
	child {node [circle, draw] (a) {$P_1$}
			edge from parent node[left] {$x_1\geq 5$}
		   }
	;
\end{tikzpicture}
\end{figure}

In \textit{forma standard} il vincolo pu� essere cos� espresso:
$$
-x_1 + s = -5
$$
In questa formula risulta molto agevole l'introduzione nel tableau iniziale di $P_0$ (tabella \vref{tab:tab411}) come in tabella \vref{tab:tab414}.
\begin{table}[htbp]
\centering
{
	\newcommand{\sm}{$\frac{7}{2}$}
	\newcommand{\nm}{$\frac{9}{2}$}
	\newcommand{\um}{$\frac{1}{2}$}
	\newcommand{\mum}{$-\frac{1}{2}$}
	\newcommand{\tm}{$\frac{3}{2}$}
\begin{tabular}{rrccccccc}
 	  & 			&$-\varphi$	& $x_1$ & $x_2$ & $x_3$ & $x_4$ & $x_5$	& $s$\\
$R_0$ & $\OL{c_j}$ 	& \Sc{8} 	& 0 	& 0 	& 1 	& 0 	& \Sc{0}& 0\\
\cline{3-9}
$R_1$ & $x_2$ 		& \Sc{\sm}	& 0		& 1		& \um 	& \um 	& \Sc{0}& 0 \\
$R_2$ & $x_1$ 		& \Sc{\nm} 	& 1		& 0		& \um	& \mum 	& \Sc{0}& 0 \\
$R_3$ & $x_5$ 		& \Sc{\tm}	& 0 	& 0 	& \mum 	& \um 	& \Sc{1}& 0 \\
\cline{3-8}
$R_4$ & $s$			& \Sc{-5}	& -1	& 0 	& 0		& 0 	& 0		& 1
\end{tabular}
}
\caption{Sottoproblema $P_1$.}
\label{tab:tab414}
\end{table}
Poich� $y_{41}=-5$ rovina la nostra matrice identit� in base, operiamo la seguente operazione elementare di riga:
$$
R_4\leftarrow R_4 + R_2
$$
Si giunge quindi al tableau in tabella \vref{tab:tab415}.
\begin{table}[htbp]
\centering
{
	\newcommand{\sm}{$\frac{7}{2}$}
	\newcommand{\nm}{$\frac{9}{2}$}
	\newcommand{\um}{$\frac{1}{2}$}
	\newcommand{\mum}{$-\frac{1}{2}$}
	\newcommand{\tm}{$\frac{3}{2}$}
\begin{tabular}{rrccccccc}
 	  & 			&$-\varphi$	& $x_1$ & $x_2$ & $x_3$ & $x_4$ & $x_5$	& $s$\\
$R_0$ & $\OL{c_j}$ 	& \Sc{8} 	& 0 	& 0 	& 1 	& 0 	& \Sc{0}& 0\\
\cline{3-9}
$R_1$ & $x_2$ 		& \Sc{\sm}	& 0		& 1		& \um 	& \um 	& \Sc{0}& 0 \\
$R_2$ & $x_1$ 		& \Sc{\nm} 	& 1		& 0		& \um	& \mum 	& \Sc{0}& 0 \\
$R_3$ & $x_5$ 		& \Sc{\tm}	& 0 	& 0 	& \mum 	& \um 	& \Sc{1}& 0 \\
\cline{3-8}
$R_4$ & $s$			& \Sc{\mum}	& 0		& 0 	& \um	&\C{\mum}& 0	& 1
\end{tabular}
}
\caption{Sottoproblema $P_1$, secondo tableau.}
\label{tab:tab415}
\end{table}

Si pu� ora procedere con il simplesso duale. L'unico elemento di $R_4$ negativo sul quale � possibile fare pivoting � $y_{44}$. Le operazioni elementari di riga sono, nell'ordine:
\begin{align*}
R_1&\leftarrow R_1 + R_4 \\
R_2&\leftarrow R_2 - R_4 \\
R_3&\leftarrow R_3 + R_4 \\
R_4&\leftarrow -2R_4
\end{align*}
Ne risulta il tableau in tabella \vref{tab:tab416}.
\begin{table}
\centering
{
	\newcommand{\sm}{$\frac{7}{2}$}
	\newcommand{\nm}{$\frac{9}{2}$}
	\newcommand{\um}{$\frac{1}{2}$}
	\newcommand{\mum}{$-\frac{1}{2}$}
	\newcommand{\tm}{$\frac{3}{2}$}
\begin{tabular}{rrccccccc}
 	  & 			&$-\varphi$	& $x_1$ & $x_2$ & $x_3$ & $x_4$ & $x_5$	& $s$\\
$R_0$ & $\OL{c_j}$ 	& \Sc{8} 	& 0 	& 0 	& 1 	& 0 	& \Sc{0}& 0\\
\cline{3-9}
$R_1$ & $x_2$ 		& \Sc{3}	& 0		& 1		& 0 	& 0 	& \Sc{0}& 1 \\
$R_2$ & $x_1$ 		& \Sc{5} 	& 1		& 0		& 0		& 0 	& \Sc{0}& -1 \\
$R_3$ & $x_5$ 		& \Sc{1}	& 0 	& 0 	& 0 	& 0 	& \Sc{1}& 1 \\
\cline{3-8}
$R_4$ & $x_4$		& \Sc{1}	& 0		& 0 	& -1		& 1		& 0		& -2
\end{tabular}
}
\caption{Siamo in una soluzione intera nel vertice $\varepsilon(5,3)$.}
\label{tab:tab416}
\end{table}
La nuova base e la nuova soluzione sono:
\begin{align*}
\mathcal{B}&={A_2,A_1,A_5,A_4} \\
x&=(5,3,0,1,1)
\end{align*}
La soluzione � intera ed il ramo attuale dell'albero � giunto ad una foglia.
\begin{figure}[ht!]
\centering
\begin{tikzpicture}[edge from parent/.style={draw=black,-latex},
					level/.style={sibling distance=50mm, level distance=20mm},
					cross/.style={path picture={ 
 						\draw[black]
						(path picture bounding box.south east) -- (path picture bounding box.north west) (path picture bounding box.south west) -- (path picture bounding box.north east);}}
					]
\node [circle, draw, label=east:{$L_B=-8$}] (z) {$P_0$}
	child {node [circle, draw, label=east:{$L_B=-8$}, label=south:{$-z=-8$}] (a) {$P_1$}
			edge from parent node[left] {$x_1\geq 5$}
		   }
	;
\end{tikzpicture}
\end{figure}

Siamo fortunati perch� l'attuale soluzione � uguale al lower bound del nodo \textit{root}, il che significa che non serve cercare altre soluzioni che potrebbero essere al pi� uguali a quella trovata. Ne conviene infine che il vertice $\varepsilon(5,3)$ � una soluzione ottima del problema ILP. Il valore della soluzione �:
$$
z(\varepsilon)=-\varphi=8
$$

\section{Esercizio 7}
In questa sezione riprenderemo da dove eravamo rimasti in \vref{sec:es7ILP}, risolvendo il problema ILP prima con il metodo dei \textbf{tagli di Gomory} e poi con il metodo \textbf{branch and bound} (il secondo in realt� non � richiesto dalla traccia, ma lo proveremo ugualmente come utile esercitazione).

\subsection{ILP - Tagli di Gomory}
Riportiamo in tabella \vref{tab:tab711} il tableau finale del problema LP.
\begin{table}[htbp]
\centering
{
	\newcommand{\uq}{$\frac{1}{4}$}
	\newcommand{\tq}{$\frac{3}{4}$}
	\newcommand{\qq}{$\frac{15}{4}$}
	\newcommand{\nq}{$\frac{9}{4}$}
	\newcommand{\muq}{$-\frac{1}{4}$}
	\newcommand{\mtq}{$-\frac{2}{4}$}
	\newcommand{\mqq}{$-\frac{15}{4}$}
	\newcommand{\ttq}{$\frac{33}{4}$}

\begin{tabular}{rrcccccc}
 	  & 			&$-\varphi$	& $x_1$ & $x_2$ & $x_3$ & $x_4$ & $x_5$	\\
$R_0$ & $\OL{c_j}$ 	& \Sc{\ttq}	& 0 	& 0 	& 0 	& \qq	& \Sc{\uq} \\
\cline{3-8}
$R_1$ & $x_3$ 		& \Sc{\nq} 	& 0		& 0 	& 1 	& \tq 	& \Sc{\uq} \\
$R_2$ & $x_2$ 		& \Sc{3} 	& 0		& 1		& 0 	& 1 	& \Sc{0} \\
$R_3$ & $x_1$ 		& \Sc{\tq} 	& 1 	& 0 	& 0 	& \mtq 	& \Sc{\muq} \\
\end{tabular}
}
\caption{Tableau finale del problema LP. Rappresenta il rilassamento continuo del problema ILP.}
\label{tab:tab711}
\end{table}
Le righe $R_0$, $R_1$ ed $R_3$ sono valide candidate per un taglio di Gomory, scegliamo quindi la prima disponibile, cio� $R_0$. Il taglio sar�:
$$
\sum_{j\in\{4,5\}} f_{0j}x_j \geq f_{00} \Rightarrow \frac{3}{4}x_4+\frac{1}{4}x_5\geq\frac{1}{4}
$$
Trasformiamolo ora in una forma algebricamente adatta al nostro tableau:
$$
-\frac{3}{4}x_4-\frac{1}{4}x_5+s=-\frac{1}{4}
$$
Aggiungendo questo vincolo si ottiene il tableau in tabella \vref{tab:tab712}.
\begin{table}[htbp]
\centering
{
	\newcommand{\uq}{$\frac{1}{4}$}
	\newcommand{\tq}{$\frac{3}{4}$}
	\newcommand{\qq}{$\frac{15}{4}$}
	\newcommand{\nq}{$\frac{9}{4}$}
	\newcommand{\muq}{$-\frac{1}{4}$}
	\newcommand{\mtq}{$-\frac{2}{4}$}
	\newcommand{\mqq}{$-\frac{15}{4}$}
	\newcommand{\ttq}{$\frac{33}{4}$}

\begin{tabular}{rrccccccc}
 	  & 			&$-\varphi$	& $x_1$ & $x_2$ & $x_3$ & $x_4$ & $x_5$	& $s$\\
$R_0$ & $\OL{c_j}$ 	& \Sc{\ttq}	& 0 	& 0 	& 0 	& \qq	& \Sc{\uq}	& 0 \\
\cline{3-9}
$R_1$ & $x_3$ 		& \Sc{\nq} 	& 0		& 0 	& 1 	& \tq 	& \Sc{\uq}	& 0 \\
$R_2$ & $x_2$ 		& \Sc{3} 	& 0		& 1		& 0 	& 1 	& \Sc{0}	& 0 \\
$R_3$ & $x_1$ 		& \Sc{\tq} 	& 1 	& 0 	& 0 	& \mtq 	& \Sc{\muq}	& 0 \\
\cline{3-8}
$R_4$ & $s$ 		& \Sc{\muq}	& 0 	& 0 	& 0 	& \mtq 	& \C{\muq}	& 1 \\
\end{tabular}
}
\caption{Applicazione del primo taglio di Gomory.}
\label{tab:tab712}
\end{table}
La soluzione attuale rispetto il criterio di ottimalit� ma non � ammissibile poich� $s<0$. Il simplesso duale ci porter� in una nuova base ottima e infine anche ammissibile.
Sceglieremo l'elemento di pivoting $y_{is}$ tale che:
$$
\vartheta=\max_{j>0:y_{ij}<0}\left(\frac{y_{0j}}{y_{ij}}\right)=\frac{y_{0s}}{y_{is}}
$$
Nel nostro caso:
$$
\vartheta=\max\left(\frac{\frac{15}{4}}{-\frac{3}{4}},\frac{\frac{1}{4}}{-\frac{1}{5}}\right)=\max(-5,-1)=-1=\frac{y_{05}}{\pmb{y_{45}}}
$$
Faremo pivoting sull'elemento $y_{45}$, cerchiato in tabella \vref{tab:tab712}. Attuiamo le operazioni elementari di riga:
\begin{align*}
R_0&\leftarrow R_0 + R_4 \\
R_1&\leftarrow R_1 + R_4 \\
R_3&\leftarrow R_3 - R_4 \\
R_4&\leftarrow -4R_4
\end{align*}
Il tableau che segue � quello in tabella \vref{tab:tab713}.
\begin{table}[htbp]
\centering
\begin{tabular}{rrccccccc}
 	  & 			&$-\varphi$	& $x_1$ & $x_2$ & $x_3$ & $x_4$ & $x_5$	& $s$\\
$R_0$ & $\OL{c_j}$ 	& \Sc{8}	& 0 	& 0 	& 0 	& 3		& \Sc{0}	& 1 \\
\cline{3-9}
$R_1$ & $x_3$ 		& \Sc{2} 	& 0		& 0 	& 1 	& 0 	& \Sc{0}	& 1 \\
$R_2$ & $x_2$ 		& \Sc{3} 	& 0		& 1		& 0 	& 1 	& \Sc{0}	& 0 \\
$R_3$ & $x_1$ 		& \Sc{1} 	& 1 	& 0 	& 0 	& 0 	& \Sc{0}	& -1 \\
\cline{3-8}
$R_4$ & $x_5$ 		& \Sc{1}	& 0 	& 0 	& 0 	& 3 	& 1			& -4 \\
\end{tabular}
\caption{Secondo tableau. Soluzione ottima e intera al vertice $\delta(1,3)$.}
\label{tab:tab713}
\end{table}
La soluzione del nuovo problema LP � ottima e intera, quindi � anche la soluzione del problema ILP. La nuova base, la nuova soluzione e il suo valore sono:
\begin{align*}
\mathcal{B}&=\{A_3,A_2,A_1,A_5\} \\
x&=(1,3,2,0,1) \\
z(\delta)&=-\varphi=8
\end{align*}

\subsubsection{Rappresentazione grafica}
Rappresentiamo il taglio di Gomory in funzione delle variabili $x_1$ e $x_2$ ricavandole dai vincoli espressi all'inizio:
$$
\begin{cases}
x_2+x_4=3 \\
4x_1+3x_2-x_5=12
\end{cases}
\Rightarrow
\begin{cases}
x_4=3-x_2 \\
x_5=-12+4x_1+3x_2
\end{cases}
$$
Sostituiamo queste espressioni nel taglio di Gomory espresso sotto forma di disequazione:
$$
\frac{3}{4}x_4+\frac{1}{4}x_5\geq\frac{1}{4}\Rightarrow 3(3-x_2)+(-12+4x_1+3x_2)\geq 1 \Rightarrow x_1\geq 1
$$
In figura \vref{fig:graph71} � rappresentato il problema dopo il taglio.
\begin{figure}[htbp]
\centering
\begin{tikzpicture}
\begin{axis}
[axis lines=middle, axis equal, enlargelimits, xlabel=$x_1$, ylabel=$x_2$,
 every axis x label/.style={
    at={(ticklabel* cs:1.01)},
    anchor=west,
 },
 every axis y label/.style={
    at={(ticklabel* cs:1.01)},
    anchor=south,
 },%xtick={1,2,3}
 ]
    \path[name path=AX] 
        (axis cs:\pgfkeysvalueof{/pgfplots/xmin},0)--
        (axis cs:\pgfkeysvalueof{/pgfplots/xmax},0);
    \path[name path=AY] 
        (axis cs:0,\pgfkeysvalueof{/pgfplots/ymin})--
        (axis cs:0,\pgfkeysvalueof{/pgfplots/ymax});
    \path[name path=UP]
    	(axis cs:\pgfkeysvalueof{/pgfplots/xmin},\pgfkeysvalueof{/pgfplots/ymax})--
    	(axis cs:\pgfkeysvalueof{/pgfplots/xmax},\pgfkeysvalueof{/pgfplots/ymax});

\dotgrid{0}{4}{0}{4}{black!50}
\addplot
[domain=0:4, samples=10, thick, blue, name path=x3] (3,x);
\addplot
[domain=0:4, samples=10, thick, red, name path=y3] {3};
\addplot
[domain=0:3, samples=10, thick, green, name path=4x3y12] {-(4/3)*x + 4};
\addplot
[domain=0:4, samples=10, thick, purple, name path=gomory]
	(1,x) node [pos=0.8, anchor=west, pin={75:{\color{purple}$x_1\geq 1$}}, inner sep= 0pt] {};
\path[name path=GOM]
	(axis cs:1.25,0)--(axis cs:1.25,4);
\addplot[thick, fill=yellow, fill opacity=0.5] fill between [of=4x3y12 and UP, soft clip={domain=0:3}];
\addplot[white] fill between [of=y3 and UP];
\addplot[white] fill between [of=gomory and AY];
\addplot[pattern=north east lines, pattern color=purple] fill between [of=gomory and GOM];
\intne{x3}{AX}{$\alpha$}{alp};
\intsw{y3}{4x3y12}{$\beta$}{bet};
\intne{y3}{x3}{$\gamma$}{gam};
\intsw{gomory}{y3}{$\delta$}{del};
\node at (axis cs:2.5,2) {$P_1$};
\addplot[-latex, thick] coordinates
           {(0,0) (-1/3.16,3/3.16)} node [pos=.3, anchor=north, label={135:{\small $\nabla z$}}] {};
\end{axis}
\end{tikzpicture}
\caption{Rappresentazione cartesiana del problema di programmazione lineare dopo il taglio di Gomory}
\label{fig:graph71}
\end{figure}
In giallo � rappresentato il nuovo politopo $P'$ risultato del vecchio politopo $P$ e del taglio di Gomory applicato.

\subsection{ILP - Branch and bound}
Il problema di LP iniziale verr� denominato $P_0$. La soluzione del rilassamento continuo $x^*=\left(\frac{3}{4},3\right)$ � $-z^*=-\frac{33}{4}$, il che imposter� il nostro lower bound a $L_B=\left\lceil -\frac{33}{4} \right\rceil = -8$.
\begin{figure}[ht!]
\centering
\begin{tikzpicture}[edge from parent/.style={draw=black,-latex},
					level/.style={sibling distance=50mm, level distance=20mm},
					cross/.style={path picture={ 
 						\draw[black]
						(path picture bounding box.south east) -- (path picture bounding box.north west) (path picture bounding box.south west) -- (path picture bounding box.north east);}}
					]
\node [circle, draw, label=east:{$L_B=-8$}] (z) {$P_0$}
	;
\end{tikzpicture}
\end{figure}

Iniziando dalla variabile $x_1$ possiamo prendere in considerazione due diversi sottoproblemi derivati dall'aggiunta di uno dei due seguenti vincoli (si noti che non avrebbe senso iniziare dalla variabile $x_2$, che � gi� intera):
\begin{align*}
x_1&\geq\left\lceil x^*_1 \right\rceil &\Rightarrow x_1&\geq 1\\
x_1&\leq\left\lfloor x^*_1 \right\rfloor &\Rightarrow x_1&\leq 0
\end{align*}
Per convincerci che considerando questi due vincoli non si esclude nessuna soluzione e che sono mutuamente esclusivi, si faccia riferimento al grafico \vref{fig:graph43} e osservando che le due aree colorate in ciano e in verde non escludono nessuno dei punti interi.

\begin{figure}[htbp]
\centering
\begin{tikzpicture}
\begin{axis}
[axis lines=middle, axis equal, enlargelimits, xlabel=$x_1$, ylabel=$x_2$,
 every axis x label/.style={
    at={(ticklabel* cs:1.01)},
    anchor=west,
 },
 every axis y label/.style={
    at={(ticklabel* cs:1.01)},
    anchor=south,
 },%xtick={-2,-1,0,1,2,3}
 ]
    \path[name path=AX] 
        (axis cs:\pgfkeysvalueof{/pgfplots/xmin},0)--
        (axis cs:\pgfkeysvalueof{/pgfplots/xmax},0);
    \path[name path=AY] 
        (axis cs:0,\pgfkeysvalueof{/pgfplots/ymin})--
        (axis cs:0,\pgfkeysvalueof{/pgfplots/ymax});
    \path[name path=UP]
    	(axis cs:\pgfkeysvalueof{/pgfplots/xmin},\pgfkeysvalueof{/pgfplots/ymax})--
    	(axis cs:\pgfkeysvalueof{/pgfplots/xmax},\pgfkeysvalueof{/pgfplots/ymax});

\dotgrid{0}{4}{0}{4}{black!50}
\addplot
[domain=-2:-1, samples=2, white, name path=enlarge] {2};
\addplot
[domain=0:4, samples=10, thick, blue, name path=x3] (3,x);
\addplot
[domain=0:4, samples=10, thick, red, name path=y3] {3};
\addplot
[domain=0:3, samples=10, thick, green, name path=4x3y12] {-(4/3)*x + 4};
\addplot
[domain=0:4, samples=10, thick, cyan, name path=gomory]
	(1,x) node [pos=0.8, anchor=west, pin={75:{\color{cyan}$x_1\geq 1$}}, inner sep= 0pt] {};
\path[name path=GOM]
	(axis cs:1.25,0)--(axis cs:1.25,4);
\addplot
[domain=0:4, samples=10, thick, purple, name path=bb]
	(0,x) node [pos=0.8, anchor=east, pin={135:{\color{purple}$x_1\leq 0$}}, inner sep= 0pt] {}
	;
\path [name path=BBi]
	(axis cs:-0.25,0)--(axis cs:-0.25,4);
\addplot[thick, fill=cyan, fill opacity=0.5] fill between [of=4x3y12 and UP, soft clip={domain=0:3}];
\addplot[white] fill between [of=y3 and UP];
\addplot[white] fill between [of=gomory and AY];
\addplot[pattern=north east lines, pattern color=cyan] fill between [of=gomory and GOM];
\addplot[pattern=north east lines, pattern color=purple] fill between [of=bb and BBi];
\intne{x3}{AX}{$\alpha$}{alp};
\intsw{y3}{4x3y12}{$\beta$}{bet};
\intne{y3}{x3}{$\gamma$}{gam};
\intsw{gomory}{y3}{$\delta$}{del};
\node at (axis cs:2.5,2) {$P'$};
\node at (axis cs:-2,2) {};
\addplot[-latex, thick] coordinates
           {(0,0) (-1/3.16,3/3.16)} node [pos=.3, anchor=north, label={135:{\small $\nabla z$}}] {};
\end{axis}
\end{tikzpicture}
\caption{Rappresentazione cartesiana dei sottoproblemi di $P_0$.}
\label{fig:graph72}
\end{figure}

Inizieremo, come di consueto, ad esplorare il ramo con la disequazione di maggioranza ma si noti dal grafico come l'altro ramo porterebbe evidentemente ad un nulla di fatto in quanto l'area di regione ammissibile sarebbe vuota.
\begin{figure}[ht!]
\centering
\begin{tikzpicture}[edge from parent/.style={draw=black,-latex},
					level/.style={sibling distance=50mm, level distance=20mm},
					cross/.style={path picture={ 
 						\draw[black]
						(path picture bounding box.south east) -- (path picture bounding box.north west) (path picture bounding box.south west) -- (path picture bounding box.north east);}}
					]
\node [circle, draw, label=east:{$L_B=-8$}] (z) {$P_0$}
	child {node [circle, draw] (a) {$P_1$}
			edge from parent node[left] {$x_1\geq 1$}
		   }
	;
\end{tikzpicture}
\end{figure}

\textbf{N.B.}: il taglio che applicheremo sar� lo stesso che abbiamo applicato in precedenza con i tagli di Gomory.

Il taglio applicato nel problema $P_1$ pu� essere cos� espresso in forma standard:
$$
-x_1+s=-1
$$
Lo introduciamo nel tableau in tabella \vref{tab:tab711} ottenendo quello in tabella \vref{tab:tab714}.
\begin{table}[htbp]
\centering
{
	\newcommand{\uq}{$\frac{1}{4}$}
	\newcommand{\tq}{$\frac{3}{4}$}
	\newcommand{\qq}{$\frac{15}{4}$}
	\newcommand{\nq}{$\frac{9}{4}$}
	\newcommand{\muq}{$-\frac{1}{4}$}
	\newcommand{\mtq}{$-\frac{2}{4}$}
	\newcommand{\mqq}{$-\frac{15}{4}$}
	\newcommand{\ttq}{$\frac{33}{4}$}

\begin{tabular}{rrccccccc}
 	  & 			&$-\varphi$	& $x_1$ & $x_2$ & $x_3$ & $x_4$ & $x_5$	& $s$\\
$R_0$ & $\OL{c_j}$ 	& \Sc{\ttq}	& 0 	& 0 	& 0 	& \qq	& \Sc{\uq}	& 0 \\
\cline{3-9}
$R_1$ & $x_3$ 		& \Sc{\nq} 	& 0		& 0 	& 1 	& \tq 	& \Sc{\uq}	& 0 \\
$R_2$ & $x_2$ 		& \Sc{3} 	& 0		& 1		& 0 	& 1 	& \Sc{0}	& 0 \\
$R_3$ & $x_1$ 		& \Sc{\tq} 	& 1 	& 0 	& 0 	& \mtq 	& \Sc{\muq}	& 0 \\
\cline{3-8}
$R_4$ & $s$ 		& \Sc{-1}	& -1 	& 0 	& 0 	& 0 	& 0	& 1 \\
\end{tabular}
}
\caption{Sottoproblema $P_1$.}
\label{tab:tab714}
\end{table}
Per riottenere la sottomatrice identit� dobbiamo rendere $y_{41}=0$, a tal fine operiamo:
$$
R_4\leftarrow R_4+R_3
$$
Otterremo cos� il tableau in tabella \vref{tab:tab715}.
\begin{table}[htbp]
\centering
{
	\newcommand{\uq}{$\frac{1}{4}$}
	\newcommand{\tq}{$\frac{3}{4}$}
	\newcommand{\qq}{$\frac{15}{4}$}
	\newcommand{\nq}{$\frac{9}{4}$}
	\newcommand{\muq}{$-\frac{1}{4}$}
	\newcommand{\mtq}{$-\frac{2}{4}$}
	\newcommand{\mqq}{$-\frac{15}{4}$}
	\newcommand{\ttq}{$\frac{33}{4}$}

\begin{tabular}{rrccccccc}
 	  & 			&$-\varphi$	& $x_1$ & $x_2$ & $x_3$ & $x_4$ & $x_5$	& $s$\\
$R_0$ & $\OL{c_j}$ 	& \Sc{\ttq}	& 0 	& 0 	& 0 	& \qq	& \Sc{\uq}	& 0 \\
\cline{3-9}
$R_1$ & $x_3$ 		& \Sc{\nq} 	& 0		& 0 	& 1 	& \tq 	& \Sc{\uq}	& 0 \\
$R_2$ & $x_2$ 		& \Sc{3} 	& 0		& 1		& 0 	& 1 	& \Sc{0}	& 0 \\
$R_3$ & $x_1$ 		& \Sc{\tq} 	& 1 	& 0 	& 0 	& \mtq 	& \Sc{\muq}	& 0 \\
\cline{3-8}
$R_4$ & $s$ 		& \Sc{\muq}	& 0 	& 0 	& 0 	& \mtq 	& \C{\muq}	& 1 \\
\end{tabular}
}
\caption{Sottoproblema $P_1$, secondo tableau.}
\label{tab:tab715}
\end{table}
Non � sorprendente che questo tableau � \textbf{identico a quello in tabella \vref{tab:tab712}}. Non ripeteremo i calcoli sul tableau che sarebbero identici a quelli del caso del taglio di Gomory. Riportiamo soltanto il risultato finale:
\begin{align*}
\mathcal{B}&=\{A_3,A_2,A_1,A_5\} \\
x&=(1,3,2,0,1) \\
z(\delta)&=-\varphi=8
\end{align*}
Il nostro ramo dell'albero � giunto a una foglia.

\begin{figure}[ht!]
\centering
\begin{tikzpicture}[edge from parent/.style={draw=black,-latex},
					level/.style={sibling distance=50mm, level distance=20mm},
					cross/.style={path picture={ 
 						\draw[black]
						(path picture bounding box.south east) -- (path picture bounding box.north west) (path picture bounding box.south west) -- (path picture bounding box.north east);}}
					]
\node [circle, draw, label=east:{$L_B=-8$}] (z) {$P_0$}
	child {node [circle, draw, label=east:{$L_B=-8$}, label=south:{$-z=-8$}] (a) {$P_1$}
			edge from parent node[left] {$x_1\geq 1$}
		   }
	;
\end{tikzpicture}
\end{figure}

Siamo fortunati perch� l'attuale soluzione � uguale al lower bound del nodo \textit{root}, il che significa che non serve cercare altre soluzioni che potrebbero essere al pi� uguali a quella trovata. Ne conviene infine che il vertice $\delta(1,3)$ � una soluzione ottima del problema ILP. Il valore della soluzione �:
$$
z(\delta)=-\varphi=8
$$

\section{Esercizio 1}

Riprenderemo l'esercizio in \vref{sec:es1ILP}. Segue la parte di traccia che richiede come affrontare il problema ILP:
\begin{enumerate}
\item Risolvere con l'algoritmo branch-and-bound, scegliendo per il branching la variabile frazionaria di indice massimo ed esplorando per primo il nodo corrispondente alla condizione $x_j\leq\alpha$.
\end{enumerate}
Durante l'esercitazione in classe il tutor ha esplorato invece per prima la \textit{variabile di indice minimo}, concludendo l'esercizio in tempo record. Noi esploreremo come da traccia: l'esercizio sar� pi� lungo ma forse didatticamente pi� utile.
In ogni caso, come sempre, proveremo anche a utilizzare i tagli di Gomory.

\subsection{ILP - Tagli di Gomory}
Riportiamo in tabella \vref{tab:tab11} il tableau finale del problema LP.
\begin{table}[htbp]
\centering
{
	\newcommand{\tm}{$\frac{13}{2}$}
	\newcommand{\cm}{$\frac{5}{2}$}
	\newcommand{\um}{$\frac{1}{2}$}
	\newcommand{\mum}{$-\frac{1}{2}$}
\begin{tabular}{rrcccccc}
 & & $-\varphi$ & $x_1$ & $x_2$ & $x_3$ & $x_4$ & $x_5$ \\
$R_0$ & $\OL{c_j}$ 	& \Sc{\tm} 	& 0 	& 0 	& \um 	& 1 	& 0 \\
\cline{3-8}
%\hline
$R_1$ & $x_1$ 		& \Sc{\cm} 	& 1 	& 0 	& \um 	& -1 	& 0 \\
$R_2$ & $x_2$ 		& \Sc{2} 	& 0 	& 1 	& 0 	& 1 	& 0 \\
$R_3$ & $x_5$ 		& \Sc{\um} 	& 0 	& 0 	& \mum 	& 2 	& 1 \\
\end{tabular}
}
\caption{Tableau finale del problema di LP.}
\label{tab:tab11}
\end{table}
Applichiamo quindi il taglio di Gomory sulla riga $R_0$, cio� la prima disponibile:
$$
\sum_{j\in\{3,4\}} f_{0j}x_j\geq f_{00} \Rightarrow \frac{1}{2}x_3+0x_4\geq\frac{1}{2}
$$
Trasformiamolo in una forma algebricamente pi� consona:
$$
-\frac{1}{2}x_3+s=-\frac{1}{2}
$$
Aggiungendo questo vincolo si ottiene il tableau in tabella \vref{tab:tab12}.
\begin{table}[htbp]
\centering
{
	\newcommand{\tm}{$\frac{13}{2}$}
	\newcommand{\cm}{$\frac{5}{2}$}
	\newcommand{\um}{$\frac{1}{2}$}
	\newcommand{\mum}{$-\frac{1}{2}$}
\begin{tabular}{rrccccccc}
 	  & 			& $-\varphi$& $x_1$ & $x_2$ & $x_3$ & $x_4$ & $x_5$ & $s$ \\
$R_0$ & $\OL{c_j}$ 	& \Sc{\tm} 	& 0 	& 0 	& \um 	& 1 	& \Sc{0}& 0 \\
\cline{3-9}
$R_1$ & $x_1$ 		& \Sc{\cm} 	& 1 	& 0 	& \um 	& -1 	& \Sc{0}& 0 \\
$R_2$ & $x_2$ 		& \Sc{2} 	& 0 	& 1 	& 0 	& 1 	& \Sc{0}& 0 \\
$R_3$ & $x_5$ 		& \Sc{\um} 	& 0 	& 0 	& \mum 	& 2 	& \Sc{1}& 0 \\
\cline{3-8}
$R_4$ & $s$			& \Sc{\mum} & 0		& 0		&\C{\mum}& 0	& 0		& 1
\end{tabular}
}
\caption{Applicazione del primo taglio di Gomory.}
\label{tab:tab12}
\end{table}
L'unico elemento su cui � possibile fare pivoting � $y_{43}$ (cerchiato in tabella \vref{tab:tab12}). Procediamo con le operazioni elementari di riga (in ordine):
\begin{align*}
R_0&\leftarrow R_0 + R_4 \\
R_1&\leftarrow R_1 + R_4 \\
R_3&\leftarrow R_3 - R_4 \\
R_4&\leftarrow -2R_4
\end{align*}
Il tableau che otteniamo � quello in tabella \vref{tab:tab13}.
\begin{table}[htbp]
\centering
\begin{tabular}{rrccccccc}
 	  & 			& $-\varphi$& $x_1$ & $x_2$ & $x_3$ & $x_4$ & $x_5$ & $s$ \\
$R_0$ & $\OL{c_j}$ 	& \Sc{6} 	& 0 	& 0 	& 0 	& 1 	& \Sc{0}& 1 \\
\cline{3-9}
$R_1$ & $x_1$ 		& \Sc{2} 	& 1 	& 0 	& 0 	& -1 	& \Sc{0}& 1 \\
$R_2$ & $x_2$ 		& \Sc{2} 	& 0 	& 1 	& 0 	& 1 	& \Sc{0}& 0 \\
$R_3$ & $x_5$ 		& \Sc{1} 	& 0 	& 0 	& 0 	& 2 	& \Sc{1}& -1 \\
\cline{3-8}
$R_4$ & $x_3$		& \Sc{1} 	& 0		& 0		& 1		& 0		& 0		& -2
\end{tabular}
\caption{Simplesso duale dopo il taglio di Gomory. Soluzione ottima e intera al vertice $\zeta(2,2)$.}
\label{tab:tab13}
\end{table}
La soluzione del nuovo problema LP � ottima e intera e quindi anche la soluzione ottima del problema ILP. I nuovi valori di base, soluzione e suo valore sono:
\begin{align*}
\mathcal{B}&=\{A_1,A_2,A_5,A_4\} \\
x&=(2,2,1,0,1) \\
z(\zeta)&=-\varphi=6
\end{align*}

\subsubsection{Rappresentazione grafica}
Rappresentiamo il taglio di Gomory in funzione delle variabili $x_1$ e $x_2$, ricavandole dal vincolo del problema iniziale:
$$
2x_1+2x_2+x_3=9\Rightarrow x_3=9-2x_1-2x_2
$$
Sostituiamo nel taglio di Gomory in forma di disequazione:
$$
\frac{1}{2}x_3\geq\frac{1}{2}\Rightarrow 9-2x_1-2x_2 \geq 1 \Rightarrow x_1+x_2 \leq 4
$$
In figura \vref{fig:graph11} � rappresentato il problema dopo il taglio. In giallo � rappresentato il politopo $P'$ risultato del vecchio politopo $P$ e del taglio di Gomory applicato.
\begin{figure}[htbp]
\centering
\begin{tikzpicture}
\begin{axis}
[axis lines=middle, axis equal, enlargelimits, xlabel=$x_1$, ylabel=$x_2$,
 every axis x label/.style={
    at={(ticklabel* cs:1.01)},
    anchor=west,
 },
 every axis y label/.style={
    at={(ticklabel* cs:1.01)},
    anchor=south,
 },]
    \path[name path=AX] 
        (axis cs:\pgfkeysvalueof{/pgfplots/xmin},0)--
        (axis cs:\pgfkeysvalueof{/pgfplots/xmax},0);
    \path[name path=AY] 
        (axis cs:0,\pgfkeysvalueof{/pgfplots/ymin})--
        (axis cs:0,\pgfkeysvalueof{/pgfplots/ymax});
    \path[name path=UP]
    	(axis cs:\pgfkeysvalueof{/pgfplots/xmin},\pgfkeysvalueof{/pgfplots/ymax})--
    	(axis cs:\pgfkeysvalueof{/pgfplots/xmax},\pgfkeysvalueof{/pgfplots/ymax});

\dotgrid{0}{5}{0}{4}{black!50}    	
    	
\addplot [domain=0:4.5, samples=10, thick, blue, name path=2x2y9] {-x+(9/2)}; 
\addplot [domain=1:5, samples=10, thick, red, name path=yx-1] {x-1};
\addplot [domain=0:5, samples = 10, thick, purple, name path=y2] {2};
\addplot [domain=0:4, samples = 10, thick, green, name path=gomory] {-x+4}
	node [pos=0.2, anchor=west, pin={75:{\color{green}$x_1+x_2\leq 4$}}, inner sep= 0pt] {};
\path[name path=GOM]
	(axis cs:0,3.75)--(axis cs:3.75,0);
\addplot[thick, fill=yellow, fill opacity=0.5] fill between [of=2x2y9 and AX, soft clip={domain=0:11/4},];
\addplot[white] fill between [of=2x2y9 and y2];
\addplot[white] fill between [of=gomory and UP];
\addplot[white] fill between [of=yx-1 and AX];
\addplot[pattern=horizontal lines, pattern color=green] fill between [of=gomory and GOM];
\intse{AX}{AY}{$\alpha$}{alp};
\intse{AY}{y2}{$\beta$}{bet};
\intne{y2}{2x2y9}{$\gamma$}{gam};
\inte{2x2y9}{yx-1}{$\delta$}{del};
\intn{yx-1}{AX}{$\varepsilon$}{eps};
\intnw{gomory}{y2}{$\zeta$}{zet};
\node at (axis cs:1,1) {$P'$};
\addplot[-latex, thick] coordinates
           {(0,0) (1/2.24,2/2.24)} node [pos=1, anchor=north, label={90:{\small $\nabla z$}}] {};
\end{axis}
\end{tikzpicture}
\caption{Rappresentazione cartesiana del problema di programmazione lineare intera.}
\label{fig:graph11}
\end{figure}

\subsection{ILP - Branch and bound}
\textbf{Premessa:} i calcoli li ho fatti io autonomamente, sono molto macchinosi ma \textit{credo} di averli fatti bene. In ogni caso, data la laboriosit� di questa parte di esercizio, fate attenzione e se trovate errori, non esitate a inviarmi correzioni (o a fare una \textit{push request} su \textit{GitHub} al repository: \href{https://github.com/slaierno/OpResearch}{https://github.com/slaierno/OpResearch}).
La sezione sar� suddivisa in sottosezioni per comodit�, ognuna riguardante un sottoproblema diverso.

\subsubsection{$\pmb{P_0}$}
Il problema $P_0$ sar� quello con cui indicheremo il problema LP iniziale. La sua soluzione � anche la soluzione del rilassamento continuo del problema ILP:
\begin{align*}
x^*=\left(\frac{5}{2},2\right) \\
-z^*=\frac{13}{2}
\end{align*}
Il rilassamento continuo imposter� il nostro \textit{lower bound} di partenza:
$$
L_B = \left\lceil -z^* \right\rceil = \left\lceil \frac{13}{2} \right\rceil = -6
$$

\begin{figure}[ht!]
\centering
\begin{tikzpicture}[edge from parent/.style={draw=black,-latex},
					level/.style={sibling distance=50mm, level distance=20mm},
					cross/.style={path picture={ 
 						\draw[black]
						(path picture bounding box.south east) -- (path picture bounding box.north west) (path picture bounding box.south west) -- (path picture bounding box.north east);}}
					]
\node [circle, draw, label=east:{$L_B=-6$}] (z) {$P_0$}
	;
\end{tikzpicture}
\end{figure}

Partendo dalla variabile di indice maggiore, dal problema $P_0$ possiamo ricavare i seguenti due sottoproblemi:
\begin{align*}
P_1 &= P_0 \cap x_5\leq \left\lfloor x_5^* \right\rfloor &\Rightarrow x_5 \leq 0 \\
P_2 &= P_0 \cap x_5\geq \left\lceil x_5^* \right\rceil &\Rightarrow x_5 \geq 1
\end{align*}

I due sottoproblemi sono rappresentati in figura \vref{fig:graph12}. Per ricavare le equazioni delle rette che rappresentano il taglio sulla variabile $x_5$ si procede come al solito facendo riferimento al vincolo che introduce la variabile (esprimiamo le equazioni delle rette nella loro forma cartesiana esplicita:
$$
x_1 - x_2 + x_5 = 1 \Rightarrow x_5 = 1 - x_1 + x_2 \Rightarrow
\begin{cases}
x_5 \leq 0 \Rightarrow 1 - x_1 + x_2 \leq 0 \\
x_5 \geq 1 \Rightarrow 1 - x_1 + x_2 \geq 1
\end{cases} \Rightarrow
\begin{cases}
x_2 \leq x_1 - 1 \\
x_2 \geq x_1
\end{cases}
$$
\begin{figure}[htbp]
\centering
\begin{tikzpicture}
\begin{axis}
[axis lines=middle, axis equal, enlargelimits, xlabel=$x_1$, ylabel=$x_2$,
 every axis x label/.style={
    at={(ticklabel* cs:1.01)},
    anchor=west,
 },
 every axis y label/.style={
    at={(ticklabel* cs:1.01)},
    anchor=south,
 },]
    \path[name path=AX] 
        (axis cs:\pgfkeysvalueof{/pgfplots/xmin},0)--
        (axis cs:\pgfkeysvalueof{/pgfplots/xmax},0);
    \path[name path=AY] 
        (axis cs:0,\pgfkeysvalueof{/pgfplots/ymin})--
        (axis cs:0,\pgfkeysvalueof{/pgfplots/ymax});
    \path[name path=UP]
    	(axis cs:\pgfkeysvalueof{/pgfplots/xmin},\pgfkeysvalueof{/pgfplots/ymax})--
    	(axis cs:\pgfkeysvalueof{/pgfplots/xmax},\pgfkeysvalueof{/pgfplots/ymax});

\dotgrid{0}{5}{0}{4}{black!50}    	
    	
\addplot [domain=0:4.5, samples=10, thick, blue, name path=2x2y9] {-x+(9/2)}; 
\addplot [domain=1:5, samples=10, thick, red, name path=yx-1] {x-1};
\addplot [domain=0:5, samples = 10, thick, purple, name path=y2] {2};
\addplot [domain=1:5, samples = 10, thick, green, name path=p1] {x-1}
	node [pos=0.6, anchor=west, pin={-75:{\color{green}$x_5 \leq 0$}}, inner sep= 0pt] {};
\path[name path=p1i]
	(axis cs:1.25,0)--(axis cs:5,3.75);
\addplot [domain=0:4, samples = 10, thick, cyan, name path=p2] {x}
	node [pos=0.7, anchor=west, pin={135:{\color{cyan}$x_5 \geq 1$}}, inner sep= 0pt] {};
\path[name path=p2i]
	(axis cs:0,0.25)--(axis cs:3.75,4);
\addplot [domain=1:2.75, samples = 10, thick, brown] {x-1};
\addplot[thick, fill=cyan, fill opacity=0.3] fill between [of=y2 and p2, soft clip={domain=0:2}];
\addplot[pattern=north west lines, pattern color=green] fill between [of=p1 and p1i];
\addplot[pattern=north west lines, pattern color=cyan] fill between [of=p2 and p2i];
\intse{AX}{AY}{$\alpha$}{alp};
\intse{AY}{y2}{$\beta$}{bet};
\intne{y2}{2x2y9}{$\gamma$}{gam};
\inte{2x2y9}{yx-1}{$\delta$}{del};
\intn{yx-1}{AX}{$\varepsilon$}{eps};
\intne{p2}{y2}{$\zeta$}{zet};
\node at (axis cs:1,1.5) {$P_1$};
\node at (axis cs:2,1) {$P_2$};
\addplot[-latex, thick] coordinates
           {(0,0) (1/2.24,2/2.24)} node [pos=1, anchor=north, label={90:{\small $\nabla z$}}] {};
\end{axis}
\end{tikzpicture}
\caption{Rappresentazione cartesiana del problema di programmazione lineare intera.}
\label{fig:graph12}
\end{figure}

Partiremo dalla relazione del problema $P_1$.
\begin{figure}[ht!]
\centering
\begin{tikzpicture}[edge from parent/.style={draw=black,-latex},
					level/.style={sibling distance=50mm, level distance=20mm},
					cross/.style={path picture={ 
 						\draw[black]
						(path picture bounding box.south east) -- (path picture bounding box.north west) (path picture bounding box.south west) -- (path picture bounding box.north east);}}
					]
\node [circle, draw, label=east:{$L_B=-6$}] (z) {$P_0$}
	child {node [circle, draw] (a) {$P_1$}
			edge from parent node[left] {$x_5\leq 0$}
		   }
	;
\end{tikzpicture}
\end{figure}

\subsubsection{$\pmb{P_1}$}
Per portare all'interno del tableau di $P_0$ il vincolo aggiuntivo del problema $P_1$, trasformiamolo in una forma standard:
$$
x_5\leq 0 \Rightarrow \pmb{x_5 + s_1 = 0}
$$
Riprendiamo il tableau in tabella \vref{tab:tab11} e aggiungiamo il vincolo e la variabile surplus. Otterremo il tableau in tabella \vref{tab:tab14}.
\begin{table}[htbp]
\centering
{
	\newcommand{\tm}{$\frac{13}{2}$}
	\newcommand{\cm}{$\frac{5}{2}$}
	\newcommand{\um}{$\frac{1}{2}$}
	\newcommand{\mum}{$-\frac{1}{2}$}
\begin{tabular}{rrccccccc}
 	  & 			& $-\varphi$& $x_1$ & $x_2$ & $x_3$ & $x_4$ & $x_5$ & $s_1$ \\
$R_0$ & $\OL{c_j}$ 	& \Sc{\tm} 	& 0 	& 0 	& \um 	& 1 	& \Sc{0}& 0 \\
\cline{3-9}
$R_1$ & $x_1$ 		& \Sc{\cm} 	& 1 	& 0 	& \um 	& -1 	& \Sc{0}& 0 \\
$R_2$ & $x_2$ 		& \Sc{2} 	& 0 	& 1 	& 0 	& 1 	& \Sc{0}& 0 \\
$R_3$ & $x_5$ 		& \Sc{\um} 	& 0 	& 0 	& \mum 	& 2 	& \Sc{1}& 0 \\
\cline{3-8}
$R_4$ & $s_1$			& \Sc{0}	& 0		& 0		& 0		& 0		& 1		& 1
\end{tabular}
}
\caption{Problema $P_1$.}
\label{tab:tab14}
\end{table}
Ricaviamo nuovamente la sottomatrice identit� facendo s� che $y_{45}=0$ con l'operazione elementare di riga:
$$
R_4\leftarrow R_4 - R_3
$$
Il tableau ottenuto � quello in tabella \vref{tab:tab15}.
\begin{table}[htbp]
\centering
{
	\newcommand{\tm}{$\frac{13}{2}$}
	\newcommand{\cm}{$\frac{5}{2}$}
	\newcommand{\um}{$\frac{1}{2}$}
	\newcommand{\mum}{$-\frac{1}{2}$}
\begin{tabular}{rrccccccc}
 	  & 			& $-\varphi$& $x_1$ & $x_2$ & $x_3$ & $x_4$ & $x_5$ & $s_1$ \\
$R_0$ & $\OL{c_j}$ 	& \Sc{\tm} 	& 0 	& 0 	& \um 	& 1 	& \Sc{0}& 0 \\
\cline{3-9}
$R_1$ & $x_1$ 		& \Sc{\cm} 	& 1 	& 0 	& \um 	& -1 	& \Sc{0}& 0 \\
$R_2$ & $x_2$ 		& \Sc{2} 	& 0 	& 1 	& 0 	& 1 	& \Sc{0}& 0 \\
$R_3$ & $x_5$ 		& \Sc{\um} 	& 0 	& 0 	& \mum 	& 2 	& \Sc{1}& 0 \\
\cline{3-8}
$R_4$ & $s_1$			& \Sc{\mum} & 0		& 0		& \um	&\C{-2}	& 0		& 1
\end{tabular}
}
\caption{Problema $P_1$, secondo tableau.}
\label{tab:tab15}
\end{table}
Per applicare il simplesso duale sulla riga $R_4$ l'unico possibile elemento di pivoting � $y_{44}$ (cerchiato in tabella \vref{tab:tab15}). Le operazioni elementari di riga da effettuare sono, nell'ordine:
\begin{align*}
R_3&\rightarrow R_3 + R_4 \\
R_4&\rightarrow -\frac{1}{2}R_4 \\
R_0&\rightarrow R_0 - R_4 \\
R_1&\rightarrow R_1 + R_4 \\
R_2&\rightarrow R_2 - R_4
\end{align*}
Ne risulta il tableau in tabella \vref{tab:tab16}.
\begin{table}[htbp]
\centering
{
	\newcommand{\um}{$\frac{1}{2}$}
	\newcommand{\mum}{$-\frac{1}{2}$}
	\newcommand{\vcq}{$\frac{25}{4}$}
	\newcommand{\unq}{$\frac{11}{4}$}
	\newcommand{\sq}{$\frac{7}{4}$}
	\newcommand{\tq}{$\frac{3}{4}$}
	\newcommand{\mtq}{$-\frac{3}{4}$}
	\newcommand{\uq}{$\frac{1}{4}$}
	\newcommand{\muq}{$-\frac{1}{4}$}
\begin{tabular}{rrccccccc}
 	  & 			& $-\varphi$& $x_1$ & $x_2$ & $x_3$ & $x_4$ & $x_5$ & $s_1$ \\
$R_0$ & $\OL{c_j}$ 	& \Sc{\vcq}	& 0 	& 0 	& \tq 	& 0 	& \Sc{0}& \um \\
\cline{3-9}
$R_1$ & $x_1$ 		& \Sc{\unq}	& 1 	& 0 	& \uq 	& 0 	& \Sc{0}& \mum \\
$R_2$ & $x_2$ 		& \Sc{\sq} 	& 0 	& 1 	& \uq 	& 0 	& \Sc{0}& \um \\
$R_3$ & $x_5$ 		& \Sc{0} 	& 0 	& 0 	& 0 	& 0 	& \Sc{1}& 1 \\
\cline{3-8}
$R_4$ & $x_4$		& \Sc{\uq}	& 0		& 0		& \muq	& 1		& 0		& \mum
\end{tabular}
}
\caption{Soluzione del problema $P_1$.}
\label{tab:tab16}
\end{table}
Il problema $P_1$ � giunto ad una soluzione:
$$
x^*=\left(\frac{11}{4}, \frac{3}{4}, 0, \frac{1}{4}, 0\right)
$$
Purtroppo questa soluzione non � intera e saremo costretti successivamente a imporre nuove condizioni sulla prossima variabile frazionaria di indice massimo: $x_4$.
Nel frattempo prendiamo nota della soluzione dell'attuale rilassamento continuo e calcoliamo il nuovo \textit{lower bound}:
$$
L_B=\left\lceil -z^* \right\rceil = \left\lceil -\frac{25}{4} \right\rceil = -6
$$
La buona notizia, per ora, � che il \textit{lower bound} non � peggiorato.

\begin{figure}[ht!]
\centering
\begin{tikzpicture}[edge from parent/.style={draw=black,-latex},
					level/.style={sibling distance=50mm, level distance=20mm},
					cross/.style={path picture={ 
 						\draw[black]
						(path picture bounding box.south east) -- (path picture bounding box.north west) (path picture bounding box.south west) -- (path picture bounding box.north east);}}
					]
\node [circle, draw, label=east:{$L_B=-6$}] (z) {$P_0$}
	child {node [circle, draw, label=east:{$L_B=-6$}] (a) {$P_1$}
			edge from parent node[left] {$x_5\leq 0$}
		   }
	;
\end{tikzpicture}
\end{figure}

Le nostre possibilit� di esplorazione ora sono le seguenti:
\begin{align*}
P_3 &= P_1 \cap x_4\leq \left\lfloor x_4^* \right\rfloor &\Rightarrow x_4 \leq 0 \\
P_4 &= P_1 \cap x_4\geq \left\lceil x_4^* \right\rceil &\Rightarrow x_4 \geq 1
\end{align*}

Cerchiamo di rappresentare graficamente i nuovi vincoli. Come al solito, prendiamo il vincolo iniziale e sfruttiamolo per ricavare disequazioni utilizzanti solo le variabili $x_1$ e $x_2$:
$$
x_2 + x_4 = 2 \Rightarrow x_4 = 2 - x_2 \Rightarrow
\begin{cases}
2 - x_2 \leq 0 \\
2 - x_2 \geq 1
\end{cases}
\Rightarrow
\begin{cases}
x_2 \geq 2 \\
x_2 \leq 1
\end{cases}
$$
Osservando il grafico in figura \vref{fig:graph13} risulta ovvio che $P_3$ non porter� ad alcun risultato ma, ahinoi, la traccia richiede l'esplorazione di questo ramo, anche perch� un calcolatore non farebbe questa deduzione e si accorgerebbe solo durante l'applicazione dell'algoritmo del simplesso che il problema � impossibile. E noi faremo lo stesso.

\begin{figure}[htbp]
\centering
\begin{tikzpicture}
\begin{axis}
[axis lines=middle, axis equal, enlargelimits, xlabel=$x_1$, ylabel=$x_2$,
 every axis x label/.style={
    at={(ticklabel* cs:1.01)},
    anchor=west,
 },
 every axis y label/.style={
    at={(ticklabel* cs:1.01)},
    anchor=south,
 },]
    \path[name path=AX] 
        (axis cs:\pgfkeysvalueof{/pgfplots/xmin},0)--
        (axis cs:\pgfkeysvalueof{/pgfplots/xmax},0);
    \path[name path=AY] 
        (axis cs:0,\pgfkeysvalueof{/pgfplots/ymin})--
        (axis cs:0,\pgfkeysvalueof{/pgfplots/ymax});
    \path[name path=UP]
    	(axis cs:\pgfkeysvalueof{/pgfplots/xmin},\pgfkeysvalueof{/pgfplots/ymax})--
    	(axis cs:\pgfkeysvalueof{/pgfplots/xmax},\pgfkeysvalueof{/pgfplots/ymax});

\dotgrid{0}{5}{0}{4}{black!50}    	
    	
\addplot [domain=1:2.75, samples = 10, thick, brown, name path=p2] {x-1};
\addplot [domain=0:5, samples=10, thick, blue, name path=p3] {2} 
	node [pos=0.3, anchor=south, pin={115:{\color{blue}$x_4 \leq 0$}}, inner sep= 0pt] {};
\path[name path=p3i]
	(axis cs:0,2.25)--(axis cs:5,2.25);
\addplot[pattern=north west lines, pattern color=blue] fill between [of=p3 and p3i];
\addplot [domain=0:5, samples=10, thick, red, name path=p4] {1} 
	node [pos=0.2, anchor=west, pin={75:{\color{red}$x_4 \geq 1$}}, inner sep= 0pt] {};
\path[name path=p4i]
	(axis cs:0,.75)--(axis cs:5,.75);
\addplot[pattern=north west lines, pattern color=red] fill between [of=p4 and p4i];
\addplot [domain=1:2, samples = 10, thick, purple] {x-1};
\intnw{p2}{p4}{$\eta$}{eta};
\node at (axis cs:2,2.5) {$P_3=\emptyset$};
\node at (axis cs:2,.5) {$P_4$};
\addplot[-latex, thick] coordinates
           {(0,0) (1/2.24,2/2.24)} node [pos=1, anchor=north, label={90:{\small $\nabla z$}}] {};
\end{axis}
\end{tikzpicture}
\caption{Alternative $P_3$ e $P_4$.}
\label{fig:graph13}
\end{figure}

\subsubsection{$\pmb{P_3}$}

In forma standard, il vincolo $x_4 \leq 0$ richiede l'introduzione di un'ulteriore variabile slack:
$$
x_4+s_2=0
$$
Aggiungiamo il vincolo al tableau \vref{tab:tab16} ottenendo il tableau \vref{tab:tab17}

\begin{table}[htbp]
\centering
{
	\newcommand{\um}{$\frac{1}{2}$}
	\newcommand{\mum}{$-\frac{1}{2}$}
	\newcommand{\vcq}{$\frac{25}{4}$}
	\newcommand{\unq}{$\frac{11}{4}$}
	\newcommand{\tq}{$\frac{3}{4}$}
	\newcommand{\mtq}{$-\frac{3}{4}$}
	\newcommand{\uq}{$\frac{1}{4}$}
	\newcommand{\muq}{$-\frac{1}{4}$}
\begin{tabular}{rrcccccccc}
 	  & 			& $-\varphi$& $x_1$ & $x_2$ & $x_3$ & $x_4$ & $x_5$ & $s_1$ 	& $s_2$\\
$R_0$ & $\OL{c_j}$ 	& \Sc{\vcq}	& 0 	& 0 	& \tq 	& 0 	& 0		& \Sc{\um} 	& 0 \\
\cline{3-10}
$R_1$ & $x_1$ 		& \Sc{\unq}	& 1 	& 0 	& \uq 	& 0 	& 0		& \Sc{\mum}	& 0\\
$R_2$ & $x_2$ 		& \Sc{\tq} 	& 0 	& 1 	& \uq 	& 0 	& 0		& \Sc{\um}	& 0\\
$R_3$ & $x_5$ 		& \Sc{0} 	& 0 	& 0 	& 0 	& 0 	& 1		& \Sc{1} 	& 0\\
$R_4$ & $x_4$		& \Sc{\uq}	& 0		& 0		& \muq	& 1		& 0		& \Sc{\mum}	& 0\\
\cline{3-9}
$R_5$ & $s_2$		& \Sc{0}	& 0		& 0		& 0		& 1		& 0		& 0			& 1
\end{tabular}
}
\caption{Tableau del problema $P_3$.}
\label{tab:tab17}
\end{table}

Rendiamo nullo l'elemento $y_{54}$ per riottenere una sottomatrice identit� da utilizzare come base.
$$
R_5\leftarrow R_5 - R_4
$$
Otterremo il tableau in \vref{tab:tab18}.

\begin{table}[htbp]
\centering
{
	\newcommand{\um}{$\frac{1}{2}$}
	\newcommand{\mum}{$-\frac{1}{2}$}
	\newcommand{\vcq}{$\frac{25}{4}$}
	\newcommand{\unq}{$\frac{11}{4}$}
	\newcommand{\tq}{$\frac{3}{4}$}
	\newcommand{\mtq}{$-\frac{3}{4}$}
	\newcommand{\uq}{$\frac{1}{4}$}
	\newcommand{\muq}{$-\frac{1}{4}$}
\begin{tabular}{rrcccccccc}
 	  & 			& $-\varphi$& $x_1$ & $x_2$ & $x_3$ & $x_4$ & $x_5$ & $s_1$ 	& $s_2$\\
$R_0$ & $\OL{c_j}$ 	& \Sc{\vcq}	& 0 	& 0 	& \tq 	& 0 	& 0		& \Sc{\um} 	& 0 \\
\cline{3-10}
$R_1$ & $x_1$ 		& \Sc{\unq}	& 1 	& 0 	& \uq 	& 0 	& 0		& \Sc{\mum}	& 0 \\
$R_2$ & $x_2$ 		& \Sc{\tq} 	& 0 	& 1 	& \uq 	& 0 	& 0		& \Sc{\um}	& 0 \\
$R_3$ & $x_5$ 		& \Sc{0} 	& 0 	& 0 	& 0 	& 0 	& 1		& \Sc{1} 	& 0 \\
$R_4$ & $x_4$		& \Sc{\uq}	& 0		& 0		& \muq	& 1		& 0		& \Sc{\mum}	& 0 \\
\cline{3-9}
$R_5$ & $s_2$		& \Sc{\mtq}	& 0		& 0		& \uq	& 0		& 0		& \um		& 1
\end{tabular}
}
\caption{Problema $P_3$, secondo tableau.}
\label{tab:tab18}
\end{table}

Il nostro viaggio nel problema $P_3$ si conclude qui. $y_{i0}<0$ solo per $i=5$, ma $y_{5j}\geq j \quad\forall j$. L'algoritmo del simplesso duale ci fa sapere che in questo caso il problema primale � impossibile. Sic est.

\begin{figure}[ht!]
\centering
\begin{tikzpicture}[edge from parent/.style={draw=black,-latex},
					level/.style={sibling distance=50mm, level distance=20mm},
					cross/.style={path picture={ 
 						\draw[black]
						(path picture bounding box.south east) -- (path picture bounding box.north west) (path picture bounding box.south west) -- (path picture bounding box.north east);}}
					]
\node [circle, draw, label=east:{$L_B=-6$}] (z) {$P_0$}
	child {node [circle, draw, label=east:{$L_B=-6$}] (a) {$P_1$}
			child {node [circle, draw, cross] {$P_3$}
				edge from parent node[left] {$x_4 \leq 0$}
			}
			edge from parent node[left] {$x_5\leq 0$}
		   }
	;
\end{tikzpicture}
\end{figure}

\subsubsection{$\pmb{P_4}$}
Esploriamo il secondo figlio del nodo $P_1$ del nostro albero di \textit{branch and bound}.

\begin{figure}[ht!]
\centering
\begin{tikzpicture}[edge from parent/.style={draw=black,-latex},
					level/.style={sibling distance=40mm, level distance=20mm},
					cross/.style={path picture={ 
 						\draw[black]
						(path picture bounding box.south east) -- (path picture bounding box.north west) (path picture bounding box.south west) -- (path picture bounding box.north east);}}
					]
\node [circle, draw, label=east:{$L_B=-6$}] (z) {$P_0$}
	child {node [circle, draw, label=east:{$L_B=-6$}] (a) {$P_1$}
			child {node [circle, draw, cross] {$P_3$}
				edge from parent node[left] {$x_4 \leq 0$}
				}
			child {node [circle, draw] {$P_4$}
				edge from parent node[right] {$x_4 \geq 1$}
				}
			edge from parent node[left] {$x_5\leq 0$}
		   }
	;
\end{tikzpicture}
\end{figure}

In forma standard, il vincolo $x_4 \geq 1$ richiede l'introduzione di una variabile surplus (ma che utilizzeremo come una variabile slack). La chiameremo comunque $s_2$, senza problemi di sovrapposizione con l'omonima variabile del problema $P_3$, innanzitutto perch� sono problemi indipendenti, in secondo luogo perch� il problema $P_3$ � morto e sepolto e possiamo dimenticarcene.
$$
x_4 - s_2 = 1 \Rightarrow \pmb{-x_4 + s_2 = -1}
$$
Aggiungiamo, sempre a \vref{tab:tab16}, il nuovo vincolo ottenendo il tableau \vref{tab:tab19}.
\begin{table}[htbp]
\centering
{
	\newcommand{\um}{$\frac{1}{2}$}
	\newcommand{\mum}{$-\frac{1}{2}$}
	\newcommand{\vcq}{$\frac{25}{4}$}
	\newcommand{\unq}{$\frac{11}{4}$}
	\newcommand{\tq}{$\frac{3}{4}$}
	\newcommand{\mtq}{$-\frac{3}{4}$}
	\newcommand{\uq}{$\frac{1}{4}$}
	\newcommand{\muq}{$-\frac{1}{4}$}
\begin{tabular}{rrcccccccc}
 	  & 			& $-\varphi$& $x_1$ & $x_2$ & $x_3$ & $x_4$ & $x_5$ & $s_1$ 	& $s_2$\\
$R_0$ & $\OL{c_j}$ 	& \Sc{\vcq}	& 0 	& 0 	& \tq 	& 0 	& 0		& \Sc{\um} 	& 0 \\
\cline{3-10}
$R_1$ & $x_1$ 		& \Sc{\unq}	& 1 	& 0 	& \uq 	& 0 	& 0		& \Sc{\mum}	& 0\\
$R_2$ & $x_2$ 		& \Sc{\tq} 	& 0 	& 1 	& \uq 	& 0 	& 0		& \Sc{\um}	& 0\\
$R_3$ & $x_5$ 		& \Sc{0} 	& 0 	& 0 	& 0 	& 0 	& 1		& \Sc{1} 	& 0\\
$R_4$ & $x_4$		& \Sc{\uq}	& 0		& 0		& \muq	& 1		& 0		& \Sc{\mum}	& 0\\
\cline{3-9}
$R_5$ & $s_2$		& \Sc{-1}	& 0		& 0		& 0		& -1	& 0		& 0			& 1
\end{tabular}
}
\caption{Tableau del problema $P_4$.}
\label{tab:tab19}
\end{table}
Come di consueto, ripristiniamo la sottomatrice identit�:
$$
R_5\leftarrow R_5 + R_4
$$
Si ottiene il tableau \vref{tab:tab110}.
\begin{table}[htbp]
\centering
{
	\newcommand{\um}{$\frac{1}{2}$}
	\newcommand{\mum}{$-\frac{1}{2}$}
	\newcommand{\vcq}{$\frac{25}{4}$}
	\newcommand{\unq}{$\frac{11}{4}$}
	\newcommand{\tq}{$\frac{3}{4}$}
	\newcommand{\mtq}{$-\frac{3}{4}$}
	\newcommand{\uq}{$\frac{1}{4}$}
	\newcommand{\muq}{$-\frac{1}{4}$}
\begin{tabular}{rrcccccccc}
 	  & 			& $-\varphi$& $x_1$ & $x_2$ & $x_3$ & $x_4$ & $x_5$ & $s_1$ 	& $s_2$\\
$R_0$ & $\OL{c_j}$ 	& \Sc{\vcq}	& 0 	& 0 	& \tq 	& 0 	& 0		& \Sc{\um} 	& 0 \\
\cline{3-10}
$R_1$ & $x_1$ 		& \Sc{\unq}	& 1 	& 0 	& \uq 	& 0 	& 0		& \Sc{\mum}	& 0\\
$R_2$ & $x_2$ 		& \Sc{\tq} 	& 0 	& 1 	& \uq 	& 0 	& 0		& \Sc{\um}	& 0\\
$R_3$ & $x_5$ 		& \Sc{0} 	& 0 	& 0 	& 0 	& 0 	& 1		& \Sc{1} 	& 0\\
$R_4$ & $x_4$		& \Sc{\uq}	& 0		& 0		& \muq	& 1		& 0		& \Sc{\mum}	& 0\\
\cline{3-9}
$R_5$ & $s_2$		& \Sc{\mtq}	& 0		& 0		& \muq	& 0		& 0		& \C{\mum}	& 1
\end{tabular}
}
\caption{Tableau del problema $P_4$.}
\label{tab:tab110}
\end{table}
Determiniamo, tramite $\vartheta$, l'elemento della riga $R_5$ su cui fare pivot:
\begin{gather*}
\vartheta=\max_{j>0:y_{ij}<0}\left(\frac{y_{0j}}{y_{ij}}\right)=\frac{y_{0s}}{y_{is}} \\
\vartheta=\max\left(\frac{\frac{3}{4}}{-\frac{1}{4}},\frac{\frac{1}{2}}{-\frac{1}{2}}\right)=\max\left(-3,-1\right)=-1=\frac{y_{06}}{\pmb{y_{56}}}
\end{gather*}
L'elemento pivot � $y_{56}$, cerchiato in \vref{tab:tab110}. Le operazioni elementari di riga sono:
\begin{align*}
R_1&\leftarrow R_1 + R_5 \\
R_2&\leftarrow R_2 - R_5 \\
R_3&\leftarrow R_3 + R_5 \\
R_4&\leftarrow R_4 - R_5
R_5&\leftarrow -2R_5
\end{align*}
Si ottiene il tableau \vref{tab:tab111}.
\begin{table}[htbp]
\centering
{
	\newcommand{\unm}{$\frac{11}{2}$}
	\newcommand{\sm}{$\frac{7}{2}$}
	\newcommand{\tm}{$\frac{3}{2}$}
	\newcommand{\mtm}{$-\frac{3}{2}$}
	\newcommand{\um}{$\frac{1}{2}$}
	\newcommand{\mum}{$-\frac{1}{2}$}
\begin{tabular}{rrcccccccc}
 	  & 			& $-\varphi$& $x_1$ & $x_2$ & $x_3$ & $x_4$ & $x_5$ & $s_1$ 	& $s_2$\\
$R_0$ & $\OL{c_j}$ 	& \Sc{\unm}	& 0 	& 0 	& \um 	& 0 	& 0		& \Sc{0} 	& 1 \\
\cline{3-10}
$R_1$ & $x_1$ 		& \Sc{\sm}	& 1 	& 0 	& \um 	& 0 	& 0		& \Sc{0}	& -1\\
$R_2$ & $x_2$		& \Sc{1}	& 0		& 1		& 0		& 0		& 0		& \Sc{0}	& 1\\
$R_3$ & $x_5$ 		& \Sc{\mtm}	& 0 	& 0 	&\C{\mum}& 0 	& 1		& \Sc{0}	& 2\\
$R_4$ & $x_4$ 		& \Sc{1} 	& 0 	& 0 	& 0 	& 1 	& 0		& \Sc{0} 	& -1\\
\cline{3-9}
$R_5$ & $s_1$		& \Sc{\tm}	& 0		& 0		& \um	& 0		& 0		& 1			& -2
\end{tabular}
}
\caption{Secondo tableau del problema $P_4$.}
\label{tab:tab111}
\end{table}
Il simplesso duale non � ancora concluso perch� $y_{03}<0$. Faremo pivoting sull'unico elemento possibile: $y_{33}$. Le operazioni elementari di riga da applicare sono:
\begin{align*}
R_0&\leftarrow R_0 + R_3 \\
R_1&\leftarrow R_1 + R_3 \\
R_5&\leftarrow R_5 + R_3 \\
R_3&\leftarrow -2R_3
\end{align*}
Si ottiene il tableau in \vref{tab:tab112}
\begin{table}[htbp]
\centering
\begin{tabular}{rrcccccccc}
 	  & 			& $-\varphi$& $x_1$ & $x_2$ & $x_3$ & $x_4$ & $x_5$ & $s_1$ 	& $s_2$\\
$R_0$ & $\OL{c_j}$ 	& \Sc{4}	& 0 	& 0 	& 0 	& 0 	& 0		& \Sc{0} 	& 3 \\
\cline{3-10}
$R_1$ & $x_1$ 		& \Sc{2}	& 1 	& 0 	& 0 	& 0 	& 0		& \Sc{0}	& 1\\
$R_2$ & $x_2$		& \Sc{1}	& 0		& 1		& 0		& 0		& 0		& \Sc{0}	& 1\\
$R_3$ & $x_3$ 		& \Sc{3}	& 0 	& 0 	& 1		& 0 	& -2	& \Sc{0}	& -4\\
$R_4$ & $x_4$ 		& \Sc{1} 	& 0 	& 0 	& 0 	& 1 	& 0		& \Sc{0} 	& -1\\
\cline{3-9}
$R_5$ & $s_1$		& \Sc{0}	& 0		& 0		& 0		& 0		& 0		& 1			& 0
\end{tabular}
\caption{Terzo tableau del problema $P_4$. Soluzione intera al vertice $\eta(2,1)$.}
\label{tab:tab112}
\end{table}
La soluzione � intera, perci� il nodo in cui ci troviamo � una foglia dell'albero. Base, soluzione e suo valore valgono:
\begin{align*}
\mathcal{B}&=\{A_1,A_2,A_3,A_4,A_6\} \\
x^*&=(2,1,3,1,0) \\
-z*&=-4
\end{align*}
La soluzione attuale � $-z=-4$, ma non � necessariamente la migliore del problema \textit{root} $P_0$, il cui \textit{lower bound} � $L_B=-6$ e ha ancora un nodo figlio che pu� essere esplorato.

\begin{figure}[ht!]
\centering
\begin{tikzpicture}[edge from parent/.style={draw=black,-latex},
					level/.style={sibling distance=40mm, level distance=20mm},
					cross/.style={path picture={ 
 						\draw[black]
						(path picture bounding box.south east) -- (path picture bounding box.north west) (path picture bounding box.south west) -- (path picture bounding box.north east);}}
					]
\node [circle, draw, label=east:{$L_B=-6$}] (z) {$P_0$}
	child {node [circle, draw, label=east:{$L_B=-6$}] (a) {$P_1$}
			child {node [circle, draw, cross] {$P_3$}
				edge from parent node[left] {$x_4 \leq 0$}
				}
			child {node [circle, draw, label=south:{$-z=-4$}] {$P_4$}
				edge from parent node[right] {$x_4 \geq 1$}
				}
			edge from parent node[left] {$x_5\leq 0$}
		   }
	;
\end{tikzpicture}
\end{figure}

\subsubsection{$\pmb{P_2}$}
Ritorniamo indietro ed esploriamo il figlio di $P_0$ corrispondente al vincolo $x_5 \geq 1$.

\begin{figure}[ht!]
\centering
\begin{tikzpicture}[edge from parent/.style={draw=black,-latex},
					level/.style={sibling distance=40mm, level distance=20mm},
					cross/.style={path picture={ 
 						\draw[black]
						(path picture bounding box.south east) -- (path picture bounding box.north west) (path picture bounding box.south west) -- (path picture bounding box.north east);}}
					]
\node [circle, draw, label=east:{$L_B=-6$}] (z) {$P_0$}
	child {node [circle, draw, label=east:{$L_B=-6$}] (a) {$P_1$}
			child {node [circle, draw, cross] {$P_3$}
				edge from parent node[left] {$x_4 \leq 0$}
				}
			child {node [circle, draw, label=south:{$-z=-4$}] {$P_4$}
				edge from parent node[right] {$x_4 \geq 1$}
				}
			edge from parent node[left] {$x_5\leq 0$}
		   }
	child {node [circle, draw] {$P_2$}
		edge from parent node[right] {$x_5 \geq 1$}
		}
	;
\end{tikzpicture}
\end{figure}

In forma standard il vincolo si presenta cos�:
$$
-x_5 + s_1 = -1
$$
Come al solito, non � un problema riutilizzare $s_1$ come variabile slack, in quanto possiamo rimuovere dalla nostra memoria i tableau precedenti.
Riprendiamo il tableau \vref{tab:tab11} e aggiungendo il vincolo otteniamo il tableau \vref{tab:tab113}.
\begin{table}[htbp]
\centering
{
	\newcommand{\tm}{$\frac{13}{2}$}
	\newcommand{\cm}{$\frac{5}{2}$}
	\newcommand{\um}{$\frac{1}{2}$}
	\newcommand{\mum}{$-\frac{1}{2}$}
\begin{tabular}{rrccccccc}
 	  & 			& $-\varphi$& $x_1$ & $x_2$ & $x_3$ & $x_4$ & $x_5$ & $s_1$\\
$R_0$ & $\OL{c_j}$ 	& \Sc{\tm} 	& 0 	& 0 	& \um 	& 1 	& \Sc{0}& 0\\
\cline{3-9}
$R_1$ & $x_1$ 		& \Sc{\cm} 	& 1 	& 0 	& \um 	& -1 	& \Sc{0}& 0\\
$R_2$ & $x_2$ 		& \Sc{2} 	& 0 	& 1 	& 0 	& 1 	& \Sc{0}& 0\\
$R_3$ & $x_5$ 		& \Sc{\um} 	& 0 	& 0 	& \mum 	& 2 	& \Sc{1}& 0\\
\cline{3-8}
$R_4$ & $s_1$		& \Sc{-1}	& 0		& 0 	& 0		& 0 	& -1	& 1
\end{tabular}
}
\caption{Tableau iniziale del problema $P_2$.}
\label{tab:tab113}
\end{table}
Rendiamo $y_{45}=0$ con l'operazione elementare di riga:
$$
R_4\leftarrow R_4 + R_3
$$
Otteniamo il tableau \vref{tab:tab114}.
\begin{table}[htbp]
\centering
{
	\newcommand{\tm}{$\frac{13}{2}$}
	\newcommand{\cm}{$\frac{5}{2}$}
	\newcommand{\um}{$\frac{1}{2}$}
	\newcommand{\mum}{$-\frac{1}{2}$}
\begin{tabular}{rrccccccc}
 	  & 			& $-\varphi$& $x_1$ & $x_2$ & $x_3$ & $x_4$ & $x_5$ & $s_1$\\
$R_0$ & $\OL{c_j}$ 	& \Sc{\tm} 	& 0 	& 0 	& \um 	& 1 	& \Sc{0}& 0\\
\cline{3-9}
$R_1$ & $x_1$ 		& \Sc{\cm} 	& 1 	& 0 	& \um 	& -1 	& \Sc{0}& 0\\
$R_2$ & $x_2$ 		& \Sc{2} 	& 0 	& 1 	& 0 	& 1 	& \Sc{0}& 0\\
$R_3$ & $x_5$ 		& \Sc{\um} 	& 0 	& 0 	& \mum 	& 2 	& \Sc{1}& 0\\
\cline{3-8}
$R_4$ & $s_1$		& \Sc{\mum}	& 0		& 0 	&\C{\mum}& 2 	& 0		& 1
\end{tabular}
}
\caption{Problema $P_2$, secondo tableau.}
\label{tab:tab114}
\end{table}
Abbiamo poco da scegliere, a questo punto. L'unica riga su cui possiamo lavorare con il simplesso duale � $R_4$ e su tale riga l'unico elemento valido per il pivoting � $y_{43}$, cerchiato in \vref{tab:tab114}.
Le operazioni elementari di riga da applicare sono:
\begin{align*}
R_0&\leftarrow R_0 + R_4 \\
R_1&\leftarrow R_1 + R_4 \\
R_3&\leftarrow R_3 - R_4 \\
R_4&\leftarrow -2R_4
\end{align*}
Si ottiene il tableau in \vref{tab:tab115}.
\begin{table}[htbp]
\centering
\begin{tabular}{rrccccccc}
 	  & 			& $-\varphi$& $x_1$ & $x_2$ & $x_3$ & $x_4$ & $x_5$ & $s_1$\\
$R_0$ & $\OL{c_j}$ 	& \Sc{6} 	& 0 	& 0 	& 0 	& 3 	& \Sc{0}& 1\\
\cline{3-9}
$R_1$ & $x_1$ 		& \Sc{2} 	& 1 	& 0 	& 0 	& 1 	& \Sc{0}& 1\\
$R_2$ & $x_2$ 		& \Sc{2} 	& 0 	& 1 	& 0 	& 1 	& \Sc{0}& 0\\
$R_3$ & $x_5$ 		& \Sc{1} 	& 0 	& 0 	& 0 	& 0 	& \Sc{1}& -1\\
\cline{3-8}
$R_4$ & $x_3$		& \Sc{1}	& 0		& 0 	& 1		& -4 	& 0		& -2
\end{tabular}
\caption{Problema $P_2$, tableau finale. Vertice $\zeta(2,2)$}
\label{tab:tab115}
\end{table}
Questa volta siamo giunti ad una soluzione intera abbastanza in fretta. Base, soluzione e suo valore sono:
\begin{align*}
\mathcal{B}&=\{A_1,A_2,A_5,A_3\} \\
x^*&=(2,2,1,0,1) \\
-z^*&=-6
\end{align*}
Il nostro problema ILP � risolto e lo sarebbe anche se ci fossero altri rami da esplorare. Siamo giunti ad una soluzione intera uguale al \textit{lower bound} del problema root. Ci� significa che non solo "uccide" tutti i nodi il cui \textit{lower bound} o la cui soluzione sono inferiori a quella attuale, ma li ucciderebbe comunque tutti perch� non potremmo trovare soluzione migliore di questa.

\begin{figure}[ht!]
\centering
\begin{tikzpicture}[edge from parent/.style={draw=black,-latex},
					level/.style={sibling distance=40mm, level distance=20mm},
					cross/.style={path picture={ 
 						\draw[black]
						(path picture bounding box.south east) -- (path picture bounding box.north west) (path picture bounding box.south west) -- (path picture bounding box.north east);}}
					]
\node [circle, draw, label=east:{$L_B=-6$}] (z) {$P_0$}
	child {node [circle, draw, label=east:{$L_B=-6$}] (a) {$P_1$}
			child {node [circle, draw, cross] {$P_3$}
				edge from parent node[left] {$x_4 \leq 0$}
				}
			child {node [circle, draw, cross, label=south:{$-z=-4$}] {$P_4$}
				edge from parent node[right] {$x_4 \geq 1$}
				}
			edge from parent node[left] {$x_5\leq 0$}
		   }
	child {node [circle, draw, label=south:{$-z=-6$}] {$P_2$}
		edge from parent node[right] {$x_5 \geq 1$}
		}
	;
\end{tikzpicture}
\end{figure}

La soluzione ottima � quindi nel vertice $\zeta(2,2)$ e il suo valore � $z=6$. Esattamente come ottenuto con il metodo dei tagli di Gomory.


\chapter{06/05/2014}

Questo capitolo sar� una pacchia da scrivere: niente tabelle e niente grafici! Solo qualche albero e \textit{tantissimi} numeri. Pi� che aver paura di errori di calcolo si rischia di fare errori di battitura, perci� invito tutti a fare attenzione ai numeri che scrivo e in caso di errore comunicare tempestivamente come sempre.

\section{Esercizio 8}
Sia dato un problema KP01 con i seguenti dati:
\begin{align*}
x&=(x_1,x_2,x_3,x_4,x_5) \\
p&=(35,21,16,10,3) \\
w&=(18,10,8,9,5) \\
c&=20
\end{align*}
D'ora in poi si intendano sempre:
\begin{itemize}
\item[$x$] il vettore delle variabili, cio� gli oggetti da introdurre nello zaino. Si ricorda che $x_i \in [0,1] \forall i$; in altri termini, un oggetto pu� essere o non essere nello zaino.
\item[$p$] il vettore dei profitti di ogni oggetto.
\item[$w$] il vettore dei pesi di ogni oggetto, corrispondenti agli oggetti del vettore $p$.
\item[$c$] la capacit� dello \textit{zaino}.
\end{itemize}

Si massimizzi la funzione $z$:
$$
z=\sum_{i=1}^5 p_i x_i
$$
Nel rispetto del vincolo:
$$
c=\sum_{i=1}^5 w_i x_i \leq 20
$$

\subsection{Ordinamento per valori $\frac{p}{w}$ decrescenti}
Premettendo che qualunque metodo di risoluzione utilizziamo saremo costretti ad operare una ricerca (quasi) esaustiva, ordinare i nostri oggetti per valori decrescenti di profitto per unit� di peso (e quindi $\frac{p}{w}$) pu� essere un'utile accortezza per ridurre i tempi di ricerca.
Infatti, operando questo ordinamento, � pi� probabile che inseriremo nel nostro zaino gli oggetti con indice minore piuttosto che quelli con indice maggiore. Quindi eviteremo almeno in parte di esplorare inutilmente delle possibilit�. \textit{Questo non ci da la certezza di procedere il pi� velocemente possibile}, ma � un utile accorgimento e non fa alcun danno. Quindi tanto vale attuarlo.
Inoltre, vedremo che per il metodo branch and bound sar� indispensabile per calcolare agevolmente gli \textit{upper bound}.
Procediamo con l'elencare tutti i rapporti:
$$
\frac{p_1}{w_1}=\frac{35}{18}=1.94 \qquad 
\frac{p_2}{w_2}=\frac{21}{10}=2.1 \qquad 
\frac{p_3}{w_3}=\frac{16}{8}=2 \qquad 
\frac{p_4}{w_4}=\frac{10}{9}=1.11 \qquad 
\frac{p_5}{w_5}=\frac{3}{5}=0.6 \qquad 
$$
Riordiniamo quindi il vettore $x$:
$$
x^*=(x_2, x_3, x_1, x_4, x_5)
$$
Per comodit�, riproponiamo il problema \textit{ex novo} ma con gli elementi nell'ordine appena stabilito:
\begin{align*}
x&=(x_1,x_2,x_3,x_4,x_5)\\
p&=(21,16,35,10,3)\\
w&=(10,8,18,9,5)\\
c&=20
\end{align*}

\subsection{Metodo branch and bound}
\subsubsection{Come funziona}
Come ogni problema di carattere discreto, il metodo \textit{branch and bound} pu� essere utilizzato per ricercarne la soluzione.
In un problema KP01, ogni nodo � caratterizzato da una serie di variabili fissate e le rimanenti variabili ancora da fissare. Considerando il caso migliore delle variabili da fissare, si determina l'\textit{upper bound} $U$.
Supponendo di trovarci al \textbf{livello i-esimo} dell'albero e che sia stato cio� appena assegnato il valore della variabile $x_i$, i passi da eseguire sono i seguenti
\begin{enumerate}
\item Se la $x_i=1$, allora l'upper bound � lo stesso del nodo padre. Altrimenti, ricalcolarlo.
\item Se quella $x_i$ � l'ultima variabile, allora l'upper bound � anche la soluzione attuale $z$ e siamo in un nodo foglia. In tal caso si controlla che non ci siano nodi con upper bound o soluzione peggiore della soluzione appena trovata. In tal caso, questi nodi vengono uccisi e non saranno pi� esplorati o considerati.
\item Se ci sono nodi foglia con una soluzione migliore dell'upper bound o della soluzione corrente, questo nodo muore e si risale al nodo padre.
\item Se il caso $x_{i+1}=0$ � stato gi� esplorato, risalire al nodo padre (perch� si suppone che $x_{i+1}=1$ sia stato gi� esplorato o non fosse ammissibile).
\item Se il caso $x_{i+1}=1$ non � stato esplorato, controllare se � un caso ammissibile. In tal caso, esplorare tale nodo.
\item Esplorare il caso $x_{i+1}=0$.
\end{enumerate}
Spiegare a parole il metodo con un unico algoritmo � difficile. Si prova di seguito a dare delle indicazioni meno sequenziali ma pi� comprensibili.
\begin{itemize}
\item Quando si entra in un nodo la prima volta, si calcola l'upper bound solo se si proviene da $x_i=0$, altrimenti si \textit{eredita} l'upper bound del padre. Se siamo all'ultima variabile da assegnare, abbiamo la soluzione.
\item Se questo upper bound/soluzione � peggiore di una soluzione gi� trovata nell'albero, il nodo in cui ci si trova viene ucciso e si risale al nodo padre. Se abbiamo trovato una soluzione, tutti gli altri nodi con upper bound o soluzione inferiore a questa vengono uccisi.
\item Si esplora prima $x_{i+1}=0$, se possibile. Se non � possibile si esplora $x_{i+1}=0$.
\item Se si risale da un ramo $x_{i+1}=0$ si pu� continuare a risalire in quanto abbiamo gi� appurato che se possibile si esplora prima il caso $x_{i+1}=1$. Quindi, questo caso � stato gi� esplorato o non era esplorabile.
\item Se si risale da un ramo $x_{i+1}=1$ dobbiamo esplorare per forza il ramo $x_{i+1}=0$.
\item A fine esplorazione, saremo rimasti solo con il nodo con la soluzione ottima.
\end{itemize}
\textbf{NdA}: Procedendo con l'esercizio vi prometto che sar� tutto pi� chiaro. Non ci ho capito nulla nemmeno io quando ho iniziato a scrivere le istruzioni.

\subsubsection{Come calcolare l'upper bound}
Il calcolo dell'upper bound consiste nello scoprire qual � il massimo profitto ottenibile con il peso a disposizione a partire dalle variabili gi� assegnate.

Se � appena stata assegnata la variabile $x_{k-1}$ (supponendo per convenzione che $x_0$ assegnato indichi che non � stata assegnata alcuna variabile), si sommano i profitti di tutte le variabile successive che entrano nello zaino senza eccedere la capacit�. Infine, assegnato l'ultimo elemento possibile, il successivo viene detto \textit{elemento critico} e se ne prende una frazione tale da riempire lo zaino col maggior profitto possibile.
Questo metodo funziona solo se, ovviamente, gli oggetti sono in ordine crescente di profitto per unit� di peso. Di fatto, l'\textit{upper bound} � il \textbf{rilassamento continuo} del problema.

In termini matematici, sia $c^*$ la \textbf{capacit� attualmente occupata} dagli elementi assegnati:
$$
c^* = \sum_{i=1}^{k-1} c_i x_i
$$
Sia $s$ l'\textbf{elemento critico}, ovvero il primo elemento che \textit{non entra}. Su un insieme di $n$ oggetti, $s$ � tale che:
$$
s\triangleq\left\lbrace i : c^*+\sum_{j=k}^i w_j > c\right\rbrace
$$
Sia $\OL{c}$ la \textbf{capacit� residua}:
$$
\OL{c}\triangleq c-c^*-\sum_{j=k}^{s-1} w_j
$$
Sia $p^*$ il profitto attuale generato dagli elementi assegnati:
$$
p^* = \sum_{i=1}^{k-1} p_i x_i
$$
L'upper bound $U$ � quindi cos� definito:
$$
U\triangleq\left\lfloor p^* + \sum_{j=1}^{s-1} p_j + \OL{c}\frac{p_s}{w_s}\right\rfloor
$$

\subsubsection{L'esercizio}

Per brevit� e per evitare di utilizzare miriadi di pagine solo per disegnare decine di alberi, introdurremo ora direttamente l'albero finale. Baster� seguire, durante l'esercizio, i nodi man mano che li si percorrono.
Si ricorda che, per convenzione, quando un upper bound $\UL{U}$ � sottolineato nell'albero, significa che � stato necessario calcolarlo. Altrimenti, � stato semplicemente ereditato dal padre.

\begin{figure}[ht!]
\centering
\begin{tikzpicture}[edge from parent/.style={draw=black,-latex},
					level/.style={sibling distance=25mm, level distance=20mm},
					level 1/.style={sibling distance=50mm},
					cross/.style={path picture={ 
 						\draw[black]
						(path picture bounding box.south east) -- (path picture bounding box.north west) (path picture bounding box.south west) -- (path picture bounding box.north east);}}
					]
\node [circle, draw, label=east:{$\UL{U}=40$}] {$0$}
	child {node [circle, draw, label=east:{$U=40$}] {$1$}
			child {node [circle, draw, label=east:{$U=40$}] {$2$}
				child {node [circle, draw, label=east:{$\UL{U}=39$}] {$3$}
					child {node [circle, draw, label=east:{$\UL{U}=38$}] {$4$}
						child {node [circle, draw, label=east:{$z=37$}] {$5$}
							edge from parent node[right] {$x_5=0$}							
							}
						edge from parent node[right] {$x_4 = 0$}
						}
					edge from parent node [right] {$x_3 = 0$}
					}
				edge from parent node[left] {$x_2 = 1$}
				}
			child {node [circle, draw, label=east:{$\UL{U}=40$}] {$6$}
				child {node [circle, draw, cross, label=east:{$\UL{U}=31$}] {$7$}
					edge from parent node[right] {$x_3=0$}
					}
				edge from parent node[right] {$x_2 = 0$}
				}
		edge from parent node[left] {$x_1 = 1$}
		}
	child {node [circle, draw, label=east:{$\UL{U}=39$}] {$8$}
		child {node [circle, draw, label=east:{$U=39$}] {$9$}
			child {node [circle, draw, cross, label=east:{$\UL{U}=27$}] {$10$}
				edge from parent node[right] {$x_3 = 0$}
				}
			edge from parent node[left] {$x_2 = 1$}
			}
		child {node [circle, draw, cross, label=east:{$\UL{U}=37$}] {$11$}
			edge from parent node[right] {$x_2 = 0$}
			}
		edge from parent node[right] {$x_1 = 0$}
		}
	;
\end{tikzpicture}
\end{figure}

\paragraph{Nodo 0}
Calcoliamo l'upper bound del nodo \textit{root}.
Gli oggetti $x_1$ e $x_2$ entrano completamente nello zaino, lasciando una capacit� residua $\OL{c}=2$. Perci� l'elemento critico � $x_3$ e l'upper bound �:
$$
U_0=p_1+p_2+\left\lfloor\OL{c}\frac{p_3}{w_3}\right\rfloor=21+16+\left\lfloor 2\frac{35}{18} \right\rfloor=40
$$
L'oggetto $x_1$ entra nello zaino, procediamo con l'esplorare il nodo figlio corrispondente.

\paragraph{Nodo 1}
L'upper bound � ereditato dal padre, quindi nessun calcolo da fare. Anche la variabile $x_2$ entra nello zaino ed esploreremo il figlio corrispondente.

\paragraph{Nodo 2}
L'upper bound � ereditato dal padre, quindi nessun calcolo da fare. La variabile $x_3$ non entra nello zaino ed esploreremo direttamente il figlio corrispondente al ramo $x_3=0$.

\paragraph{Nodo 3}
Non prendendo in considerazione l'elemento $x_3$, dovremmo provare a inserire l'elemento $x_4$. Ma non entra, quindi diventa il nuovo elemento critico.
$$
U_3=p_1+p_2+\left\lfloor\OL{c}\frac{p_4}{w_4}\right\rfloor=21+16+\left\lfloor 2\frac{10}{9} \right\rfloor=39
$$
Il nostro upper bound � leggermente calato. In ogni caso, l'elemento $x_4$ non entra nel nostro zaino e l'unica prossima opzione � esplorare il ramo $x_4=0$.

\paragraph{Nodo 4}
Anche questa volta dobbiamo calcolare il nuovo upper bound. Nemmeno l'elemento $x_5$ entra nello zaino e diventer� il nuovo elemento critico.\footnote{Che poi, per capirci, a che serve calcolare un altro upper bound se sappiamo gi� che $x_5$ non entrer�? Tanto vale andare al prossimo nodo e calcolare la soluzione direttamente. Ma va be', calcoliamo lo stesso tutto l'ambaradan.}
$$
U_4=p_1+p_2+\left\lfloor\OL{c}\frac{p_5}{w_5}\right\rfloor=21+16+\left\lfloor 2\frac{3}{5} \right\rfloor=38
$$
Abbiamo appena detto che $x_5$ non entra nello zaino, perci� esploreremo il ramo $x_5 = 0$

\paragraph{Nodo 5}
Ci siamo. Siamo arrivati all'ultima variabile, non serve calcolare l'upper bound nuovamente, ma possiamo ottenere una soluzione.
$$
z^*=\sum_{i=1}^5 p_i x_i = 21+16=37
$$
Fatto ci�, risaliamo fino a tornare al nodo 1. Ricordiamo che se si sale da un ramo $x_i=0$, significa che il rispettivo ramo $x_i=1$ � stato gi� esplorato o era non ammissibile, quindi si risale finch� non si giunge ad un nodo tramite un ramo $x_i=1$.
Giunti al nodo 1, esploriamo il figlio corrispondente al ramo $x_2=0$.

\paragraph{Nodo 6}
Siamo scesi da un ramo $x_i=0$, quindi ricalcoliamo l'upper bound. Questa volta non prenderemo pi� nello zaino l'elemento $x_2$ e guarderemo al successivo. Purtroppo, $x_3$ non entra nello zaino e diventa il nostro nuovo elemento critico.
$$
U_6=p_1+\left\lfloor\OL{c}\frac{p_3}{w_3}\right\rfloor=21+\left\lfloor 10\frac{35}{18} \right\rfloor=40
$$
L'upper bound appena calcolato ci fa ben sperare, essendo superiore a $z^*$. Non entrando $x_3$ nello zaino, esploriamo il ramo $x_3=0$.

\paragraph{Nodo 7}
L'upper bound deve essere ricalcolato. Siamo fortunati che almeno l'elemento $x_4$ entra nello zaino. L'elemento critico diventa $x_5$.
$$
U_7=p_1+p_4+\left\lfloor\OL{c}\frac{p_5}{w_5}\right\rfloor=21+10+\left\lfloor 1\frac{3}{5} \right\rfloor=31
$$
Possiamo notare che $U_7\leq z^*$, il che significa che anche andando avanti nell'esplorazione potremmo trovare solo soluzioni peggiori di $z^*$. Queste non ci interessano, e il nodo 5 \textit{uccide} il \textit{nodo 7}.
Risalendo, giungiamo di nuovo al nodo \textit{root}. Esploriamo l'altro ramo del nodo 0, quello corrispondente a $x_1=0$.

\paragraph{Nodo 8}
Ricalcoliamo l'upper bound. $x_2$ entra nello zaino ma $x_3$ no, diventando elemento critico.
$$
U_8=p_2+\left\lfloor\OL{c}\frac{p_3}{w_3}\right\rfloor=16+\left\lfloor 12\frac{35}{18} \right\rfloor=39
$$
Vale ancora la pena continuare l'esplorazione, poich� stando all'upper bound attuale potremmo trovare soluzioni migliori di $z^*$. $x_2$ entra nello zaino, quindi esploreremo per primo il ramo $x_2=1$.

\paragraph{Nodo 9}
L'upper bound viene ereditato dal nodo padre, qunidi non cambia. Sappiamo gi� che $x_3$ non entra nello zaino, quindi esploriamo direttamente il ramo $x_3=0$.

\paragraph{Nodo 10}
L'upper bound deve essere ricalcolato. Stando alle variabili fissate finora, il prossima elemento da analizzare � $x_4$. Questo entra tranquillamente nello zaino insieme ad $x_2$, ma lo stesso non si pu� dire di $x_5$ che diventa elemento critico.
$$
U_10=p_2+p_4+\left\lfloor\OL{c}\frac{p_5}{w_5}\right\rfloor=16+10+\left\lfloor 3\frac{3}{5}\right\rfloor=27
$$
Fine del viaggio per questo ramo dell'albero. $U_9\leq z^*$ e il nodo 5 uccide anche il nodo 10.
Risaliamo l'albero fino ad arrivare al nodo 8, che aveva ancora un figlio da esplorare: quello corrispondente al ramo $x_2=0$.

\paragraph{Nodo 11}
Upper bound da ricalcolare anche questa volta. $x_1$ e $x_2$ abbiamo gi� stabilito che non saranno introdotti nello zaino. Perci� introdurremo $x_3$, che ha peso $p_3=18$ e non lascer� spazio per nessun altro oggetto. L'elemento critico � di conseguenza $x_4$.
$$
U_11=p_3+\left\lfloor\OL{c}\frac{p_4}{w_4}\right\rfloor=35+\left\lfloor 2\frac{10}{9} \right\rfloor=37
$$
Non � un cattivo nodo e tecnicamente potrebbe portarci ad una soluzione uguale a $37$.\footnote{In realt� no, ma noi facciamo finta di essere stupidi calcolatori e di non intuire il risultato.} A noi comunque non importa se troviamo una soluzione peggiore o \textit{al pi� uguale} a quella che abbiamo gi� ottenuto in precedenza; ce ne basta una. Quindi, da bravo serial killer, il nodo 5 uccider� anche il nodo 11. E a questo punto tutti i rami sono stati esplorati o \textit{potati}.

\paragraph{Conclusione}
Il vettore della soluzione e il suo valore sono:
\begin{align*}
x^*&=(1,1,0,0,0) \\
z^*&=37
\end{align*}
C'era la probabilit� che ci fossero delle altre soluzioni con lo stesso valore, ma a noi non importa. Trovata una, ci teniamo quella.

\subsection{Metodo della programmazione dinamica}

\subsubsection{Come funziona}
Stabiliamo prima di tutto un concetto base.

L'insieme $M_i$ � l'insieme delle coppie (o triple, vedremo poi perch�) contenenti tutte le possibili combinazioni ottenibili della coppia $(\textit{peso},\textit{profitto})$ con i primi $i$ elementi a disposizione. Nel caso si parli di triple, e non di coppie, alla coppia $(\textit{peso},\textit{profitto})$ viene aggiunto l'insieme di elementi utilizzati per ottenere quella coppia, diventando la tripla $(\textit{peso},\textit{profitto}, \{\textit{elementi})$. Questo � utile se vogliamo risalire velocemente a quali elementi sono stati utilizzati per quella soluzione. In ogni caso, sia che si parli di coppie che di triple, esse assumono il nome di \textbf{stati}.

Per ottenere un insieme $M_i$ si parte dall'insieme $M_{i-1}$ e si aggiungono, oltre a tutti gli stati gi� presenti, quelli ottenibili combinando quelli gi� presenti insieme all'elemento $i$, ove si rispetti il vincolo di capacit�.

In aggiunta a tutto ci�, � utile eliminare gli stati inutili o ridondanti. In altri termini, se ci sono due stati in cui uno dei due ottiene un profitto minore con un peso maggiore si pu� eliminare, in quanto non porter� di certo a stati migliori dell'altro.
Formalmente, dati due stati:
\begin{align*}
S'&=(P',W') \\
S''&=(P'',W'')
\end{align*}
Si dice che $S'$ \textbf{domina} $S''$ se:
$$
P'\geq P'' \wedge W' \leq W''
$$
Non ha alcuna importanza se si utilizzano pi� o meno elementi per raggiungere il profitto. Questo metodo contribuir� ad alleggerire la dimensione dell'insieme $M_i$.

Infine, lo stato con il profitto maggiore sar� ovviamente la nostra soluzione. Se abbiamo tenuto traccia delle variabili utilizzate, queste ultime saranno le variabili da porre a $1$ nella soluzione e le altre saranno poste a $0$. Altrimenti, bisogner� risalire dagli stati al momento in cui � stato aggiunto un determinato stato. Questo metodo per� risulta pi� scomodo del primo.

Durante la spiegazione dell'esercizio il procedimento risulter� chiaro in quanto � estremamente meccanico da applicare in pratica.

\subsubsection{L'esercizio}
\paragraph{$\pmb{M_0}$}
Si parte dall'insieme $M_0$ che conterr� lo stato in cui non prendiamo alcun elemento.
$$
M_0=\{(\emptyset,0,0)\}
$$
Nient'altro da fare.

\paragraph{$\pmb{M_1}$}
Si aggiunge ad ogni stato di $M_0$ l'elemento $x_1$, formando l'insieme $M_1$.
$$
M_1=M_0\cup (M_0\uplus x_1) = \{(\emptyset,0,0),\;(\{x_1\},21,10)\}
$$
Anche qui, nient'altro da fare. Il secondo stato � formato dallo stato di $M_0$ con in pi� l'elemento $x_1$:
$$
S_2=S_1+(\{x_1\},21,10\}=(\emptyset\cup\{x_1\},0+21,0+10)=(\{x_1\},21,10)\}
$$

\paragraph{$\pmb{M_2}$}
Si aggiunge ad ogni stato di $M_1$ l'elemento $x_2$, formando l'insieme $M_2$.
$$
M_2=M_1\cup (M_1\uplus x_2) = \{(\emptyset,0,0),\;(\{x_1\},21,10),\;(\{x_2\},16,8),\;(\{x_1,x_2\},37,18\}
$$
Soffermiamoci un attimo, solo questa volta, su come abbiamo formato $M_2$.
\begin{align*}
M_1\uplus x_2=&\{(\emptyset,0,0),\;(\{x_1\},21,10)\}+(\{x_2\},16,8) \\
			 =&\{(\emptyset\cup\{x_2\},0+16,0+8),\;(\{x_1\}\cup\{x_2\},21+16,10+8)\} \\
			 =&\{(\{x_2\},16,8),\;(\{x_1,x_2\},37,18\} \\
M_2=M_1\cup (M_1\uplus x_2) =& \{(\emptyset,0,0),\;(\{x_1\},21,10)\}\cup\{(\{x_2\},16,8),\;(\{x_1,x_2\},37,18)\}\\
			 =&\{(\emptyset,0,0),\;(\{x_1\},21,10),\;(\{x_2\},16,8),\;(\{x_1,x_2\},37,18)\}
\end{align*}
Sperando che tutto sia chiaro, non ci soffermeremo pi� su come si formano i vari gruppi $M_i$.

\paragraph{$\pmb{M_3}$}
Si noti che l'elemento $x_3$ non pu� coesistere con gli elementi $x_1$ e $x_2$ poich� sommando il suo peso $18$ con qualsiasi altro peso si eccederebbe la capacit� $c=20$, quindi sar� presente solo da solo.
\begin{align*}
M_3=M_2\cup (M_2\uplus x_3) = \{(\emptyset,0,0),\;
								(\{x_1\},21,10),\;
								(\{x_2\},16,8),\;
								\DA{(\{x_1,x_2\},37,18)},\;
								\SA{(\{x_3\},35,18)}\}
\end{align*}
L'ultimo stato � dominato dal precedente poich� a parit� di peso offriva un profitto maggiore.
\footnote{
Per indicare uno stato che ne domina un altro si useranno colori diverse per indicare dominazioni diverse. Inoltre, lo stato dominante sar� in grassetto e lo stato dominato sar� sbarrato. Si prenda come esempio il seguente insieme:
$$
M_n=\{\DA{S_1},\SB{S_2},\SA{S_3},\DC{S_4},\SA{S_5},\DB{S_6},\SC{S_7},S_8\}
$$
Cos� si indica che $S_1$ domina $S_3$ e $S_5$, $S_4$ domina $S_7$ e infine $S_6$ domina $S_2$. $S_8$, invece, non domina n� � dominato da nessuno stato.
}
Lo cancelliamo, quindi, e non comparir� pi� nei nostri insiemi.

\paragraph{$\pmb{M_4}$}
\begin{align*}
M_4=&M_3\cup(M_3\uplus x_4) = \\
=&\{
	(\emptyset,0,0)		,\;
	(\{x_1\},21,10)		,\;
\DA{(\{x_2\},16,8)}		,\;
\DB{(\{x_1,x_2\},37,18)},\; \\&
\SA{(\{x_4\},10,9)}		,\;
\SB{(\{x_1,x_4\},31,19)},\;
	(\{x_2,x_4\},26,17)
	\}
\end{align*}

\paragraph{$\pmb{M_5}$}
\begin{align*}
M_5=&M_4\cup(M_4\uplus x_5) = \\
=&\{
	(\emptyset,0,0)		,\;
\DA{(\{x_1\},21,10)}	,\;
	(\{x_2\},16,8)		,\;
	\opt{(\{x_1,x_2\},37,18)}	,\; \\&
	(\{x_2,x_4\},26,17)	,\;
	(\{x_5\},3,5)		,\;
	(\{x_1,x_5\},24,15)	,\;
\SA{(\{x_2,x_5\},19,13)}
	\}
\end{align*}
Lo stato ottimo � stato evidenziato. Se ne deduce quindi che la soluzione ottima e il suo valore sono:
\begin{align*}
x^*=& (1,1,0,0,0) \\
z^*=& 37
\end{align*}

\section{Esercizio 9}
Sia dato un problema KP01 con i seguenti dati:
\begin{align*}
x&=(x_1,x_2,x_3,x_4,x_5) \\
p&=(49,100,36,15,18) \\
w&=(12,15,9,4,5) \\
c&=30
\end{align*}
Si massimizzi la funzione $z$:
$$
z=\sum_{i=1}^5 p_i x_i
$$
Nel rispetto del vincolo:
$$
c=\sum_{i=1}^5 w_i x_i \leq 20
$$

\subsection{Ordinamento per valori $\frac{p}{w}$ decrescenti}
Procediamo con l'elencare tutti i rapporti:
$$
\frac{p_1}{w_1}=\frac{49}{12}=4.083 \qquad 
\frac{p_2}{w_2}=\frac{100}{15}=6.67 \qquad 
\frac{p_3}{w_3}=\frac{36}{9}=4 \qquad 
\frac{p_4}{w_4}=\frac{15}{4}=3.75 \qquad 
\frac{p_5}{w_5}=\frac{18}{5}=3.6 \qquad 
$$
Riordiniamo quindi il vettore $x$:
$$
x^*=(x_2, x_1, x_3, x_5, x_4)
$$
Per comodit�, riproponiamo il problema \textit{ex novo} ma con gli elementi nell'ordine appena stabilito:
\begin{align*}
x&=(x_1,x_2,x_3,x_4,x_5)\\
p&=(100,49,36,15,18)\\
w&=(15,12,9,4,5)\\
c&=20
\end{align*}

\subsection{Metodo branch and bound}
Oramai siamo diventati \textit{smart} e non � pi� necessario stare a spiegare tutto il procedimento. Inoltre saranno snellite le formule per gli upper bound che saranno solo numeriche.

Segue l'albero di branching.

\begin{figure}[ht!]
\centering
\begin{tikzpicture}[edge from parent/.style={draw=black,-latex},
					level/.style={sibling distance=25mm, level distance=20mm},
					level 1/.style={sibling distance=50mm},
					level 2/.style={sibling distance=50mm},
					cross/.style={path picture={ 
 						\draw[black]
						(path picture bounding box.south east) -- (path picture bounding box.north west) (path picture bounding box.south west) -- (path picture bounding box.north east);}}
					]
\node [circle, draw, label=east:{$\UL{U}=161$}] {$0$}
	child {node [circle, draw, label=east:{$U=161$}] {$1$}
		child {node [circle, draw, label=east:{$U=161$}] {$2$}
			child {node [circle, draw, label=east:{$\UL{U}=160$}] {$3$}
				child {node [circle, draw, label=east:{$\UL{U}=159$}] {$4$}
					child {node [circle, draw, cross, label=south:{$z=149$}] {$5$}
						edge from parent node[right] {$x_5=0$}}
					edge from parent node[right] {$x_4=0$}}
				edge from parent node[right] {$x_3=0$}}
			edge from parent node[left] {$x_2=1$}}
		child {node [circle, draw, label=east:{$\UL{U}=158$}] {$6$}
			child {node [circle, draw, label=east:{$U=158$}] {$7$}
				child {node [circle, draw, label=east:{$U=158$}] {$8$}
					child {node [circle, draw, cross, label=south:{$z=151$}] {$9$}
						edge from parent node[right] {$x_5=0$}}
					edge from parent node[left] {$x_4=1$}}
				child {node [circle, draw, label=east:{$\UL{U}=154$}] {$10$}
					child {node [circle, draw, label=south:{$z=151$}] {$11$}
						edge from parent node[left] {$x_5=1$}}
					edge from parent node[right] {$x_4=0$}}
				edge from parent node[left] {$x_3=1$}}
			child {node [circle, draw, cross, label=east:{$\UL{U}=133$}] {$12$}
				edge from parent node[right] {$x_3=0$}}
			edge from parent node[right] {$x_2=0$}}
		edge from parent node[left] {$x_1=1$}}
	child {node [circle, draw, cross, label=east:{$\UL{U}=118$}] {$13$}
		edge from parent node[right] {$x_1=0$}}
	;
\end{tikzpicture}
\end{figure}

\paragraph{Nodo 0}
Calcoliamo l'upper bound del nodo \textit{root}.
Gli oggetti $x_1$ e $x_2$ entrano completamente nello zaino, lasciando una capacit� residua $\OL{c}=3$. Perci� l'elemento critico � $x_3$ e l'upper bound �:
$$
U_0=100+49+\rest{3}{36}{9}=161
$$
L'oggetto $x_1$ entra nello zaino, procediamo con l'esplorare il nodo figlio corrispondente.

\paragraph{Nodo 1}
L'upper bound � ereditato dal padre, quindi nessun calcolo da fare. Anche la variabile $x_2$ entra nello zaino ed esploreremo il figlio corrispondente.

\paragraph{Nodo 2}
L'upper bound � ereditato dal padre, quindi nessun calcolo da fare. La variabile $x_3$ non entra nello zaino ed esploreremo direttamente il figlio corrispondente al ramo $x_3=0$.

\paragraph{Nodo 3}
Non prendendo in considerazione l'elemento $x_3$, dovremmo provare a inserire l'elemento $x_4$. Ma non entra, quindi diventa il nuovo elemento critico.
$$
U_3=100+49+\rest{3}{15}{4}=160
$$
Il nostro upper bound � leggermente calato. In ogni caso, l'elemento $x_4$ non entra nel nostro zaino e l'unica prossima opzione � esplorare il ramo $x_4=0$.

\paragraph{Nodo 4}
Anche questa volta dobbiamo calcolare il nuovo upper bound. Nemmeno l'elemento $x_5$ entra nello zaino e diventer� il nuovo elemento critico.
$$
U_4=100+49+\rest{3}{18}{5}=159
$$
Abbiamo appena detto che $x_5$ non entra nello zaino, perci� esploreremo il ramo $x_5 = 0$

\paragraph{Nodo 5}
Siamo arrivati all'ultima variabile, non serve calcolare l'upper bound nuovamente, ma possiamo ottenere una soluzione.
$$
z^*=\sum_{i=1}^5 p_i x_i = 100+49=149
$$
Fatto ci�, risaliamo fino a tornare al nodo 1. Giunti ad esso, esploriamo il figlio corrispondente al ramo $x_2=0$.

\paragraph{Nodo 6}
Siamo scesi da un ramo $x_i=0$, quindi ricalcoliamo l'upper bound. Questa volta non prenderemo pi� nello zaino l'elemento $x_2$ e guarderemo al successivo. $x_3$ e $x_4$ entrano nello zaino e $x_5$ diventa il nostro nuovo elemento critico.
$$
U_6=100+36+15+\rest{2}{18}{5}
$$
L'upper bound appena calcolato ci fa ben sperare, essendo superiore a $z^*$. Esploriamo il ramo $x_3=1$.

\paragraph{Nodo 7}
L'upper bound � ereditato dal padre, quindi nessun calcolo da fare. Anche la variabile $x_2$ entra nello zaino ed esploreremo il figlio corrispondente.

\paragraph{Nodo 8}
L'upper bound � ereditato dal padre, quindi nessun calcolo da fare. Abbiamo gi� stabilito prima che la variabile $x_5$ non entra nello zaino, perci� esploreremo direttamente il ramo $x_5=0$.

\paragraph{Nodo 9}
Siamo arrivati all'ultima variabile, non serve calcolare l'upper bound nuovamente, ma possiamo ottenere una soluzione.
$$
z^*=\sum_{i=1}^5 p_i x_i = 100+36+15 = 151
$$
Questa soluzione � migliore di quella ottenuta al nodo 5, che perci� verr� ucciso dal nodo 9.
Fatto ci�, risaliamo fino a tornare al nodo 7. Giunti ad esso, esploriamo il figlio corrispondente al ramo $x_4=0$.

\paragraph{Nodo 10}
L'upper bound deve essere ricalcolato. Stando alle variabili fissate finora, il prossima elemento da analizzare � $x_5$. Questo entra tranquillamente nello zaino.
$$
U_10=100+36+18=154
$$
Esploriamo il ramo $x_5=1$.

\paragraph{Nodo 11}
Dato che siamo scesi dal ramo $x_5=1$ e che questa � l'ultima variabile, l'upper bound del padre � la soluzione del figlio (provare per credere).
$$
z^*=U_10=154
$$
La nostra soluzione � migliorata ulteriormente e il nodo 9 viene ucciso dall'attuale nodo 11.

Risaliamo al nodo 10. In teoria avremmo ancora da esplorare un figlio ma dato che l'upper bound del nodo 10 � uguale alla soluzione che abbiamo attualmente tra le mani, � inutile esplorare ulteriori figli. Ovviamente, se prendendo $x_5$ abbiamo avuto una soluzione, non prendendolo non potremo fare altro che peggiorarla. Ancora una volta, i calcolatori elettronici sono stupidi e non faranno questa intuizione.

Risaliamo quindi fino al nodo 6. Da qui esploriamo il ramo $x_3=0$.

\paragraph{Nodo 12}
L'upper bound deve essere ricalcolato. Stando alle variabili fissate finora, il prossima elemento da analizzare � $x_4$. Questo entra tranquillamente nello zaino e lo stesso si pu� dire per $x_5$.
$$
U_12=100+15+18=133
$$
L'upper bound � minore della soluzione migliore trovata finora, quindi il nodo 11 uccide il nodo 12.

Risaliamo fino al nodo 0 ed esploriamo il suo figlio corrispondente al ramo $x_1=0$

\paragraph{Nodo 13}
Ricalcoliamo l'upper bound. Non prenderemo in considerazione l'elemento $x_1$. Senza di questo, tutti gli altri elementi entrano nello zaino.
$$
U_13=49+36+15+18=118
$$
Entrando tutti, questa � anche una soluzione del problema. Ma non ce ne importa granch� perch� � un upper bound nettamente inferiore alla soluzione trovata. Il nodo 11 uccider� anche il nodo 13.

Tutti i rami sono stati esplorati o potati e siamo giunti ad una conclusione.

\paragraph{Conclusione}
Il vettore della soluzione e il suo valore sono:
\begin{align*}
x^*&=(1,0,1,0,1) \\
z^*&=154
\end{align*}

\section{Metodo della programmazione dinamica}

\paragraph{$\pmb{M_0}$}
Si parte dall'insieme $M_0$ che conterr� lo stato in cui non prendiamo alcun elemento.
$$
M_0=\{(\emptyset,0,0)\}
$$
Nient'altro da fare.

\paragraph{$\pmb{M_1}$}
Si aggiunge ad ogni stato di $M_0$ l'elemento $x_1$, formando l'insieme $M_1$.
$$
M_1=M_0\cup (M_0\uplus x_1) = \{(\emptyset,0,0),\;(\{x_1\},100,15)\}
$$
Anche qui, nient'altro da fare.

\paragraph{$\pmb{M_2}$}
Si aggiunge ad ogni stato di $M_1$ l'elemento $x_2$, formando l'insieme $M_2$.
$$
M_2=M_1\cup (M_1\uplus x_2) = \{(\emptyset,0,0),\;(\{x_1\},100,15),\;(\{x_2\},49,12),\;(\{x_1,x_2\},149,27)\}
$$
Nessuno stato dominante. Si passa all'insieme successivo.

\paragraph{$\pmb{M_3}$}
\begin{align*}
M_3=&M_2\cup (M_2\uplus x_3) = \\ =&
							  \{(\emptyset,0,0),\;
							\DA{(\{x_1\},100,15)},\;
								(\{x_2\},49,12),\; 
								(\{x_1,x_2\},149,27),\; \\ &
								(\{x_3\},36,9),\;
								(\{x_1,x_3\},136,24),\;
							\SA{(\{x_2,x_3\},85,21)}
								\}
\end{align*}

\paragraph{$\pmb{M_4}$}
\begin{align*}
M_4=&M_3\cup(M_3\uplus x_4) = \\ =&
							  \{(\emptyset,0,0),\;
							\DA{(\{x_1\},100,15)},\;
								(\{x_2\},49,12),\; 
								(\{x_1,x_2\},149,27),\; \\ &
								(\{x_3\},36,9),\;
								(\{x_1,x_3\},136,24),\;
								(\{x_4\},15,4),\;
								(\{x_1,x_4\},115,19),\;
							\SA{(\{x_2,x_4\},64,16)},\; \\ &
								(\{x_3,x_5\},51,13),\;
								(\{x_1,x_3,x_4\},151,28)
								\}
\end{align*}

\paragraph{$\pmb{M_5}$}
\begin{align*}
M_5=&M_5\cup(M_4\uplus x_5) = \\ =&
							  \{(\emptyset,0,0),\;
							\DA{(\{x_1\},100,15)},\;
								(\{x_2\},49,12),\; 
								(\{x_1,x_2\},149,27),\; \\ &
							\DB{(\{x_3\},36,9)},\;
							\DC{(\{x_1,x_3\},136,24)},\;
								(\{x_4\},15,4),\;
								(\{x_1,x_4\},115,19),\; \\ &
								(\{x_3,x_5\},51,13),\;
								(\{x_1,x_3,x_4\},151,28),\;
								(\{x_5\},18,5),\;
								(\{x_1,x_5\},118,20),\; \\ &
							\SA{(\{x_2,x_5\},67,17)},\;
								(\{x_3,x_5\},54,14),\;
									\opt{(\{x_1,x_3,x_5\},154,29)},\;
							\SB{(\{x_4,x_5\},33,9)},\; \\ &
							\SC{(\{x_1,x_4,x_5\},133,24)},\;
							\SA{(\{x_3,x_4,x_5\},69,18)}
								\}
\end{align*}
Lo stato ottimo � stato evidenziato. Se ne deduce quindi che la soluzione ottima e il suo valore sono:
\begin{align*}
x^*=& (1,0,1,0,1) \\
z^*=& 154
\end{align*}


\end{document}

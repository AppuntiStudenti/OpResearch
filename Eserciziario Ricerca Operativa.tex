\documentclass[a4paper]{report}
\usepackage{amssymb, amsfonts, amsmath, eurosym}
\usepackage{graphicx, import, wrapfig}
\usepackage{fixltx2e}
\usepackage[T1]{fontenc}
\usepackage[latin1]{inputenc}
\usepackage[italian]{babel}
\usepackage{vmargin}
\usepackage[usenames,dvipsnames]{color}
\usepackage[usenames,dvipsnames,svgnames,table]{xcolor}
\usepackage[italian]{varioref}
\usepackage{array}
%\usepackage{booktabs}
\usepackage{tikz,pgfplots,fp,ifthen}
\usepgfplotslibrary{fillbetween}
\usetikzlibrary{shapes,arrows,intersections, patterns}


\usepackage{hyperref}

\renewcommand*\arraystretch{1.5}

\newcommand{\OL}[1]{\overline{#1}}
\newcommand{\st}{\mathrm{s.t.}}
\newcommand{\Sc}[1]{\multicolumn{1}{c|}{#1}}
\newcommand{\Hr}[1]{%
  \colorbox{red!50}{$\displaystyle#1$}}
\newcommand*\C[1]{\tikz[baseline=(char.base)]{
    \node[shape=circle,draw,inner sep=2pt] (char) {#1};}}
    
\newcommand{\intne}[4]{\node [label={[above right=-4pt]45:#3}, name intersections={of=#1 and #2, by=#4}] at (#4) {$\bullet$}}
\newcommand{\intse}[4]{\node [label={[below right=-4pt]-45:#3}, name intersections={of=#1 and #2, by=#4}] at (#4) {$\bullet$}}
\newcommand{\intnw}[4]{\node [label={[above left=-4pt]135:#3}, name intersections={of=#1 and #2, by=#4}] at (#4) {$\bullet$}}
\newcommand{\intsw}[4]{\node [label={[below left=-4pt]-135:#3}, name intersections={of=#1 and #2, by=#4}] at (#4) {$\bullet$}}
\newcommand{\intn}[4]{\node [label={[above=-4pt]90:#3}, name intersections={of=#1 and #2, by=#4}] at (#4) {$\bullet$}}
\newcommand{\ints}[4]{\node [label={[below=-4pt]-90:#3}, name intersections={of=#1 and #2, by=#4}] at (#4) {$\bullet$}}
\newcommand{\inte}[4]{\node [label={[right=-4pt]0:#3}, name intersections={of=#1 and #2, by=#4}] at (#4) {$\bullet$}}
\newcommand{\intw}[4]{\node [label={[left=-4pt]180:#3}, name intersections={of=#1 and #2, by=#4}] at (#4) {$\bullet$}}

%\newcolumntype{C}{>{\centering\arraybackslash$}p{\linewidth}<{$}}

\newcommand{\CG}[1]{\color{green}{#1}}

%\includeonly{capitolo2}

\begin{document}
\setpapersize{A4}
\title{Esercizi di Ricerca Operativa}
\author{Simone Laierno}

\maketitle
\tableofcontents
\chapter*{Introduzione}
Questa �, o almeno si propone di essere, una raccolta degli esercizi proposti a lezione del corso Ricerca Operativa M tenuto dal prof. Silvano Martello all'interno del CdL di Ingegneria Informatica M dell'Universit� di Bologna.

Non ha pretese di esattezza, tutt'altro, ma spero sia d'aiuto a chi segue o seguir� il corso. \textbf{Qualsiasi errore, dubbio, correzione, ecc.} � pi� che bene accetto e pu� essere comunicato privatamente al mio contatto Facebook o al mio indirizzo \textbf{e-mail}: \href{mailto:simonelaierno@gmail.com}{simonelaierno@gmail.com}

\chapter{18/03/2014}

\section{Esercizio 1}

Sia dato - in linguaggio naturale - il seguente problema di ottimizzazione:
\begin{enumerate}
\item Un'azienda realizza due tipi di prodotti X e Y; 
\item Il profitto di 1T di prodotto Y � doppio di quello di 1T di prodotto X; 
\item La produzione di 1T di qualsiasi prodotto richiede 2 ore; 
\item Non si pu� comunque produrre per pi� di 9 ore;
\item Non si possono produrre pi� di 2T di Y; 
\item La produzione di X non pu� superare di pi� di una tonnellata la produzione di Y;
\end{enumerate}
Si modelli il problema come un problema di programmazione lineare, lo si porti in forma standard, si realizzi una rappresentazione grafica del problema e si ottimizzi la funzione di profitto attraverso il \textbf{metodo del simplesso} affinch� \textbf{si ottenga il massimo profitto dalla produzione dei prodotti} nel rispetto dei vincoli assegnati. Si utilizzi la \textit{regola di Dantzig} per scegliere le basi su cui fare pivot.

\subsection{Modellizzazione}
Si indichi con:
\begin{itemize}
\item $x_1$ il numero di tonnellate di prodotto X;
\item $x_2$ il numero di tonnellate di prodotto Y.
\end{itemize}
Lo scopo del nostro problema � di massimizzare i profitti ottenuti dalla produzione. Anche se non siamo a conoscenza degli esatti profitti dati da ogni prodotto, abbiamo comunque a disposizione la relazione data dalla proposizione 2, cio� la variabile $x_2$ rende il doppio della variabile $x_1$. Possiamo quindi esprimere cos� la funzione di profitto:
$$
\max z = x_1 + 2x_2
$$

Modelliamo ora i vincoli espressi dal problema.

Le relazione 3 e 4 ci impongono di non produrre per pi� di 9 ore, considerando che ogni tonnellata di prodotto richiede 2 ore per essere prodotta. Perci� il vincolo sar� espresso come:
$$
2x_1 + 2x_2 \leq 9
$$
La relazione 5 � molto semplice, indica semplicemente che non potremo produrre pi� di due tonnellate di prodotto Y:
$$
x_2 \leq 2
$$
L'ultima relazione (6) ci impone un limite superiore alla produzione del prodotto X, che non deve superare di pi� di una tonnellata la produzione del prodotto Y. Perci�:
$$
x_1 \leq x_2 + 1
$$
Infine imponiamo il vincolo, implicito, che la produzione non pu� essere negativa:
$$
x_1,x_2 \geq0
$$
Il modello matematico pu� essere quindi cos� riassunto:
\begin{align*}
\max z	&= x_1+2x_2 \\
\st\;\;2& x_1+2x_2 \leq 9\\
	  	& x_2 \leq 2\\
	  	& x_1 \leq x_2 + 1\\
	  	& x_1,x_2 \geq 0
\end{align*}

\subsection{Problema in forma grafica}

In figura \vref{fig:graph1} � rappresentato graficamente il problema presentato. In giallo � rappresentato il politopo $P$ e sono stati chiamati $\alpha,\beta,\gamma,\delta,\varepsilon$ i suoi cinque vertici, i quali sappiamo corrispondere ognuno ad una BFS.
Inoltre in figura � riportato il verso del gradiente della funzione obiettivo. Ricordiamo che � necessario che il politopo $P$ sia limitato nella direzione del gradiente o, pi� precisamente, che sia limitato nella direzione opposta al gradiente dopo aver trasformato la funzione obiettivo in una funzione di minimo. Ovviamente i due casi sono gli stessi, basti osservare che il problema � espresso equivalentemente dalle equazioni:
\begin{align*}
\max z&=x_1+2x_2 \\
\min \varphi=-z&=-x_1-2x_2
\end{align*}
I gradienti delle due funzioni $z$ e $\varphi$ sono perci�:
\begin{align*}
\nabla(z)&=\left(\frac{\partial z}{\partial x},\frac{\partial z}{\partial y}\right) = \left(1,2\right) \\
\nabla(\varphi)&=\left(\frac{\partial\varphi}{\partial x},\frac{\partial\varphi}{\partial y}\right) = -\nabla(z) = \left(-1,-2\right)
\end{align*}
I due vettori sono ovviamente uno l'opposto dell'altro e di conseguenza il gradiente di $z$ cresce dove decresce quello di $\varphi$. I problemi sono quindi equivalenti.

\begin{figure}[htbp]
\centering
\begin{tikzpicture}
\begin{axis}
[axis lines=middle, axis equal, enlargelimits, xlabel=$x_1$, ylabel=$x_2$,
 every axis x label/.style={
    at={(ticklabel* cs:1.01)},
    anchor=west,
 },
 every axis y label/.style={
    at={(ticklabel* cs:1.01)},
    anchor=south,
 },]
    \path[name path=AX] 
        (axis cs:\pgfkeysvalueof{/pgfplots/xmin},0)--
        (axis cs:\pgfkeysvalueof{/pgfplots/xmax},0);
    \path[name path=AY] 
        (axis cs:0,\pgfkeysvalueof{/pgfplots/ymin})--
        (axis cs:0,\pgfkeysvalueof{/pgfplots/ymax});
    \path[name path=UP]
    	(axis cs:\pgfkeysvalueof{/pgfplots/xmin},\pgfkeysvalueof{/pgfplots/ymax})--
    	(axis cs:\pgfkeysvalueof{/pgfplots/xmax},\pgfkeysvalueof{/pgfplots/ymax});
\addplot
[domain=0:4.5, samples=10, thick, blue, name path=2x2y9]
{-x+(9/2)} node [pos=0.15,pin={75:{\color{blue}$2x_1+2x_2=9$}}, inner sep=0pt] {};
\addplot
[domain=1:5, samples=10, thick, red, name path=yx-1]
{x-1} node [pos=0.7, anchor=east, pin={165:{\color{red}$x_1=x_2+1$}}, inner sep=0pt] {};
\addplot
[domain=0:5, samples = 10, thick, purple, name path=y2]
{2} node [pos=0.1, anchor=north, pin={90:{\color{purple}$x_2=2$}}, inner sep= 0pt] {};
\addplot[thick, fill=yellow, fill opacity=0.5] fill between [of=2x2y9 and AX, soft clip={domain=0:11/4},];
\addplot[white] fill between [of=2x2y9 and y2];
\addplot[pattern=north west lines, pattern color=red!10] fill between [reverse=true, of=AX and UP, soft clip={domain=0:5}];
\addplot[white] fill between [of=yx-1 and AX];
\addplot[pattern=north east lines, pattern color=blue!10] fill between [of=AX and 2x2y9, soft clip={domain=0:5}];
\addplot[pattern=vertical lines, pattern color=purple!10] fill between [of=AX and y2];
%\node [label={[above right=-6pt]45:$\alpha$}, name intersections={of=AX and AY, by=alp}] at (alp) {$\bullet$};
%\node [label={[below right=-6pt]-45:$\beta$}, name intersections={of=AY and y2, by=bet}] at (bet) {$\bullet$};
%\node [label={[above right=-6pt]45:$\gamma$}, name intersections={of=y2 and 2x2y9, by=gam}] at (gam) {$\bullet$};
%\node [name intersections={of=2x2y9 and yx-1, by=del}, label={[right=-4pt]0:$\delta$}] at (del) {$\bullet$};
%\node [name intersections={of=yx-1 and AX, by=eps}, label={[above=-6pt]90:$\epsilon$}] at (eps) {$\bullet$};
\intse{AX}{AY}{$\alpha$}{alp};
\intse{AY}{y2}{$\beta$}{bet};
\intne{y2}{2x2y9}{$\gamma$}{gam};
\inte{2x2y9}{yx-1}{$\delta$}{del};
\intn{yx-1}{AX}{$\varepsilon$}{eps};
\node at (axis cs:1,1) {$P$};
\addplot[-latex, thick] coordinates
           {(0,0) (1/2.24,2/2.24)} node [pos=1, anchor=north, label={90:{\small $\nabla z$}}] {};
\end{axis}
\end{tikzpicture}
\caption{Rappresentazione cartesiana del problema di programmazione lineare}
\label{fig:graph1}
\end{figure}

\subsection{Forma standard}
Ricordiamo che un problema di \textbf{programmazione lineare in forma standard} � nella forma (matriciale):
\begin{align*}
\min c'x& \\
Ax& = b \\
x& \geq 0
\end{align*}
Cio�, la funzione obiettivo deve essere sotto forma di minimo (il che � molto semplice, dato che basta moltiplicarla per $-1$), i vincoli devono essere tutti espressi sotto forma di \textit{equazioni} e tutte le variabili devono essere positive.
A tal scopo introduciamo una \textbf{variabile slack} per ogni disequazione con simbolo $\leq$.
\begin{alignat*}{7}
&\min \varphi = \quad && -x_1 \quad && -2x_2\\
&\;\st  &&+2x_1		&&+2x_2 		&&+\pmb{x_3}	&&		 		&&\qquad\qquad		&&=9\\
&	 	&&\qquad\qquad &&+x_2		&&\qquad\qquad	&& +\pmb{x_4}	&&					&&=2\\
&	 	&&+x_1		&&-x_2 \qquad	&&				&&\qquad\qquad	&&+\pmb{x_5}		&&=1\\
&		&&\quad\; x_1,	&&\quad\; x_2,		&&\quad\; x_3,		&&\quad\; x_4,		&&\quad\; x_5		&&\geq 0
\end{alignat*}

\subsection{Risoluzione tramite tableau}

\subsubsection{Richiami (molto blandi) di teoria}
Una \textbf{base} � determinata da una sottomatrice della matrice A dei vincoli, di dimensione $m\times n$, di $n$ colonne linearmente indipendenti (nel \textbf{tableau} la matrice A � quella delimitata dalle due righe disegnate). Spesso l'individuazione delle colonne della prima base � semplice perch� l'introduzione di variabili slack o di variabili surplus crea nella nostra matrice delle colonne con tutti 0 e solo un 1, il che rende probabile la formazione di una sottomatrice \textbf{identit�}.
Ricordiamo che scelta una base $\mathcal{B}$ tale che:
$$
\mathcal{B}=A_{\beta(i)}; \quad i=1,\ldots,m
$$
ad essa � associata una \textbf{soluzione base} $x$ tale che:
$$
x=x_j; \quad j=1,\ldots,n \\
x_j = 0 \quad \forall j : A_j\not\in \mathcal{B}
$$
cio� il valore di una variabile non in base � 0. Questa � inoltre detta una soluzione \textbf{ammissibile} (\textbf{BFS}) se si trova nelle regione ammissibile determinata dai vincoli.

Al tableau aggiungeremo in alto una riga che indicher� il \textbf{costo relativo} $\OL{c_j}$ della colonna $j$-esima. Basti sapere che se facciamo in modo che $\forall A_j \in \mathcal{B} : \OL{c_j} = 0$, avremo nella prima colonna il guadagno $-\varphi$ della funzione obiettivo e in tutti gli altri avremo effettivamente i costi relativi. Per una spiegazione del perch� di questo fenomeno magico, si rimanda al testo o alle slide del docente.

Si ricorda, infine, che la colonna $b$ dei termini noti verr� inserita a sinistra. Non � indispensabile, ma una semplice convenzione.

\subsubsection{Risoluzione}

Per realizzare il \textbf{tableau} � sufficiente ricordare le regole base. La matrice $A$ e la colonna $b$ si riportano fedelmente sotto la loro consueta forma di matrice. Le variabili $x_j$ sono i coefficienti della rispettiva variabile nella funzione obiettivo. Ovviamente per tutte le variabili slack e surplus, che sono state aggiunte artificialmente da noi, il loro valore � 0. 

La prima colonna, una volta scelta una base che ha la forma di una matrice identit� (e per ora assumeremo che sia sempre gi� pronta o facilmente costruibile) rappresenta banalmente la soluzione del sistema $Ix=b$ che altro non � che $b$ stesso.

L'ultimo valore da inserire � quello di $-\varphi$, che varie a seconda delle colonne che assumeremo inizialmente come base. Se, come spesso accadr�, scegliamo tutte colonne associate a variabili slack o surplus, il loro valore non influir� sulla funzione obiettivo ed essendo tutte le altre variabili automaticamente nulle perch� non sono in base, sar� nullo anche $-\varphi$. Il nostro caso attuale ricade in quest'ultimo descritto, ma se fosse stato altrimenti, avremmo semplicemente dovuto calcolare il valore di $-\varphi$ in base al valore delle variabili $x_1$ e $x_2$.

\begin{table}[htbp]
\centering
\begin{tabular}{rcccccc}
 & $-\varphi$ & $x_1$ & $x_2$ & $x_3$ & $x_4$ & $x_5$ \\
$\OL{c_j}$ & \Sc{0} & -1 & -2 & 0 & 0 & 0 \\
\cline{2-7}
$x_3$ & \Sc{9} & 2 & 2 & 1 & 0 & 0 \\
$x_4$ & \Sc{2} & 0 & 1 & 0 & 1 & 0 \\
$x_5$ & \Sc{1} & 1 & -1 & 0 & 0 & 1 \\
\end{tabular}
\caption{Tableau iniziale. Vertice $\alpha(0,0)$}
\label{tab:tab1}
\end{table}

In tabella \vref{tab:tab1} il tableau definitivo ricavato dal nostro problema. Si noti che le ultime 3 colonne formano gi� una matrice identit�, perci� le assumeremo come base.
\begin{align*}
\mathcal{B}&=\{A_3,A_4,A_5\}\\
x&=(0,0,9,2,1)
\end{align*}

Per sapere in che punto dello spazio originale a due dimensioni ci troviamo, � sufficiente guardare le variabili $x_1$ e $x_2$. \'E evidente che ci troviamo nell'origine, che appartiene al politopo $P$ trovato in precedenza e che in particolare � il \textbf{vertice} $\pmb{\alpha}$.
Poich� non tutti i $\OL{c_j}$ sono non negativi, la nostra non � la BFS ottima e dobbiamo muoverci in una BFS migliore. Applicando la \textbf{regola di Dantzig}, facciamo entrare in base la colonna con il costo relativo maggiore in valore assoluto (cio� il "pi� negativo"). Nel nostro caso, prenderemo in considerazione quindi la colonna $\pmb{A_2}$.
Per scegliere su quale elemento fare \textbf{pivoting}, dobbiamo ottenere il valore di $y_{\ell 2}$ tale che:
$$
\vartheta_{\max}=\min_{i:y_{i2}>0}\frac{y_{i0}}{y_{i2}}=\frac{y_{i0}}{y_{\ell 2}}
$$
Perci�, operando con gli elementi nel tableau:
\begin{align*}
\vartheta_{\max}=\min\left(\frac{9}{2},\frac{2}{1}\right)=\frac{2}{1}=\frac{y_{20}}{\pmb{y_{22}}}
\end{align*}
Faremo pivoting sull'elemento $y_{22}$ (cerchiato in tabella \vref{tab:tab2}). Il nostro scopo � ora far comparire uno 0 nella colonna dell'elemento pivot in tutte le righe tranne quella in cui si trova l'elemento pivot e far comparire un 1 in quest'ultima.

\begin{table}[htbp]
\centering
\begin{tabular}{rrcccccc}
 & & $-\varphi$ & $x_1$ & $x_2$ & $x_3$ & $x_4$ & $x_5$ \\
$R_0$ & $\OL{c_j}$ & \Sc{0} & -1 & -2 & 0 & 0 & 0 \\
\cline{3-8}
$R_1$ & $x_3$ & \Sc{9} & 2 & 2 & 1 & 0 & 0 \\
$R_2$ & $x_4$ & \Sc{2} & 0 & \C{1} & 0 & 1 & 0 \\
$R_3$ & $x_5$ & \Sc{1} & 1 & -1 & 0 & 0 & 1 \\
\end{tabular}
\caption{Pivoting su $y_{22}$. $A_2$ entra in base e $A_4$ esce.}
\label{tab:tab2}
\end{table}

Possiamo felicemente notare che $y_{22}=1$, perci� nulla da fare su $R_2$. Se cos� non fosse stato sarebbe bastato moltiplicare $R_2R$ per un coefficiente $h$. Algebricamente, per far comparire uno 0 in tutti gli altri elementi della colonna $A_2$, possiamo sostituire ad ogni riga la riga stessa sommata ad un'altra qualsiasi riga moltiplicata per un coefficiente $k$. Ovviamente la riga pi� comoda da sommare � la riga su cui stiamo facendo pivot $R_\ell$, avendo un comodissimo 1 nella colonna interessata. Perci� possiamo riassumere che l'operazione consentita su ogni riga $R_i$ e sulla riga di pivot $R_\ell$�:
\begin{align*}
R_\ell&\leftarrow hR_l \\
R_i&\leftarrow R_i+kR_\ell
\end{align*}
Queste sono dette \textbf{operazioni elementari di riga}. Applicando le regole al nostra tableau, operiamo:
\begin{align*}
R_0&\leftarrow R_0 + 2R_2; \\
R_1&\leftarrow R_1 - 2R_2; \\
R_3&\leftarrow R_3 + R_2
\end{align*}
Il nostro nuovo tableau diventa quindi quello in tabella \vref{tab:tab3}.

\begin{table}[htbp]
\centering
\begin{tabular}{rrcccccc}
 & & $-\varphi$ & $x_1$ & $x_2$ & $x_3$ & $x_4$ & $x_5$ \\
$R_0$ & $\OL{c_j}$ & \Sc{4} & -1 & 0 & 0 & 2 & 0 \\
\cline{3-8}
$R_1$ & $x_3$ & \Sc{5} & 2 & 0 & 1 & -2 & 0 \\
$R_2$ & $x_2$ & \Sc{2} & 0 & 1 & 0 & 1 & 0 \\
$R_3$ & $x_5$ & \Sc{3} & 1 & 0 & 0 & 1 & 1 \\
\end{tabular}
\caption{Secondo tableau. Vertice $\beta(0,2)$}
\label{tab:tab3}
\end{table}

Ora che $A_4$ � entrato in base e $A_2$ ne � uscito, abbiamo una nuova base $\mathcal{B}$ e una nuova BFS $x$:
\begin{align*}
\mathcal{B}&=\{A_3,A_2,A_5\} \\
x&=(0,2,5,0,3)
\end{align*}
Ci troviamo nel \textbf{vertice} $\pmb{\beta}$, ma questa non � ancora la BFS ottima. Possiamo osservare, infatti, che la colonna $A_1$ presenta un $\OL{c_j}$ negativo e sar� necessario fare ulteriormente pivoting su un elemento di questa. Otteniamo quindi il valore di $y_{\ell 1}$ tale che:
\begin{align*}
\vartheta_{\max}&=\min_{i:y_{i1}>0}\frac{y_{i0}}{y_{i1}}=\frac{y_{i0}}{y_{\ell 1}} \\
\vartheta_{\max}&=\min\left(\frac{5}{2},\frac{3}{1}\right)=\frac{5}{2}=\frac{y_{10}}{\pmb{y_{11}}}
\end{align*}
Faremo pivoting sull'elemento $y_{11}$ (cerchiato in tabella \vref{tab:tab4}). 
\begin{table}[htbp]
\centering
\begin{tabular}{rrcccccc}
 & & $-\varphi$ & $x_1$ & $x_2$ & $x_3$ & $x_4$ & $x_5$ \\
$R_0$ & $\OL{c_j}$ & \Sc{4} & -1 & 0 & 0 & 2 & 0 \\
\cline{3-8}
$R_1$ & $x_3$ & \Sc{5} & \C{2} & 0 & 1 & -2 & 0 \\
$R_2$ & $x_2$ & \Sc{2} & 0 & 1 & 0 & 1 & 0 \\
$R_3$ & $x_5$ & \Sc{3} & 1 & 0 & 0 & 1 & 1 \\
\end{tabular}
\caption{Pivoting su $y_{11}$. $A_1$ entra in base e $A_3$ esce.}
\label{tab:tab4}
\end{table}

Questa volta dobbiamo lavorare anche sulla riga dell'elemento pivot, dividendola per 2:
$$
R_1\rightarrow \frac{1}{2}R_1
$$
Partendo dal nuovo tableau in tabella \vref{tab:tab5}, facciamo pivoting sulle restanti righe in questo modo:
\begin{align*}
R_0&\rightarrow R_0 + R_1 \\
R_3&\rightarrow R_3 - R_1
\end{align*}
\begin{table}[htbp]
\centering
\begin{tabular}{rrcccccc}
 & & $-\varphi$ & $x_1$ & $x_2$ & $x_3$ & $x_4$ & $x_5$ \\
$R_0$ & $\OL{c_j}$ & \Sc{4} & -1 & 0 & 0 & 2 & 0 \\
\cline{3-8}
$R_1$ & $x_1$ & \Sc{$\frac{5}{2}$} & \C{1} & 0 & $\frac{1}{2}$ & -1 & 0 \\
$R_2$ & $x_2$ & \Sc{2} & 0 & 1 & 0 & 1 & 0 \\
$R_3$ & $x_5$ & \Sc{3} & 1 & 0 & 0 & 1 & 1 \\
\end{tabular}
\caption{Pivoting su $y_{11}$. $R_1$ divisa per 2.}
\label{tab:tab5}
\end{table}
Il nostro nuovo tableau, quindi, � quello in tabella \vref{tab:tab6}. 
\begin{table}[htbp]
\centering
\begin{tabular}{rrcccccc}
 & & $-\varphi$ & $x_1$ & $x_2$ & $x_3$ & $x_4$ & $x_5$ \\
$R_0$ & $\OL{c_j}$ & \Sc{$\frac{13}{2}$} & 0 & 0 & $\frac{1}{2}$ & 1 & 0 \\
\cline{3-8}
%\hline
$R_1$ & $x_1$ & \Sc{$\frac{5}{2}$} & 1 & 0 & $\frac{1}{2}$ & -1 & 0 \\
$R_2$ & $x_2$ & \Sc{2} & 0 & 1 & 0 & 1 & 0 \\
$R_3$ & $x_5$ & \Sc{$\frac{1}{2}$} & 0 & 0 & $-\frac{1}{2}$ & 2 & 1 \\
\end{tabular}
\caption{Terzo tableau. Vertice $\gamma\left(\frac{5}{2},2\right)$}
\label{tab:tab6}
\end{table}
Notiamo che tutti i $\OL{c_j}$ sono non negativi, perci� ci troviamo nella \textbf{BFS ottima}. La base $\mathcal{B}$ e la soluzione $x$ sono quindi:
\begin{align*}
\mathcal{B}&={A_1,A_2,A_5} \\
x&=\left(\frac{5}{2},2,0,0,\frac{1}{2}\right)
\end{align*}
La soluzione ottima � quella del vertice $\pmb{\gamma\left(\frac{5}{2},2\right)}$. Riassumendo, tutti i valori delle variabili in gioco sono i seguenti:
\begin{align*}
z&=-\varphi=\frac{13}{2}=6.5 \\
x_1&=\frac{5}{2}=2.5 \\
x_2&=2
\end{align*}
\subsection{Conclusione}
La soluzione ottima consiste nel produrre $2.5$T di prodotto X e $2$T di prodotto Y, ottenendo un \textbf{profitto} pari a $6.5$ volte quello di $1$T di prodotto X.

\section{Esercizio 2}

Sia dato - in linguaggio naturale - il seguente problema di ottimizzazione:
\begin{enumerate}
\item Un'azienda chimica produce due composti, 1 e 2, composti da due sostanze chimiche A e B;
\item Un lotto di composto 1 richiede $3$T di sostanza A e $3$T di sostanza B;
\item Un lotto di composto 2 richiede $6$T di sostanza A e $3$T di sostanza B;
\item Per motivi di mercato non si possono produrre pi� di 3 lotti di composto 1;
\item Si hanno a disposizione $12$T di composto A e $9$T di composto B;
\item Il profitto di un lotto di composto A � di $12\,000$\euro ;
\item Il profitto di un lotto di composto B � di $15\,000$\euro .
\end{enumerate}
Si modelli il problema come un problema di programmazione lineare, lo si porti in forma standard, si realizzi una rappresentazione grafica del problema e si ottimizzi la funzione di profitto attraverso il \textbf{metodo del simplesso} affinch� \textbf{si ottenga il massimo profitto dalla produzione dei prodotti} nel rispetto dei vincoli assegnati. Si utilizzi la \textit{regola di Bland} per scegliere le basi su cui fare pivot.

\subsection{Modellizzazione}
Si indichi con:
\begin{itemize}
\item $x_1$ il numero di lotti di composto 1;
\item $x_2$ il numero di lotti di composto 2.
\end{itemize}
Lo scopo del nostro problema � di massimizzare i profitti ottenuti dalla produzione. Per comodit� di rappresentazione, stabiliamo che la funzione di profitto $z$ esprima il profitto in \textbf{migliaia di euro}:
$$
\max z = 12x_1 + 15x_2
$$

Modelliamo ora i vincoli espressi dal problema.

Dalle relazioni 3, 4 e 5 possiamo dedurre i seguenti vincoli:
\begin{itemize}
\item Servono $3$T di prodotto A per produrre un lotto di composto 1 e $6$T per produrre un lotto di composto 2. In tutto non possiamo utilizzare pi� di $12$T di prodotto A.
\item Servono $3$T di prodotto B per produrre un lotto di composto 1 e $3$T per produrre un lotto di composto 2. In tutto non possiamo utilizzare pi� di $9$T di prodotto B.
\end{itemize}
\begin{align*}
3x_1 + 6x_2 &\leq 12 \\
3x_1 + 3x_2 &\leq 9
\end{align*}
La relazione 4 � cos� facilmente esprimibile:
$$
x_1 \leq 3
$$
Infine imponiamo il vincolo, implicito, che la produzione non pu� essere negativa:
$$
x_1,x_2 \geq0
$$
Il modello matematico pu� essere quindi cos� riassunto (sono state apportate semplificazione algebriche):
\begin{align*}
\max z	&= 12x_1+15x_2 \\
\st\;\; & x_1+2x_2 \leq 4\\
	  	& x_1+x_2 \leq 3\\
	  	& x_1 \leq 3\\
	  	& x_1,x_2 \geq 0
\end{align*}

\subsection{Problema in forma grafica}

In figura \vref{fig:graph2} � rappresentato graficamente il problema presentato. In giallo � rappresentato il politopo $P$ e sono stati chiamati $\alpha,\beta,\gamma,\delta$ i suoi quattro vertici, i quali sappiamo corrispondere ognuno ad una BFS.
Il gradiente della funzione obiettivo vale
\begin{equation*}
\nabla(z)=\left(\frac{\partial z}{\partial x},\frac{\partial z}{\partial y}\right) = \left(12,15\right) \\
\end{equation*}
Il politopo $P$ �, ovviamente, limitato nella direzione del gradiente (si fa notare che finora $P$ � sempre limitato in ogni direzione, quindi qualsiasi direzione avesse il gradiente non ci sarebbero problemi).

\begin{figure}[htbp]
\centering
\begin{tikzpicture}
\begin{axis}
[axis lines=middle, axis equal, enlargelimits, xlabel=$x_1$, ylabel=$x_2$,
 every axis x label/.style={
    at={(ticklabel* cs:1.01)},
    anchor=west,
 },
 every axis y label/.style={
    at={(ticklabel* cs:1.01)},
    anchor=south,
 },]
    \path[name path=AX] 
        (axis cs:\pgfkeysvalueof{/pgfplots/xmin},0)--
        (axis cs:\pgfkeysvalueof{/pgfplots/xmax},0);
    \path[name path=AY] 
        (axis cs:0,\pgfkeysvalueof{/pgfplots/ymin})--
        (axis cs:0,\pgfkeysvalueof{/pgfplots/ymax});
    \path[name path=UP]
    	(axis cs:\pgfkeysvalueof{/pgfplots/xmin},\pgfkeysvalueof{/pgfplots/ymax})--
    	(axis cs:\pgfkeysvalueof{/pgfplots/xmax},\pgfkeysvalueof{/pgfplots/ymax});
\addplot
[domain=0:4, samples=10, thick, blue, name path=x2y4]
{-.5*x+2} node [pos=0.15,pin={75:{\color{blue}$x_1+2x_2=4$}}, inner sep=0pt] {};
\addplot
[domain=0:3, samples=10, thick, red, name path=xy3]
{-x+3} node [pos=0.1, pin={75:{\color{red}$x_1+x_2=3$}}, inner sep=0pt] {};
\addplot
[domain=0:5, samples = 10, thick, purple, name path=x3]
(3,x) node [pos=0.5, anchor=north, pin={0:{\color{purple}$x_1=3$}}, inner sep= 0pt] {};
\addplot[thick, fill=yellow, fill opacity=0.5] fill between [of=x2y4 and AX, soft clip={domain=0:3}];
\addplot[white] fill between [of=xy3 and UP];
%\addplot[pattern=north east lines, pattern color=red!10] fill between [reverse=true, of=AX and UP, soft clip={domain=0:5}];
%\addplot[white] fill between [of=xy3 and AX];
\addplot[pattern=north east lines, pattern color=red!10] fill between [of=xy3 and AX];
\addplot[pattern=vertical lines, pattern color=blue!10] fill between [of=x2y4 and AX];
%\addplot[pattern=north east lines, pattern color=blue!10] fill between [of=AX and 2x2y9, soft clip={domain=0:5}];
\addplot[pattern=horizontal lines, pattern color=purple!10] fill between [of=AY and x3];
\intse{AX}{AY}{$\alpha$}{alp};
\intne{AY}{x2y4}{$\beta$}{bet};
\intne{x2y4}{xy3}{$\gamma$}{gam};
\intne{xy3}{AX}{$\delta$}{del};
\node at (axis cs:1.5,.5) {$P$};
\addplot[-latex, thick] coordinates
           {(0,0) (4/6.4,5/6.4)} node [pos=1, anchor=north, label={90:{\small $\nabla z$}}] {};
\end{axis}
\end{tikzpicture}
\caption{Rappresentazione cartesiana del problema di programmazione lineare}
\label{fig:graph2}
\end{figure}

\subsection{Forma standard}
Ricordiamo che un problema di \textbf{programmazione lineare in forma standard} � nella forma (matriciale):
\begin{align*}
\min c'x& \\
Ax& = b \\
x& \geq 0
\end{align*}
Trasformiamo la funzione obiettivo $z$ in $\varphi$ tale che:
\begin{equation*}
\varphi=-\frac{z}{3}=-4x_1-5x_2
\end{equation*}
Quindi introduciamo una \textbf{variabile slack} per ogni disequazione con simbolo $\leq$. Otterremo infine:
\begin{alignat*}{7}
&\min \varphi = \quad && -4x_1 \quad && -5x_2\\
&\;\st  &&+x_1		&&+2x_2 		&&+\pmb{x_3}	&&		 		&&\qquad\qquad		&&=4\\
&	 	&&+x_1		&&+x_2			&&\qquad\qquad	&& +\pmb{x_4}	&&					&&=3\\
&	 	&&+x_1		&&\qquad\qquad	&&				&&\qquad\qquad	&&+\pmb{x_5}		&&=3\\
&		&&\quad\; x_1,	&&\quad\; x_2,		&&\quad\; x_3,		&&\quad\; x_4,		&&\quad\; x_5		&&\geq 0
\end{alignat*}

\subsection{Risoluzione tramite tableau}

\begin{table}[htbp]
\centering
\begin{tabular}{rcccccc}
			&$-\varphi$ & $x_1$ & $x_2$ & $x_3$ & $x_4$ & $x_5$ \\
$\OL{c_j}$ 	& \Sc{0} 	& -4 	& -5 	& 0 	& 0 	& 0 \\
\cline{2-7}
$x_3$ 		& \Sc{4} 	& 1 	& 2 	& 1 	& 0 	& 0 \\
$x_4$ 		& \Sc{3} 	& 1 	& 1		& 0 	& 1 	& 0 \\
$x_5$ 		& \Sc{3} 	& 1 	& 0 	& 0 	& 0 	& 1 \\
\end{tabular}
\caption{Tableau iniziale. Vertice $\alpha(0,0)$}
\label{tab:tab21}
\end{table}

In tabella \vref{tab:tab21} il tableau ricavato dal nostro problema. Si noti che le ultime 3 colonne formano gi� una matrice identit�, perci� le assumeremo come base.
\begin{align*}
\mathcal{B}&=\{A_3,A_4,A_5\}\\
x&=(0,0,4,3,3)
\end{align*}
Ci troviamo nell'origine, che appartiene al politopo $P$ trovato in precedenza e che in particolare � il \textbf{vertice} $\pmb{\alpha}$.
Poich� non tutti i $\OL{c_j}$ sono non negativi, la nostra non � la BFS ottima e dobbiamo muoverci in una BFS migliore. Applicando la \textbf{regola di Bland}, facciamo entrare in base la colonna con l'indice minore. Nel nostro caso, prenderemo in considerazione quindi la colonna $\pmb{A_1}$.
Per scegliere su quale elemento fare \textbf{pivoting}, dobbiamo ottenere il valore di $y_{\ell 1}$ tale che:
$$
\vartheta_{\max}=\min_{i:y_{i1}>0}\frac{y_{i0}}{y_{i1}}=\frac{y_{i0}}{y_{\ell 1}}
$$
Perci�, operando con gli elementi nel tableau:
\begin{align*}
\vartheta_{\max}=\min\left(\frac{4}{1},\frac{3}{1},\frac{3}{1}\right)=\frac{3}{1}=\frac{y_{20}}{\pmb{y_{21}}}=\frac{y_{30}}{\pmb{y_{31}}}
\end{align*}
Abbiamo un \textbf{pareggio} tra gli elementi $y_{21}$ e $y_{31}$.Seguendo la \textbf{regola di Bland}, sceglieremo come pivot l'elemento che far� uscire dalla base la variabile con l'\textbf{indice minore}%
\footnote{Si fa notare che l'utilizzo della \textbf{regola di Bland} evita i casi di \textbf{loop} in presenza di basi degeneri durante l'algoritmo del simplesso. Questa propriet� � dimostrabile ma la dimostrazione esula dai nostri scopi.}.
Faremo quindi pivoting sull'elemento $y_{21}$ (cerchiato in tabella \vref{tab:tab22}) poich� far� uscire dalla base la variabile $x_4$. Il nostro scopo � ora far comparire uno 0 nella colonna dell'elemento pivot in tutte le righe tranne quella in cui si trova l'elemento pivot e far comparire un 1 in quest'ultima.
\begin{table}[htbp]
\centering
\begin{tabular}{rrcccccc}
 	  & 			&$-\varphi$ & $x_1$ & $x_2$ & $x_3$ & $x_4$ & $x_5$ \\
$R_0$ & $\OL{c_j}$ 	& \Sc{0} 	& -4 	& -5 	& 0 	& 0 	& 0 \\
\cline{3-8}
$R_1$ & $x_3$ 		& \Sc{4} 	& 1 	& 2 	& 1 	& 0 	& 0 \\
$R_2$ & $x_4$ 		& \Sc{3} 	& \C{1}	& 1		& 0 	& 1 	& 0 \\
$R_3$ & $x_5$ 		& \Sc{3} 	& 1 	& 0 	& 0 	& 0 	& 1 \\
\end{tabular}
\caption{Pivoting su $y_{21}$. $A_1$ entra in base e $A_4$ esce.}
\label{tab:tab22}
\end{table}
Poich� $y_{21}=1$ non c'� nulla da fare su $R_2$. Applichiamo le operazioni elementari di riga al nostro tableau come segue:
\begin{align*}
R_0&\leftarrow R_0 + 4R_2; \\
R_1&\leftarrow R_1 - R_2; \\
R_3&\leftarrow R_3 - R_2
\end{align*}
Il nostro nuovo tableau diventa quindi quello in tabella \vref{tab:tab23}.
\begin{table}[htbp]
\centering
\begin{tabular}{rrcccccc}
 	  & 			&$-\varphi$ & $x_1$ & $x_2$ & $x_3$ & $x_4$ & $x_5$ \\
$R_0$ & $\OL{c_j}$ 	& \Sc{12} 	& 0 	& -1 	& 0 	& 4 	& 0 \\
\cline{3-8}
$R_1$ & $x_3$ 		& \Sc{1} 	& 0 	& 1 	& 1 	& -1 	& 0 \\
$R_2$ & $x_1$ 		& \Sc{3} 	& 1		& 1		& 0 	& 1 	& 0 \\
$R_3$ & $x_5$ 		& \Sc{0} 	& 0 	& -1 	& 0 	& -1 	& 1 \\
\end{tabular}
\caption{Secondo tableau. Vertice $\delta(3,0)$}
\label{tab:tab23}
\end{table}

Ora che $A_1$ � entrato in base e $A_4$ ne � uscito, abbiamo una nuova base $\mathcal{B}$ e una nuova BFS $x$:
\begin{align*}
\mathcal{B}&=\{A_3,A_1,A_5\} \\
x&=(3,0,1,0,0)
\end{align*}
Ci troviamo nel \textbf{vertice} $\pmb{\delta}$, ma questa non � ancora la BFS ottima. Inoltre, ci troviamo nel caso di una \textbf{base degenere}: la variabile $x_5$, che � in base, ha valore nullo. Questo non dovrebbe comunque crearci problemi in quanto stiamo applicando la regola di Bland.
L'unica colonna ad avere un $\OL{c_j}$ negativo � $A_2$ ed � su questa che cercheremo l'elemento di pivot $y_{\ell 2}$:
\begin{align*}
\vartheta_{\max}&=\min_{i:y_{i2}>0}\frac{y_{i0}}{y_{i2}}=\frac{y_{i0}}{y_{\ell 2}} \\
\vartheta_{\max}&=\min\left(\frac{1}{1},\frac{3}{1}\right)=\frac{1}{1}=\frac{y_{10}}{\pmb{y_{12}}}
\end{align*}
Faremo pivoting sull'elemento $y_{12}$ (cerchiato in tabella \vref{tab:tab24}). 
\begin{table}[htbp]
\centering
\begin{tabular}{rrcccccc}
 	  & 			&$-\varphi$ & $x_1$ & $x_2$ & $x_3$ & $x_4$ & $x_5$ \\
$R_0$ & $\OL{c_j}$ 	& \Sc{12} 	& 0 	& -1 	& 0 	& 4 	& 0 \\
\cline{3-8}
$R_1$ & $x_3$ 		& \Sc{1} 	& 0 	& \C{1}	& 1 	& -1 	& 0 \\
$R_2$ & $x_1$ 		& \Sc{3} 	& 1		& 1		& 0 	& 1 	& 0 \\
$R_3$ & $x_5$ 		& \Sc{0} 	& 0 	& -1 	& 0 	& -1 	& 1 \\
\end{tabular}
\caption{Pivoting su $y_{12}$. $A_2$ entra in base e $A_3$ esce.}
\label{tab:tab24}
\end{table}
Le operazione di pivoting saranno:
\begin{align*}
R_0&\rightarrow R_0 + R_1 \\
R_2&\rightarrow R_2 - R_1 \\
R_3&\rightarrow R_3 + R_1
\end{align*}
Il nostro nuovo tableau, quindi, � quello in tabella \vref{tab:tab25}. 
\begin{table}[htbp]
\centering
\begin{tabular}{rrcccccc}
 	  & 			&$-\varphi$ & $x_1$ & $x_2$ & $x_3$ & $x_4$ & $x_5$ \\
$R_0$ & $\OL{c_j}$ 	& \Sc{13} 	& 0 	& 0 	& 1 	& 3 	& 0 \\
\cline{3-8}
$R_1$ & $x_2$ 		& \Sc{1} 	& 0 	& 1 	& 1 	& -1 	& 0 \\
$R_2$ & $x_1$ 		& \Sc{2} 	& 1		& 0		& -1 	& 2 	& 0 \\
$R_3$ & $x_5$ 		& \Sc{1} 	& 0 	& 0 	& 1 	& -2 	& 1 \\
\end{tabular}
\caption{Terzo tableau. Vertice $\gamma(2,1)$}
\label{tab:tab25}
\end{table}
Notiamo che tutti i $\OL{c_j}$ sono non negativi, perci� ci troviamo nella \textbf{BFS ottima}. La base $\mathcal{B}$ e la soluzione $x$ sono quindi:
\begin{align*}
\mathcal{B}&={A_2,A_1,A_5} \\
x&=(2,1,0,0,1)
\end{align*}
La soluzione ottima � quella del vertice $\pmb{\gamma(2,1)}$. Riassumendo, tutti i valori delle variabili in gioco sono i seguenti:
\begin{align*}
z&=-3\varphi=-3(-13)=39 \\
x_1&=2 \\
x_2&=1
\end{align*}
\subsection{Conclusione}
La soluzione ottima consiste nel produrre $2$ lotti di composto 1 e $1$ lotto di composto 2, ottenendo un \textbf{profitto} pari a $39\,000$\euro .

\section{Esercizio 3}

Sia dato - in linguaggio naturale - il seguente problema di ottimizzazione:
\begin{enumerate}
\item Un'azienda produce due tipi di composto A e B;
\item Il profitto del composto A � il doppio di quello del composto B;
\item Per motivi di mercato non si possono produrre pi� di $2$T di composto A;
\item Ogni tonnellata di ogni composto contiene $1$Q di sostanza base;
\item Ho a disposizione $3$Q di sostanza base;
\item $1$T di composto A contiene $1$Q di sostanza chimica;
\item $1$T di composto B contiene $2$Q di sostanza chimica;
\item Ho a disposizione $5$Q di sostanza chimica.
\end{enumerate}
Si modelli il problema come un problema di programmazione lineare, lo si porti in forma standard, si realizzi una rappresentazione grafica del problema e si ottimizzi la funzione di profitto attraverso il \textbf{metodo del simplesso} affinch� \textbf{si ottenga il massimo profitto dalla produzione dei prodotti} nel rispetto dei vincoli assegnati. Si utilizzi la \textit{regola di Dantzig} per scegliere le basi su cui fare pivot.

\subsection{Modellizzazione}
Si indichi con:
\begin{itemize}
\item $x_1$ il numero di tonnellate di composto 1;
\item $x_2$ il numero di tonnellate di composto 2.
\end{itemize}
Lo scopo del nostro problema � di massimizzare i profitti ottenuti dalla produzione. Anche se non siamo a conoscenza degli esatti profitti dati da ogni prodotto, abbiamo comunque a disposizione la relazione data dalla proposizione 2, cio� la variabile $x_1$ rende il doppio della variabile $x_2$. Possiamo quindi esprimere cos� la funzione di profitto:
$$
\max z = 2x_1 + x_2
$$

Modelliamo ora i vincoli espressi dal problema.

La relazione 3 � cos� facilmente esprimibile:
$$
x_1 \leq 2
$$
Dalle relazioni 4 e 5 possiamo dedurre il seguente vincolo:
\begin{equation*}
x_1 + x_2 \leq 3
\end{equation*}
Dalle relazioni 6, 7 e 8 possiamo dedurre il seguente vincolo:
\begin{equation*}
x_1 + 2x_2 \leq 5
\end{equation*}
Infine imponiamo il vincolo, implicito, che la produzione non pu� essere negativa:
$$
x_1,x_2 \geq0
$$
Il modello matematico pu� essere quindi cos� riassunto (sono state apportate semplificazione algebriche):
\begin{align*}
\max z	&= 2x_1+x_2 \\
\st\;\; & x_1 \leq 2\\
	  	& x_1+x_2 \leq 3\\
		& x_1+2x_2 \leq 5\\
	  	& x_1,x_2 \geq 0
\end{align*}

\subsection{Problema in forma grafica}

In figura \vref{fig:graph3} � rappresentato graficamente il problema presentato. In giallo � rappresentato il politopo $P$ e sono stati chiamati $\alpha,\beta,\gamma,\delta,\varepsilon$ i suoi cinque vertici, i quali sappiamo corrispondere ognuno ad una BFS.
Il gradiente della funzione obiettivo vale
\begin{equation*}
\nabla(z)=\left(\frac{\partial z}{\partial x},\frac{\partial z}{\partial y}\right) = \left(2,1\right) \\
\end{equation*}
Il politopo $P$ �, ovviamente, limitato nella direzione del gradiente (si fa notare che finora $P$ � sempre limitato in ogni direzione, quindi qualsiasi direzione avesse il gradiente non ci sarebbero problemi).

\begin{figure}[htbp]
\centering
\begin{tikzpicture}
\begin{axis}
[axis lines=middle, axis equal, enlargelimits, xlabel=$x_1$, ylabel=$x_2$,
 every axis x label/.style={
    at={(ticklabel* cs:1.01)},
    anchor=west,
 },
 every axis y label/.style={
    at={(ticklabel* cs:1.01)},
    anchor=south,
 },]
    \path[name path=AX] 
        (axis cs:\pgfkeysvalueof{/pgfplots/xmin},0)--
        (axis cs:\pgfkeysvalueof{/pgfplots/xmax},0);
    \path[name path=AY] 
        (axis cs:0,\pgfkeysvalueof{/pgfplots/ymin})--
        (axis cs:0,\pgfkeysvalueof{/pgfplots/ymax});
    \path[name path=UP]
    	(axis cs:\pgfkeysvalueof{/pgfplots/xmin},\pgfkeysvalueof{/pgfplots/ymax})--
    	(axis cs:\pgfkeysvalueof{/pgfplots/xmax},\pgfkeysvalueof{/pgfplots/ymax});
\addplot
[domain=0:4, samples=10, thick, blue, name path=x2y5]
{-.5*x+2.5} node [pos=0.8,pin={75:{\color{blue}$x_1+2x_2=5$}}, inner sep=0pt] {};
\addplot
[domain=0:3, samples=10, thick, red, name path=xy3]
{-x+3} node [pos=0.2, pin={85:{\color{red}$x_1+x_2=3$}}, inner sep=0pt] {};
\addplot
[domain=0:5, samples = 10, thick, purple, name path=x2]
(2,x) node [pos=0.5, anchor=north, pin={0:{\color{purple}$x_1=2$}}, inner sep= 0pt] {};
\addplot[thick, fill=yellow, fill opacity=0.5] fill between [of=x2y5 and AX, soft clip={domain=0:3}];
\addplot[white] fill between [of=xy3 and UP];
%\addplot[pattern=north east lines, pattern color=red!10] fill between [reverse=true, of=AX and UP, soft clip={domain=0:5}];
\addplot[white] fill between [of=x2 and AX];
\addplot[pattern=north east lines, pattern color=red!10] fill between [of=xy3 and AX];
\addplot[pattern=vertical lines, pattern color=blue!10] fill between [of=x2y5 and AX];
%\addplot[pattern=north east lines, pattern color=blue!10] fill between [of=AX and 2x2y9, soft clip={domain=0:5}];
\addplot[pattern=horizontal lines, pattern color=purple!10] fill between [of=AY and x2];
\intse{AX}{AY}{$\alpha$}{alp};
\intse{AY}{x2y5}{$\beta$}{bet};
\intne{x2y5}{xy3}{$\gamma$}{gam};
\inte{xy3}{x2}{$\delta$}{del};
\intne{x2}{AX}{$\varepsilon$}{eps};
\node at (axis cs:1,1) {$P$};
\addplot[-latex, thick] coordinates
           {(0,0) (2/2.24,1/2.24)} node [pos=.7, anchor=south, label={0:{\small $\nabla z$}}] {};
\end{axis}
\end{tikzpicture}
\caption{Rappresentazione cartesiana del problema di programmazione lineare}
\label{fig:graph3}
\end{figure}

\subsection{Forma standard}
Ricordiamo che un problema di \textbf{programmazione lineare in forma standard} � nella forma (matriciale):
\begin{align*}
\min c'x& \\
Ax& = b \\
x& \geq 0
\end{align*}
Trasformiamo la funzione obiettivo $z$ in $\varphi$ tale che:
\begin{equation*}
\varphi=-z=-2x_1-x_2
\end{equation*}
Quindi introduciamo una \textbf{variabile slack} per ogni disequazione con simbolo $\leq$. Otterremo infine:
\begin{alignat*}{7}
&\min \varphi = \quad && -2x_1 \quad\; && -x_2 \quad\;\; && \qquad\qquad && \qquad\qquad && \qquad\qquad && \\
&\;\st  &&+x_1			&&		 		&&+\pmb{x_3}	&&		 		&&					&&=2\\
&	 	&&+x_1			&&+x_2			&&				&& +\pmb{x_4}	&&					&&=3\\
&	 	&&+x_1			&&+2x_2			&&				&&				&&+\pmb{x_5}		&&=5\\
&		&&\quad\; x_1,	&&\quad\; x_2,	&&\quad\; x_3,	&&\quad\; x_4,	&&\quad\; x_5		&&\geq 0
\end{alignat*}

\subsection{Risoluzione tramite tableau}

\begin{table}[htbp]
\centering
\begin{tabular}{rcccccc}
			&$-\varphi$ & $x_1$ & $x_2$ & $x_3$ & $x_4$ & $x_5$ \\
$\OL{c_j}$ 	& \Sc{0} 	& -2 	& -1 	& 0 	& 0 	& 0 \\
\cline{2-7}
$x_3$ 		& \Sc{2} 	& 1 	& 0 	& 1 	& 0 	& 0 \\
$x_4$ 		& \Sc{3} 	& 1 	& 1		& 0 	& 1 	& 0 \\
$x_5$ 		& \Sc{5} 	& 1 	& 2 	& 0 	& 0 	& 1 \\
\end{tabular}
\caption{Tableau iniziale. Vertice $\alpha(0,0)$}
\label{tab:tab31}
\end{table}

In tabella \vref{tab:tab31} il tableau ricavato dal nostro problema. Si noti che le ultime 3 colonne formano gi� una matrice identit�, perci� le assumeremo come base.
\begin{align*}
\mathcal{B}&=\{A_3,A_4,A_5\}\\
x&=(0,0,2,3,5)
\end{align*}
Ci troviamo nell'origine, che appartiene al politopo $P$ trovato in precedenza e che in particolare � il \textbf{vertice} $\pmb{\alpha}$.
Poich� non tutti i $\OL{c_j}$ sono non negativi, la nostra non � la BFS ottima e dobbiamo muoverci in una BFS migliore. Applicando la \textbf{regola di Dantzig}, facciamo entrare in base la colonna il cui $\OL{c_j}$ � maggiore in valore assoluto. Nel nostro caso, prenderemo in considerazione quindi la colonna $\pmb{A_1}$.
Per scegliere su quale elemento fare \textbf{pivoting}, dobbiamo ottenere il valore di $y_{\ell 1}$ tale che:
$$
\vartheta_{\max}=\min_{i:y_{i1}>0}\frac{y_{i0}}{y_{i1}}=\frac{y_{i0}}{y_{\ell 1}}
$$
Perci�, operando con gli elementi nel tableau:
\begin{align*}
\vartheta_{\max}=\min\left(\frac{2}{1},\frac{3}{1},\frac{5}{1}\right)=\frac{2}{1}=\frac{y_{10}}{\pmb{y_{11}}}
\end{align*}
Faremo pivoting sull'elemento $y_{11}$ (cerchiato in tabella \vref{tab:tab32}). Il nostro scopo � ora far comparire uno 0 nella colonna dell'elemento pivot in tutte le righe tranne quella in cui si trova l'elemento pivot e far comparire un 1 in quest'ultima.
\begin{table}[htbp]
\centering
\begin{tabular}{rrcccccc}
 	  & 			&$-\varphi$ & $x_1$ & $x_2$ & $x_3$ & $x_4$ & $x_5$ \\
$R_0$ & $\OL{c_j}$ 	& \Sc{0} 	& -2 	& -1 	& 0 	& 0 	& 0 \\
\cline{3-8}
$R_1$ & $x_3$ 		& \Sc{2} 	& \C{1}	& 0 	& 1 	& 0 	& 0 \\
$R_2$ & $x_4$ 		& \Sc{3} 	& 1		& 1		& 0 	& 1 	& 0 \\
$R_3$ & $x_5$ 		& \Sc{5} 	& 1 	& 2 	& 0 	& 0 	& 1 \\
\end{tabular}
\caption{Pivoting su $y_{11}$. $A_1$ entra in base e $A_3$ esce.}
\label{tab:tab32}
\end{table}
Poich� $y_{11}=1$ non c'� nulla da fare su $R_1$. Applichiamo le operazioni elementari di riga al nostro tableau come segue:
\begin{align*}
R_0&\leftarrow R_0 + 2R_2; \\
R_2&\leftarrow R_2 - R_1; \\
R_3&\leftarrow R_3 - R_1.
\end{align*}
Il nostro nuovo tableau diventa quindi quello in tabella \vref{tab:tab33}.
\begin{table}[htbp]
\centering
\begin{tabular}{rrcccccc}
 	  & 			&$-\varphi$ & $x_1$ & $x_2$ & $x_3$ & $x_4$ & $x_5$ \\
$R_0$ & $\OL{c_j}$ 	& \Sc{4} 	& 0 	& -1 	& 2 	& 0 	& 0 \\
\cline{3-8}
$R_1$ & $x_1$ 		& \Sc{2} 	& 1 	& 0 	& 1 	& 0 	& 0 \\
$R_2$ & $x_4$ 		& \Sc{1} 	& 0		& 1		& -1 	& 1 	& 0 \\
$R_3$ & $x_5$ 		& \Sc{3} 	& 0 	& 2 	& -1 	& 0 	& 1 \\
\end{tabular}
\caption{Secondo tableau. Vertice $\varepsilon(2,0)$}
\label{tab:tab33}
\end{table}

Ora che $A_1$ � entrato in base e $A_3$ ne � uscito, abbiamo una nuova base $\mathcal{B}$ e una nuova BFS $x$:
\begin{align*}
\mathcal{B}&=\{A_1,A_4,A_5\} \\
x&=(2,0,0,1,3)
\end{align*}
Ci troviamo nel \textbf{vertice} $\pmb{\varepsilon}$, ma questa non � ancora la BFS ottima.
L'unica colonna ad avere un $\OL{c_j}$ negativo � $A_2$ ed � su questa che cercheremo l'elemento di pivot $y_{\ell 2}$:
\begin{align*}
\vartheta_{\max}&=\min_{i:y_{i2}>0}\frac{y_{i0}}{y_{i2}}=\frac{y_{i0}}{y_{\ell 2}} \\
\vartheta_{\max}&=\min\left(\frac{1}{1},\frac{3}{2}\right)=\frac{1}{1}=\frac{y_{20}}{\pmb{y_{22}}}
\end{align*}
Faremo pivoting sull'elemento $y_{22}$ (cerchiato in tabella \vref{tab:tab34}). 
\begin{table}[htbp]
\centering
\begin{tabular}{rrcccccc}
 	  & 			&$-\varphi$ & $x_1$ & $x_2$ & $x_3$ & $x_4$ & $x_5$ \\
$R_0$ & $\OL{c_j}$ 	& \Sc{4} 	& 0 	& -1 	& 2 	& 0 	& 0 \\
\cline{3-8}
$R_1$ & $x_1$ 		& \Sc{2} 	& 1 	& 0 	& 1 	& 0 	& 0 \\
$R_2$ & $x_4$ 		& \Sc{1} 	& 0		& \C{1}	& -1 	& 1 	& 0 \\
$R_3$ & $x_5$ 		& \Sc{3} 	& 0 	& 2 	& -1 	& 0 	& 1 \\
\end{tabular}
\caption{Pivoting su $y_{22}$. $A_2$ entra in base e $A_4$ esce.}
\label{tab:tab34}
\end{table}
Le operazioni di pivoting saranno:
\begin{align*}
R_0&\rightarrow R_0 + R_2 \\
R_3&\rightarrow R_3 - 2R_2
\end{align*}
Il nostro nuovo tableau, quindi, � quello in tabella \vref{tab:tab35}. 
\begin{table}[htbp]
\centering
\begin{tabular}{rrcccccc}
 	  & 			&$-\varphi$ & $x_1$ & $x_2$ & $x_3$ & $x_4$ & $x_5$ \\
$R_0$ & $\OL{c_j}$ 	& \Sc{5} 	& 0 	& 0 	& 1 	& 1 	& 0 \\
\cline{3-8}
$R_1$ & $x_1$ 		& \Sc{2} 	& 1 	& 0 	& 1 	& 0 	& 0 \\
$R_2$ & $x_2$ 		& \Sc{1} 	& 0		& \C{1}	& -1 	& 1 	& 0 \\
$R_3$ & $x_5$ 		& \Sc{1} 	& 0 	& 0 	& 1 	& -2 	& 1 \\
\end{tabular}
\caption{Terzo tableau. Vertice $\delta(2,1)$}
\label{tab:tab35}
\end{table}
Notiamo che tutti i $\OL{c_j}$ sono non negativi, perci� ci troviamo nella \textbf{BFS ottima}. La base $\mathcal{B}$ e la soluzione $x$ sono quindi:
\begin{align*}
\mathcal{B}&={A_2,A_1,A_5} \\
x&=(2,1,0,0,1)
\end{align*}
La soluzione ottima � quella del vertice $\pmb{\delta(2,1)}$. Riassumendo, tutti i valori delle variabili in gioco sono i seguenti:
\begin{align*}
z&=-\varphi=5 \\
x_1&=2 \\
x_2&=1
\end{align*}
\subsection{Conclusione}
La soluzione ottima consiste nel produrre $2$T di composto A e $1$T di composto B ottenendo un \textbf{profitto} pari a 5 volte il profitto di $1$T di composto B.


\chapter{31/03/2014}

I problemi saranno posti in maniera leggermente diversa, cio� quella fornita sul pdf reperibile sul sito del docente al seguente link (se il testo � effettivamente disponibile, s'intende): \url{http://www.or.deis.unibo.it/staff_pages/martello/testi_esercizi_ottimizzazione.pdf}.
Inoltre, anche se durante l'esercitazione non � stata trovata la soluzione dei problemi duali, dato che il metodo per individuarli � stato spiegato dal prof. nella lezione subito successiva, ho ritenuto opportuno e interessante cercarle io stesso e inserirle in questo eserciziario. A maggior ragione, le soluzioni dei duali \textbf{potrebbero essere errate}, per cui chiedo ad ognuno di provare a rivederle e comunicarmi gli eventuali errori trovati.
Inoltre, ho deciso - in maniera del tutto personale e arbitraria - di preporre la rappresentazione grafica alla risoluzione con tableau negli esercizi di ottimizzazione. L'unico motivo � che mi piace avere un'idea un po' pi� concreta di quello che sta succedendo sul piano geometrico.

\section{Esercizio 1}
Un'azienda chimica produce due tipi di composto, A e B, che danno lo stesso profitto, utilizzando una sostanza base della quale sono disponibili 8 quintali. Ogni tonnellata di composto (indipendentemente dal tipo) contiene un quintale di sostanza base. Il numero di tonnellate di composto A prodotto deve superare di almeno una unit� il numero di tonnellate di composto B prodotto. Per problemi di stoccaggio non si possono produrre pi� di 6 tonnellate di composto A. Si associ la variabile $x_1$ al composto A e la variabile $x_2$ al composto B.
\begin{enumerate}
\item Definire il modello LP che determina la funzione di massimo profitto.
\item Porre il modello in forma standard e risolverlo con il metodo delle due fasi e la regola di Bland, introducendo il minimo numero di variabili artificiali. Dire esplicitamente qual � la soluzione trovata.
\item Disegnare con cura la regione ammissibile.
\item Costruire il duale del modello definito al punto 2 e ricavarne le soluzioni ottime.
\item Imporre il vincolo di interezza sulle variabili (supporre che non si possano produrre frazioni di tonnellate) e risolvere il problema con il metodo branch-and-bound. [\textit{Questo punto non sar� analizzato perch� in data di stesura del documento (04/04/2014) l'argomento non � ancora stato trattato dal prof}]
\end{enumerate}

\subsection{Modellizzazione}

Si indichi con:
\begin{itemize}
\item $x_1$ il numero di tonnellate di composto A;
\item $x_2$ il numero di tonnellate di composto B.
\end{itemize}
Lo scopo del nostro problema � di massimizzare i profitti ottenuti dalla produzione. Anche se non siamo a conoscenza degli esatti profitti dati da ogni prodotto, sappiamo che entrambi i composti portano allo stesso profitto. Possiamo quindi esprimere cos� la funzione di profitto:
$$
\max z = x_1 + x_2
$$

Modelliamo ora i vincoli espressi dal problema.
Il modello matematico pu� essere quindi cos� riassunto (sono state apportate semplificazione algebriche):
\begin{align*}
\max z	&= 2x_1+x_2 \\
\st\;\;	& x_1+x_2 \leq 8\\
		& x_1 \geq x_2 + 1\\
		& x_1 \leq 6 \\
	  	& x_1,x_2 \geq 0
\end{align*}

\subsection{Problema in forma grafica}

In figura \vref{fig:graph4} � rappresentato graficamente il problema presentato. In giallo � rappresentato il politopo $P$ e sono stati chiamati $\alpha,\beta,\gamma,\delta$ i suoi quattro vertici, i quali sappiamo corrispondere ognuno ad una BFS.
Il gradiente della funzione obiettivo vale
\begin{equation*}
\nabla(z)=\left(\frac{\partial z}{\partial x},\frac{\partial z}{\partial y}\right) = \left(1,1\right) \\
\end{equation*}
Il politopo $P$ �, ovviamente, limitato nella direzione del gradiente (si fa notare che finora $P$ � sempre limitato in ogni direzione, quindi qualsiasi direzione avesse il gradiente non ci sarebbero problemi).

\begin{figure}[htbp]
\centering
\begin{tikzpicture}
\begin{axis}
[axis lines=middle, axis equal, enlargelimits, xlabel=$x_1$, ylabel=$x_2$,
 every axis x label/.style={
    at={(ticklabel* cs:1.01)},
    anchor=west,
 },
 every axis y label/.style={
    at={(ticklabel* cs:1.01)},
    anchor=south,
 },]
    \path[name path=AX] 
        (axis cs:\pgfkeysvalueof{/pgfplots/xmin},0)--
        (axis cs:\pgfkeysvalueof{/pgfplots/xmax},0);
    \path[name path=AY] 
        (axis cs:0,\pgfkeysvalueof{/pgfplots/ymin})--
        (axis cs:0,\pgfkeysvalueof{/pgfplots/ymax});
    \path[name path=UP]
    	(axis cs:\pgfkeysvalueof{/pgfplots/xmin},\pgfkeysvalueof{/pgfplots/ymax})--
    	(axis cs:\pgfkeysvalueof{/pgfplots/xmax},\pgfkeysvalueof{/pgfplots/ymax});
\addplot
[domain=0:8, samples=10, thick, blue, name path=xy8]
{-x+8} node [pos=0.2,pin={75:{\color{blue}$x_1+x_2=8$}}, inner sep=0pt] {};
\addplot
[domain=0:8, samples=10, thick, red, name path=xy1]
{x-1} node [pos=0.8, pin={-85:{\color{red}$x_1=x_2+1$}}, inner sep=0pt] {};
\addplot
[domain=0:8, samples = 10, thick, purple, name path=x6]
(6,x) node [pos=0.3, anchor=north, pin={0:{\color{purple}$x_1=6$}}, inner sep= 0pt] {};
\addplot[thick, fill=yellow, fill opacity=0.5] fill between [of=xy1 and AX, soft clip={domain=1:6}];
\addplot[white] fill between [of=xy8 and UP];
%\addplot[pattern=north east lines, pattern color=red!10] fill between [reverse=true, of=AX and UP, soft clip={domain=0:5}];
\addplot[white] fill between [of=x6 and AX];
\addplot[pattern=north east lines, pattern color=blue!10] fill between [of=xy8 and AX];
\addplot[pattern=north west lines, pattern color=red!10] fill between [of=xy1 and AX];
%\addplot[pattern=north east lines, pattern color=blue!10] fill between [of=AX and 2x2y9, soft clip={domain=0:5}];
\addplot[pattern=horizontal lines, pattern color=purple!10] fill between [of=AY and x6];
\ints{AX}{xy1}{$\alpha$}{alp};
\ints{xy1}{xy8}{$\beta$}{bet};
\intw{xy8}{x6}{$\gamma$}{gam};
\intnw{x6}{AX}{$\delta$}{del};
\node at (axis cs:4.5,1.5) {$P$};
\addplot[-latex, thick] coordinates
           {(0,0) (1/1.414,1/1.414)} node [pos=.3, anchor=south, label={45:{\small $\nabla z$}}] {};
\end{axis}
\end{tikzpicture}
\caption{Rappresentazione cartesiana del problema di programmazione lineare}
\label{fig:graph4}
\end{figure}

Possiamo osservare che anche solo dal grafico � facilmente intuibile dove si trover� la soluzione ottima. Il gradiente $\nabla z$ � \textbf{perpendicolare} allo spigolo $\OL{\beta \gamma}$, da ci� potremmo dedurre che non esiste una soluzione ottima, ma che ve ne sono infinite e tutte posizionate su questo spigolo. Riprenderemo questa considerazione in seguito, dopo aver risolto il problema con il metodo del simplesso.

\subsection{Forma standard}
Ricordiamo che un problema di \textbf{programmazione lineare in forma standard} � nella forma (matriciale):
\begin{align*}
\min c'x& \\
Ax& = b \\
x& \geq 0
\end{align*}
Trasformiamo la funzione obiettivo $z$ in $\varphi$ tale che:
\begin{equation*}
\varphi=-z=-x_1-x_2
\end{equation*}
Quindi introduciamo una \textbf{variabile slack} per ogni disequazione con simbolo $\leq$ e una \textbf{variabile surplus} per ogni disequazione con simbolo $\geq$. Otterremo infine:
\begin{alignat*}{7}
&\min \varphi = \quad && -x_1 \quad\; && -x_2 \quad\;\; && \qquad\qquad && \qquad\qquad && \qquad\qquad && \\
&\;\st  &&+x_1			&&+x_2	 		&&+\pmb{x_3}	&&		 		&&					&&=8\\
&	 	&&+x_1			&&-x_2			&&				&& -\pmb{x_4}	&&					&&=1\\
&	 	&&+x_1			&&				&&				&&				&&+\pmb{x_5}		&&=6\\
&		&&\quad\; x_1,	&&\quad\; x_2,	&&\quad\; x_3,	&&\quad\; x_4,	&&\quad\; x_5		&&\geq 0
\end{alignat*}

\subsection{Risoluzione tramite tableau}

\begin{table}[htbp]
\centering
\begin{tabular}{rcccccc}
			&$-\varphi$ & $x_1$ & $x_2$ & $x_3$ & $x_4$ & $x_5$ \\
$\OL{c_j}$ 	& \Sc{0} 	& -1 	& -1 	& 0 	& 0 	& 0 \\
\cline{2-7}
$R_1$ 		& \Sc{8} 	& 1 	& 1 	& 1 	& 0 	& 0 \\
$R_2$		& \Sc{1} 	& 1 	& -1	& 0 	& -1 	& 0 \\
$R_3$		& \Sc{6} 	& 1 	& 0 	& 0 	& 0 	& 1 \\
\end{tabular}
\caption{Tableau iniziale.}
\label{tab:tab41}
\end{table}

In tabella \vref{tab:tab41} il tableau ricavato dal nostro problema. A differenza dei precedenti esercizi, la fortuna non � dalla nostra parte e non abbiamo nessuna sottomatrice identit� a disposizione da utilizzare come base ammissibile.
Si potrebbe \textit{erroneamente} pensare che per ottenere una BFS sia sufficiente operare $R_2\leftarrow -1\cdot R_2$. Ma si fa subito notare che cos� facendo otterremo come base:
\begin{align*}
\mathcal{B}&=\{A_3,A_4,A_5\}\\
x&=(0,0,8,1,6)
\end{align*}
Questa \textbf{non � una BFS} in quanto ricade \textit{all'esterno} del politopo $P$. Per ottenere una BFS di partenza, quindi, ricorriamo alla \textbf{fase 1 del metodo del simplesso}.

\subsubsection{Fase 1 - aggiunta di variabili artificiali}

Per ottenere una BFS aggiungiamo un numero $n'\leq m$ di variabili artificiali tali da riuscire ad ottenere una BFS nel nuovo problema con $m$ vincoli e $n+n'$ variabili. Ipoteticamente, potremmo aggiungere sempre $n'=m$ variabili artificiali tali da formare gi� loro una sottomatrice identit� nel tableau, ma tale metodo risulterebbe molto sconveniente nel caso in cui i vincoli e le variabili fossero centinaia o migliaia. Inoltre, ma non meno importante, la traccia dell'esercizio richiede esplicitamente di \textbf{introdurre il minore numero di variabili artificiali}.

Per ridurre al minimo le variabili artificiali $x_i^a,\quad i=1,\cdots,n'$ � sufficiente aggiungerne una per ogni colonna della matrice identit� mancante nel tableau originale. Nel nostro caso manca solo la seconda colonna e sar� quella che introdurremo con l'\textit{unica} variabile artificiale $x^a$, trasformando il secondo vincolo in:
$$
x_1 - x_2 - x_4 + x^a = 1
$$
Il nostro scopo, dopo l'introduzione di $x^a$, sar� quello di \textbf{eliminarla} dalla base. Per far ci� bisogna fare in modo che questa valga zero e quindi introduciamo, a tale scopo, una nuova funzione obiettivo da minimizzare $\psi$ tale che:
$$
\psi = \sum_{i=1}^{n'}x_i^a = x^a
$$
Scriviamo il nuovo tableau in tabella \vref{tab:tab42} e applichiamo il simplesso per ottimizzare la nostra funzione $\psi$.
\begin{table}[htbp]
\centering
\begin{tabular}{rrccccccc}
 	  & 			&$-\psi$	& $x_1$ & $x_2$ & $x_3$ & $x_4$ & $x_5$	& $x^a$\\
$R_0$ & $\OL{c_j}$ 	& \Sc{0} 	& 0 	& 0 	& 0 	& 0 	& \Sc{0}& 1\\
\cline{3-9}
$R_1$ & $x_3$ 		& \Sc{8} 	& 1		& 1 	& 1 	& 0 	& \Sc{0}& 0 \\
$R_2$ & $x^a$ 		& \Sc{1} 	& 1		& -1	& 0 	& -1 	& \Sc{0}& 1 \\
$R_3$ & $x_5$ 		& \Sc{6} 	& 1 	& 0 	& 0 	& 0 	& \Sc{1}& 0 \\
\end{tabular}
\caption{Nuovo tableau con la variabile artificiale $x^a$.}
\label{tab:tab42}
\end{table}
Abbiamo una sottomatrice identit� formata dalla base:
$$
\mathcal{B}=\{A_3,A_6,A_5\}
$$
Per avere a avere a disposizione i valori delle coordinate della BFS del nuovo problema, � necessario che:
$$
y_{ij}=0 \quad \forall i,j:A_j\in\mathcal{B},i\neq j
$$
Condizione vera per ogni valore tranne $y_{06}$ che provvediamo ad annullare tramite l'operazione elementare di riga:
$$
R_0\leftarrow R_0 - R_2
$$
Nel nuovo tableau in figura \vref{tab:tab43} faremo pivoting sull'unica colonna con $\OL{c_j}<0$, cio� su $A_1$.
Per scegliere su quale elemento fare \textbf{pivoting}, dobbiamo ottenere il valore di $y_{\ell 1}$ tale che:
$$
\vartheta_{\max}=\min_{i:y_{i1}>0}\frac{y_{i0}}{y_{i1}}=\frac{y_{i0}}{y_{\ell 1}}
$$
Perci�, operando con gli elementi nel tableau:
$$
\vartheta_{\max}=\min\left(\frac{8}{1},\frac{1}{1},\frac{6}{1}\right)=\frac{1}{1}=\frac{y_{20}}{\pmb{y_{21}}}
$$
Faremo pivoting sull'elemento $y_{21}$ (cerchiato in tabella). Il nostro scopo � ora far comparire uno 0 nella colonna dell'elemento pivot in tutte le righe tranne quella in cui si trova l'elemento pivot e far comparire un 1 in quest'ultima.
\begin{table}[htbp]
\centering
\begin{tabular}{rrccccccc}
 	  & 			&$-\psi$	& $x_1$ & $x_2$ & $x_3$ & $x_4$ & $x_5$	& $x^a$\\
$R_0$ & $\OL{c_j}$ 	& \Sc{-1} 	& -1 	& 1 	& 0 	& 1 	& \Sc{0}& 0\\
\cline{3-9}
$R_1$ & $x_3$ 		& \Sc{8} 	& 1		& 1 	& 1 	& 0 	& \Sc{0}& 0 \\
$R_2$ & $x^a$ 		& \Sc{1} 	& \C{1} & -1	& 0 	& -1 	& \Sc{0}& 1 \\
$R_3$ & $x_5$ 		& \Sc{6} 	& 1 	& 0 	& 0 	& 0 	& \Sc{1}& 0 \\
\end{tabular}
\caption{Pivoting su $y_{21}$. $A_1$ entra in base e $A_6$ esce.}
\label{tab:tab43}
\end{table}
Poich� $y_{21}=1$ non c'� nulla da fare su $R_2$. Applichiamo le operazioni elementari di riga al nostro tableau come segue:
\begin{align*}
R_0&\leftarrow R_0 + R_2; \\
R_1&\leftarrow R_1 - R_2; \\
R_3&\leftarrow R_3 - R_2.
\end{align*}
Il nostro nuovo tableau diventa quindi quello in tabella \vref{tab:tab44}.
\begin{table}[htbp]
\centering
\begin{tabular}{rrccccccc}
 	  & 			&$-\psi$	& $x_1$ & $x_2$ & $x_3$ & $x_4$ & $x_5$	& $x^a$\\
$R_0$ & $\OL{c_j}$ 	& \Sc{0} 	& 0 	& 0 	& 0 	& 0 	& \Sc{0}& 1\\
\cline{3-9}
$R_1$ & $x_3$ 		& \Sc{7} 	& 0		& 2 	& 1 	& 1 	& \Sc{0}& -1 \\
$R_2$ & $x_1$ 		& \Sc{1} 	& 1		& -1	& 0 	& -1 	& \Sc{0}& 1 \\
$R_3$ & $x_5$ 		& \Sc{5} 	& 0 	& 1 	& 0 	& 1 	& \Sc{1}& -1 \\
\end{tabular}
\caption{Secondo tableau. Vertice $\alpha(1,0)$}
\label{tab:tab44}
\end{table}
Siamo giunti alla soluzione ottima, essendo $\OL{c_j}>0 \quad\forall j$. Inoltre la variabile artificiale $x^a$ non � pi� in base. La nuova base e la nuova soluzione sono:
\begin{align*}
\mathcal{B}&=\{A_3,A_1,A_5\} \\
x&=(1,0,7,0,5,0)
\end{align*}
Siamo nel vertice $\alpha(1,0)$ e quindi in una BFS da cui possiamo partire per la \textbf{fase 2} del metodo del simplesso.
\subsubsection{Fase 2 - Simplesso}
Per questa fase useremo come tableau di partenza quello in tabella \vref{tab:tab44} sostituendo la funzione obiettivo fittizia $\psi$ utilizzata in precedenza con la nostra vera funzione obiettivo $\varphi$. Manterremo la variabile artificiale (che si fa notare non cambia in alcun modo il nostro problema in quanto non faremo mai entrare in base) perch�, come vedremo poi, il suo costo relativo finale sar� utile ai fini della soluzione del problema duale.
Il tableau cos� ottenuto � quello in tabella \vref{tab:tab45}
\begin{table}[htbp]
\centering
\begin{tabular}{rrccccccc}
 	  & 			&$-\varphi$	& $x_1$ & $x_2$ & $x_3$ & $x_4$ & $x_5$	& $x^a$\\
$R_0$ & $\OL{c_j}$ 	& \Sc{0} 	& -1 	& -1 	& 0 	& 0 	& \Sc{0}& 0\\
\cline{3-9}
$R_1$ & $x_3$ 		& \Sc{7} 	& 0		& 2 	& 1 	& 1 	& \Sc{0}& -1 \\
$R_2$ & $x_1$ 		& \Sc{1} 	& 1		& -1	& 0 	& -1 	& \Sc{0}& 1 \\
$R_3$ & $x_5$ 		& \Sc{5} 	& 0 	& 1 	& 0 	& 1 	& \Sc{1}& -1 \\
\end{tabular}
\caption{Secondo tableau. Vertice $\alpha(1,0)$ e funzione obiettivo $\varphi$.}
\label{tab:tab45}
\end{table}
Per applicare il simplesso, dobbiamo fare in modo che:
$$
y_{ij}=0 \quad\forall i,j:j\in\mathcal{B}, i\neq j
$$
L'elemento $y_{01}$ � l'unico a non essere nullo. Ovviamo al problema con l'operazione di riga:
$$
R_0\leftarrow R_0 + R_1
$$
Otteniamo quindi il tableau in tabella \vref{tab:tab46}.
\begin{table}[htbp]
\centering
\begin{tabular}{rrccccccc}
 	  & 			&$-\varphi$	& $x_1$ & $x_2$ & $x_3$ & $x_4$ & $x_5$	& $x^a$\\
$R_0$ & $\OL{c_j}$ 	& \Sc{1} 	& 0 	& -2 	& 0 	& -1 	& \Sc{0}& 1\\
\cline{3-9}
$R_1$ & $x_3$ 		& \Sc{7} 	& 0		& 2 	& 1 	& 1 	& \Sc{0}& -1 \\
$R_2$ & $x_1$ 		& \Sc{1} 	& 1		& -1	& 0 	& -1 	& \Sc{0}& 1 \\
$R_3$ & $x_5$ 		& \Sc{5} 	& 0 	& 1 	& 0 	& 1 	& \Sc{1}& -1 \\
\end{tabular}
\caption{Secondo tableau. Vertice $\alpha(1,0)$}
\label{tab:tab46}
\end{table}
Per fare pivoting sceglieremo la colonna $A_2$ in base alla regola di Bland (avremmo scelto la stessa colonna anche con la regola di Dantzig). Cerchiamo quindi l'elemento pivot $y_{\ell 2}$.
\begin{align*}
\vartheta_{\max}&=\min_{i:y_{i2}>0}\frac{y_{i0}}{y_{i2}}=\frac{y_{i0}}{y_{\ell 2}} \\
\vartheta_{\max}&=\min\left(\frac{7}{2},\frac{5}{1}\right)=\frac{7}{2}=\frac{y_{10}}{\pmb{y_{12}}}
\end{align*}
Faremo pivoting sull'elemento $y_{12}$ (cerchiato in tabella \vref{tab:tab47}). 
\begin{table}[htbp]
\centering
\begin{tabular}{rrccccccc}
 	  & 			&$-\varphi$	& $x_1$ & $x_2$ & $x_3$ & $x_4$ & $x_5$	& $x^a$\\
$R_0$ & $\OL{c_j}$ 	& \Sc{1} 	& 0 	& -2 	& 0 	& -1 	& \Sc{0}& 1\\
\cline{3-9}
$R_1$ & $x_3$ 		& \Sc{7} 	& 0		& \C{2}	& 1 	& 1 	& \Sc{0}& -1 \\
$R_2$ & $x_1$ 		& \Sc{1} 	& 1		& -1	& 0 	& -1 	& \Sc{0}& 1 \\
$R_3$ & $x_5$ 		& \Sc{5} 	& 0 	& 1 	& 0 	& 1 	& \Sc{1}& -1 \\
\end{tabular}
\caption{Terzo tableau. Vertice $\alpha(1,0)$}
\label{tab:tab47}
\end{table}
Le operazioni elementari di riga, \textbf{in ordine}, sono:
\begin{align*}
R_0&\leftarrow R_0 + R_1 \\
R_1&\leftarrow \frac{R_1}{2} \\
R_2&\leftarrow R_2 + R_1 \\
R_3&\leftarrow R_3 - R_1
\end{align*}
Otterremo il tableau in tabella \vref{tab:tab48}.
\begin{table}[htbp]
\centering
{
	\newcommand{\sm}{$\frac{7}{2}$}
	\newcommand{\nm}{$\frac{9}{2}$}
	\newcommand{\um}{$\frac{1}{2}$}
	\newcommand{\tm}{$\frac{3}{2}$}
\begin{tabular}{rrccccccc}
 	  & 			&$-\varphi$	& $x_1$ & $x_2$ & $x_3$ & $x_4$ & $x_5$	& $x^a$\\
$R_0$ & $\OL{c_j}$ 	& \Sc{8} 	& 0 	& 0 	& 1 	& 0 	& \Sc{0}& 0\\
\cline{3-9}
$R_1$ & $x_2$ 		& \Sc{\sm}	& 0		& 1		& \um 	& \um 	& \Sc{0}& -\um \\
$R_2$ & $x_1$ 		& \Sc{\nm} 	& 1		& 0		& \um	& -\um 	& \Sc{0}& \um \\
$R_3$ & $x_5$ 		& \Sc{-\tm}	& 0 	& 0 	& -\um 	& \um 	& \Sc{1}& -\um \\
\end{tabular}
}
\caption{Quarto tableau. Vertice $\beta(\frac{9}{2},\frac{7}{2})$. $A_2$ entra in base al posto di $A_3$, che esce.}
\label{tab:tab48}
\end{table}

Siamo giunti alla soluzione ottima, essendo $\OL{c_j}>0 \quad\forall j$. La nuova base e la nuova soluzione sono:
\begin{align*}
\mathcal{B}&=\{A_2,A_1,A_5\} \\
x&=(1,0,7,0,5,0)
\end{align*}
Siamo nel vertice $\beta(\frac{9}{2},\frac{7}{2})$ ed appartiene, come previsto durante l'analisi geometrica, allo spigolo $\OL{\beta\gamma}$. Il valore della soluzione ottima � $\varphi=-8$, proviamo ora a calcolare il valore della soluzione con il vertice $\gamma(6,2)$:
$$
\varphi(6,2)=-6-2=-8
$$
Anche il vertice $\gamma$ � una soluzione ottima del nostro problema. Da ci� possiamo desumere che l'intero spigolo $\OL{\beta\gamma}$ � composto da infinite soluzioni ottime. D'altronde, spostandoci lungo $\OL{\beta\gamma}$ avanzeremo in direzione perpendicolare al gradiente della funzione obiettivo e il valore della soluzione non pu� cambiare.
Si fa notare infine che la nostra funzione obiettivo iniziale �:
$$
z=-\varphi=8
$$

\subsection{Soluzione del problema primale}
La soluzione ottima consiste nel produrre $4.5$T di composto A e $3.5$T di composto B ottenendo un profitto pari a 8 volte quello di $1$T di composto A (o di composto B, equivalentemente).

\subsection{Costruzione del problema duale}
Riportiamo, per comodit�, il problema primale espresso in forma standard.
\begin{alignat*}{7}
&\min \varphi = \quad && -x_1 \quad\; && -x_2 \quad\;\; && \qquad\qquad && \qquad\qquad && \qquad\qquad && \\
&\;\st  &&+x_1			&&+x_2	 		&&+\pmb{x_3}	&&		 		&&					&&=8\\
&	 	&&+x_1			&&-x_2			&&				&& -\pmb{x_4}	&&					&&=1\\
&	 	&&+x_1			&&				&&				&&				&&+\pmb{x_5}		&&=6\\
&		&&\quad\; x_1,	&&\quad\; x_2,	&&\quad\; x_3,	&&\quad\; x_4,	&&\quad\; x_5		&&\geq 0
\end{alignat*}
Ricordiamo che le regole base per la creazione del problema duale (considereremo solo quelle in grassetto nel caso di problemi primali in forma standard):
\begin{itemize}
\item \textbf{Ad ogni vincolo corrisponde una variabile duale};
\item \textbf{Ad ogni vincolo di uguaglianza, la rispettiva variabile duale � una variabile libera};
\item Ad ogni vincolo di non minoranza corrisponde una variabile duale non negativa;
\item \textbf{Ad ogni variabile non negativa nel primale corrisponde un vincolo con relazione di non maggioranza nel duale};
\item Ad ogni variabile libera nel primale corrisponde un vincolo di uguaglianza nel duale.
\end{itemize}
In dettaglio, ridefiniamo in questo modo il generico problema primale in forma standard:
\begin{align*}
\min c'x& \\
Ax& = b \\
x& \geq 0
\end{align*}
Sia $\pi$ il vettore delle variabili duali, il problema duale � il seguente:
\begin{align*}
\max \pi'b& \\
\pi'A& \leq c' \\
\pi'&\gtreqless 0
\end{align*}
Ove, i vettori $x,\pi,b,c$ e la matrice $A$ sono:
\begin{align*}
x'&=
\begin{bmatrix}
x_1 & x_2 & x_3 & x_4 & x_5
\end{bmatrix} \\
\pi'&=
\begin{bmatrix}
\pi_1 & \pi_2 & \pi_3
\end{bmatrix} \\
b'&=
\begin{bmatrix}
8 & 1 & 6
\end{bmatrix} \\
c'&=
\begin{bmatrix}
-1 & -1 & 0 & 0 & 0
\end{bmatrix} \\
A&=
\begin{bmatrix}
1  & 1  & 1  & 0  & 0 \\
1  & -1 & 0  & -1 & 0 \\
1  & 0  & 0  & 0  & 1 \\
\end{bmatrix}
\end{align*}
Da ci�, il corrispondente problema duale con la sua funzione obiettivo $\xi$:
\begin{alignat*}{7}
&\max \xi = \quad && +8\pi_1 \quad\; && +\pi_2 \quad\;\; && +6\pi_3 \quad\;\; && \\
&\;\st  &&+\pi_1		&&+\pi_2 		&&+\pi_3		&& \leq -1\\
&	 	&&+\pi_1		&&-\pi_2 		&&				&& \leq -1\\
&	 	&&+\pi_1		&&				&&				&& \leq 0 \\
&	 	&&				&&-\pi_2		&&				&& \leq 0 \\
&	 	&&				&&				&&+\pi_3		&& \leq 0 \\
&		&&\quad\;\pi_1,	&&\quad\;\pi_2,	&&\quad\;\pi_3,	&& \gtreqless 0
\end{alignat*}
Per trovare la soluzione del problema duale non � necessario trasformarlo in forma standard e applicare il metodo del simplesso. Il tableau del problema primale sul quale abbiamo applicato il metodo del simplesso contiene tutte le informazioni per avere la soluzione del problema duale.
\subsubsection{Richiami (sempre molto blandi) di teoria}
Per ottenere dal tableau del problema primale la soluzione del problema duale, � sufficiente ricordare che il problema duale � ottenuto a partire dal \textbf{criterio di ottimalit�}.
Per questo motivo, il costo relativo nel tableau finale - corrispondente alla soluzione ottima - � cos� esprimibile:
$$
\OL{c_j}=c_j-z_j=c_j-\pi'A_j \quad \forall j
$$
Se consideriamo le colonne $A_j$ corrispondenti alla base iniziale $\mathcal{B}_0$ di partenza del primo tableau - ricordando che � una matrice identit� - possiamo ottenere:
$$
\OL{c_j}=c_j-\pi_j \quad \forall j:A_j\in\mathcal{B}_0
$$
Applicando un semplice passaggio algebrico:
$$
\pi_j=c_j-\OL{c_j}
$$
Ove $c_j$ � il costo iniziale nel primo tableau e $\OL{c_j}$ il costo relativo nel tableau finale.
Nel caso in cui abbiamo fatto uso di variabili artificiali e della fase 1 del metodo del simplesso, allora per tale variabile - il cui costo � $c_j=0$ - vale:
$$
\pi_j=-\OL{c_j}
$$
\textit{� importante ricordare che bisogna utilizzare il primo tableau con le variabili artificiali ma con il vettore dei costi originario in cui le variabili artificiali hanno costo nullo.}
\subsection{Soluzione del problema duale}
Riportiamo nuovamente i tableau iniziale e finale rispettivamente nelle tabella \vref{tab:tab49} e \vref{tab:tab410}.
\begin{table}[htbp]
\centering
\begin{tabular}{rrccccccc}
 	  & 			&$-\varphi$	& $x_1$ & $x_2$ & $x_3$ & $x_4$ & $x_5$	& $x^a$\\
$R_0$ & $\OL{c_j}$ 	& \Sc{0} 	& -1 	& -1 	& 0 	& 0 	& \Sc{0}& 0\\
\cline{3-9}
$R_1$ & $x_3$ 		& \Sc{8} 	& 1		& 1 	& 1 	& 0 	& \Sc{0}& 0 \\
$R_2$ & $x^a$ 		& \Sc{1} 	& 1		& -1	& 0 	& -1 	& \Sc{0}& 1 \\
$R_3$ & $x_5$ 		& \Sc{6} 	& 1 	& 0 	& 0 	& 0 	& \Sc{1}& 0 \\
\end{tabular}
\caption{Tableau iniziale.}
\label{tab:tab49}
\end{table}
\begin{table}[htbp]
\centering
{
	\newcommand{\sm}{$\frac{7}{2}$}
	\newcommand{\nm}{$\frac{9}{2}$}
	\newcommand{\um}{$\frac{1}{2}$}
	\newcommand{\tm}{$\frac{3}{2}$}
\begin{tabular}{rrccccccc}
 	  & 			&$-\varphi$	& $x_1$ & $x_2$ & $x_3$ & $x_4$ & $x_5$	& $x^a$\\
$R_0$ & $\OL{c_j}$ 	& \Sc{8} 	& 0 	& 0 	& 1 	& 0 	& \Sc{0}& 0\\
\cline{3-9}
$R_1$ & $x_2$ 		& \Sc{\sm}	& 0		& 1		& \um 	& \um 	& \Sc{0}& -\um \\
$R_2$ & $x_1$ 		& \Sc{\nm} 	& 1		& 0		& \um	& -\um 	& \Sc{0}& \um \\
$R_3$ & $x_5$ 		& \Sc{-\tm}	& 0 	& 0 	& -\um 	& \um 	& \Sc{1}& -\um \\
\end{tabular}
}
\caption{Tableau finale.}
\label{tab:tab410}
\end{table}
La base iniziale � $\mathcal{B}_1={A_3,A_6,A_5}$. Applicando delle semplici sottrazioni, ricaviamo la soluzione del problema duale:
\begin{align*}
\pi_1&=c_3-\OL{c_3}=-1 \\
\pi_2&=c_6-\OL{c_6}=0 \\
\pi_3&=c_5-\OL{c_5}=0
\end{align*}
Perci�, la soluzione del problema duale � il vettore:
$$
\pi'=
\begin{bmatrix}
-1 & 0 & 0
\end{bmatrix}
$$
Per verificare la correttezza dei calcoli, applichiamo la soluzione alla funzione obiettivo del problema duale:
$$
\xi(-1,0,0)=8(-1)+0+6(0)=-8
$$
Il risultato �, come atteso, lo stesso del problema primale.

\subsubsection{Vincolo di interezza}

\textit{Sezione in fase di allestimento...ci rivediamo appena il prof. spiegher� i metodi per la ILP}.

\section{Esercizio 2}

Sia dato il seguente modello matematico di un problema di LP:
\begin{align*}
\min z	&= 2x_1+x_2 \\
\st\;\;	& x_1+x_2 \leq 2\\
		& 2x_1+3x_2\geq 6\\
	  	& x_1,x_2 \geq 0
\end{align*}
\begin{itemize}
\item Si rappresenti accuratamente il problema in forma grafica;
\item Si ricavi la forma standard;
\item Si risolva il problema tramite il metodo del simplesso a due fasi utilizzando il minor numero di variabili artificiali.
\end{itemize}

\subsection{Problema in forma grafica}

In figura \vref{fig:graph5} � rappresentato graficamente il problema presentato. In giallo � rappresentato il politopo $P$ e sono stati chiamati $\alpha,\beta,\gamma$ i suoi tre vertici, i quali sappiamo corrispondere ognuno ad una BFS.
Il gradiente della funzione obiettivo vale
\begin{equation*}
\nabla(z)=\left(\frac{\partial z}{\partial x},\frac{\partial z}{\partial y}\right) = \left(2,1\right) \\
\end{equation*}
Poich� questa volta il problema � rappresentato sotto forma di minimo, saremo interessati alla limitazione del politopo nella direzione \textbf{opposta} al gradiente. A tal fine utilizziamo una funzione obiettivo ausiliaria $\varphi$ tale che:
\begin{equation*}
\varphi = -z \Rightarrow \nabla\varphi = -\nabla z = (-2,-1)
\end{equation*}

\begin{figure}[htbp]
\centering
\begin{tikzpicture}
\begin{axis}
[axis lines=middle, axis equal, enlargelimits, xlabel=$x_1$, ylabel=$x_2$,
 every axis x label/.style={
    at={(ticklabel* cs:1.01)},
    anchor=west,
 },
 every axis y label/.style={
    at={(ticklabel* cs:1.01)},
    anchor=south,
 },xtick={1,2,3}]
    \path[name path=AX] 
        (axis cs:\pgfkeysvalueof{/pgfplots/xmin},0)--
        (axis cs:\pgfkeysvalueof{/pgfplots/xmax},0);
    \path[name path=AY] 
        (axis cs:0,\pgfkeysvalueof{/pgfplots/ymin})--
        (axis cs:0,\pgfkeysvalueof{/pgfplots/ymax});
    \path[name path=UP]
    	(axis cs:\pgfkeysvalueof{/pgfplots/xmin},\pgfkeysvalueof{/pgfplots/ymax})--
    	(axis cs:\pgfkeysvalueof{/pgfplots/xmax},\pgfkeysvalueof{/pgfplots/ymax});

\foreach \q in {1,...,6} {
	\ifthenelse{\q < 2}{\newcommand{\x}{2}}{\newcommand{\x}{\q}}
	\addplot [domain=(\x/2)-1:(\q/2), samples=10, ultra thin, purple] {-2*x+\q};
	}

\addplot
[domain=0:2.01, samples=10, thick, blue, name path=xy2]
{-x+2} node [pos=0.2,pin={75:{\color{blue}$x_1+x_2=2$}}, inner sep=0pt] {};

\addplot
[domain=0:3.01, samples=10, thick, red, name path=2x3y6]
{-(2/3)*x+2} node [pos=0.5, pin={75:{\color{red}$2x_1+3x_2=6$}}, inner sep=0pt] {};
\addplot[thick, fill=yellow, fill opacity=0.5] fill between [of=2x3y6 and AX];
\addplot[white] fill between [of=xy2 and AX];
%\addplot[pattern=north east lines, pattern color=red!10] fill between [reverse=true, of=AX and UP, soft clip={domain=0:5}];
\addplot[pattern=north east lines, pattern color=blue!10] fill between [of=xy2 and UP];
\addplot[pattern=vertical lines, pattern color=red!10] fill between [of=2x3y6 and AX];
\intne{AY}{xy2}{$\alpha$}{alp};
\intne{AX}{2x3y6}{$\beta$}{bet};
\intne{xy2}{AX}{$\gamma$}{gam};
\node at (axis cs:1.8,0.5) {$P$};
\addplot[-latex, thick] coordinates
           {(2/2.236,1/2.236) (0,0)} node [pos=.3, anchor=south, label={45:{\small $\nabla\varphi$}}] {};
\end{axis}
\end{tikzpicture}
\caption{Rappresentazione cartesiana del problema di programmazione lineare}
\label{fig:graph5}
\end{figure}

Osservando il fascio di rette perpendicolari al gradiente si pu� intuire che la nostra soluzione ottima si trover� nel vertice $\alpha$.

\subsection{Forma standard}

Aggiungendo una variabile slack e una variabile surplus, il problema in forma standard si presenta cos�:
\begin{alignat*}{6}
&\min z = \quad && +2x_1 \quad\; && +x_2 \quad\;\; && \qquad\qquad && \qquad\qquad && \\
&\;\st  &&+x_1			&&+x_2	 		&&+\pmb{x_3}	&&		 		&&=2\\
&	 	&&+2x_1			&&+3x_2			&&				&& -\pmb{x_4}	&&=6\\
&		&&\quad\; x_1,	&&\quad\; x_2,	&&\quad\; x_3,	&&\quad\; x_4,	&&\geq 0
\end{alignat*}

\subsection{Risoluzione tramite tableau}

\begin{table}[htbp]
\centering
\begin{tabular}{rccccc}
			&$-z$ & $x_1$ & $x_2$ & $x_3$ & $x_4$ \\
$\OL{c_j}$ 	& \Sc{0} 	& 2 	& 1 	& 0 	& 0 \\
\cline{2-6}
$R_1$ 		& \Sc{2} 	& 1 	& 1 	& 1 	& 0\\
$R_2$		& \Sc{6} 	& 2 	& 3		& 0 	& -1 \\
\end{tabular}
\caption{Tableau iniziale.}
\label{tab:tab411}
\end{table}

In tabella \vref{tab:tab411} il tableau ricavato dal nostro problema. Non abbiamo nessuna sottomatrice identit� a disposizione da utilizzare come base ammissibile quindi ricorriamo alla \textbf{fase 1 del metodo del simplesso} per ottenere una BFS di partenza.

\subsubsection{Fase 1 - aggiunta di variabili artificiali}

Manca solo la seconda colonna della matrice identit� con cui formare la BFS di partenza e la che introdurremo con l'\textit{unica} variabile artificiale $x^a$, trasformando il secondo vincolo in:
$$
2x_1 + 3x_2 - x_4 + x^a = 6
$$
Il nostro scopo, dopo l'introduzione di $x^a$, sar� quello di \textbf{eliminarla} dalla base. Per far ci� bisogna fare in modo che questa valga zero e quindi introduciamo, a tale scopo, una nuova funzione obiettivo da minimizzare $\psi$ tale che:
$$
\psi = \sum_{i=1}^{n'}x_i^a = x^a
$$
Scriviamo il nuovo tableau in tabella \vref{tab:tab412} e applichiamo il simplesso per ottimizzare la nostra funzione $\psi$.
\begin{table}[htbp]
\centering
\begin{tabular}{rrcccccc}
 	  & 			&$-\psi$	& $x_1$ & $x_2$ & $x_3$ & $x_4$	& $x^a$\\
$R_0$ & $\OL{c_j}$ 	& \Sc{0} 	& 0 	& 0 	& 0 	& \Sc{0}& 1\\
\cline{3-8}
$R_1$ & $x_3$ 		& \Sc{2} 	& 1		& 1 	& 1 	& \Sc{0}& 0 \\
$R_2$ & $x^a$ 		& \Sc{6} 	& 2		& 3		& 0 	&\Sc{-1}& 1 \\
\end{tabular}
\caption{Nuovo tableau con la variabile artificiale $x^a$.}
\label{tab:tab42}
\end{table}
Abbiamo una sottomatrice identit� formata dalla base:
$$
\mathcal{B}=\{A_3,A_5\}
$$
Per avere a avere a disposizione i valori delle coordinate della BFS del nuovo problema, � necessario che:
$$
y_{ij}=0 \quad \forall i,j:A_j\in\mathcal{B},i\neq j
$$
Condizione vera per ogni valore tranne $y_{05}$ che provvediamo ad annullare tramite l'operazione elementare di riga:
$$
R_0\leftarrow R_0 - R_2
$$
Nel nuovo tableau in figura \vref{tab:tab413} dobbiamo scegliere su quale colonna fare pivoting. Dato che la traccia non ci specifica nulla sulla regola da utilizzare \footnote{Oppure non ricordo se era stato specificato dal tutor all'inizio dell'esercizio durante la lezione [NdA]} applicheremo la regola di Dantzig e sceglieremo la colonna con il $\OL{c_j}$ pi� negativo, cio� $A_2$.
Per scegliere su quale elemento fare \textbf{pivoting}, dobbiamo ottenere il valore di $y_{\ell 2}$ tale che:
$$
\vartheta_{\max}=\min_{i:y_{i2}>0}\frac{y_{i0}}{y_{i2}}=\frac{y_{i0}}{y_{\ell 2}}
$$
Perci�, operando con gli elementi nel tableau:
$$
\vartheta_{\max}=\min\left(\frac{2}{1},\frac{6}{3}\right)
$$
Abbiamo un pareggio. Applichiamo ora la regola di Bland per risolvere il pareggio, scegliendo tra gli elementi su cui fare pivot quello con l'indice di riga minore:
$$
\vartheta_{\max}=\frac{2}{1}=\frac{y_{10}}{\pmb{y_{12}}}
$$
Faremo pivoting sull'elemento $y_{12}$ (cerchiato in tabella). Il nostro scopo � ora far comparire uno 0 nella colonna dell'elemento pivot in tutte le righe tranne quella in cui si trova l'elemento pivot e far comparire un 1 in quest'ultima.
\begin{table}[htbp]
\centering
\begin{tabular}{rrcccccc}
 	  & 			&$-\psi$	& $x_1$ & $x_2$ & $x_3$ & $x_4$	& $x^a$\\
$R_0$ & $\OL{c_j}$ 	& \Sc{-6} 	& -2 	& -3 	& 0 	& \Sc{1}& 0\\
\cline{3-8}
$R_1$ & $x_2$ 		& \Sc{2} 	& \C{1}	& 1 	& 1 	& \Sc{0}& 0 \\
$R_2$ & $x^a$ 		& \Sc{6} 	& 2		& 3		& 0 	&\Sc{-1}& 1 \\
\end{tabular}
\caption{Pivoting su $y_{12}$. $A_2$ entra in base e $A_3$ esce.}
\label{tab:tab413}
\end{table}
Poich� $y_{12}=1$ non c'� nulla da fare su $R_1$. Applichiamo le operazioni elementari di riga al nostro tableau come segue:
\begin{align*}
R_0&\leftarrow R_0 + 3R_1; \\
R_2&\leftarrow R_2 - 3R_1.
\end{align*}
Il nostro nuovo tableau diventa quindi quello in tabella \vref{tab:tab414}.
\begin{table}[htbp]
\centering
\begin{tabular}{rrcccccc}
 	  & 			&$-\psi$	& $x_1$ & $x_2$ & $x_3$ & $x_4$	& $x^a$\\
$R_0$ & $\OL{c_j}$ 	& \Sc{0} 	& 1 	& 0 	& 3 	& \Sc{1}& 0\\
\cline{3-8}
$R_1$ & $x_2$ 		& \Sc{2} 	& 1		& 1 	& 1 	& \Sc{0}& 0 \\
$R_2$ & $x^a$ 		& \Sc{0} 	& -1	& 0		&\C{-3}	&\Sc{-1}& 1 \\
\end{tabular}
\caption{Secondo tableau. $x^a$ ancora in base.}
\label{tab:tab414}
\end{table}
Siamo giunti alla soluzione ottima, ma non � ancora sufficiente. Possiamo osservare, infatti, che $x_3$ � subentrata in base al posto di $x_2$ e che quindi $x^a$ � ancora in base e noi non lo vogliamo. Se il problema fosse risolvibile dovrebbe esserlo \textit{a prescindere} dalla variabile artificiale, cio� dovremmo essere in grado di trovare una soluzione a questo tableau con $x^a$ fuori base.
Non tutto � ancora perduto. Osserviamo che la base in cui ci troviamo ora � \textbf{degenere}, e possiamo fare entrare al posto di $x^a$ una qualsiasi altra variabile senza creare problemi. Qualsiasi operazione elementare di riga, in tal caso, non apporterebbe modifiche al valore di $-\psi$ e rimarremmo comunque in basi ottime.
Possiamo fare pivot su qualsiasi elemento di $R_2$ purch� non sia nullo (e purch� non sia la stessa variabile artificiale, ovviamente). Questa volta sceglieremo $y_{23}$, consapevoli che sarebbero andati bene anche $y_{21}$ e $y_{24}$.
Applichiamo le operazioni elementari di riga, \textbf{nell'ordine}, per completare l'operazione di pivoting:
\begin{align*}
R_0&\leftarrow R_0 + R_2 \\
R_2&\leftarrow \frac{1}{3}R_2 \\
R_1&\leftarrow R_1 + R_2
\end{align*}
Otteniamo il tableau in tabella \vref{tab:tab415}. � ancora un tableau ottimo (non poteva essere diversamente) e questa volta nessuna fastidiosa variabile artificiale � in base.
\begin{table}[htbp]
\centering
{
	\newcommand{\ut}{$\frac{1}{3}$}
	\newcommand{\dt}{$\frac{2}{3}$}
\begin{tabular}{rrcccccc}
 	  & 			&$-\psi$	& $x_1$ & $x_2$ & $x_3$ & $x_4$	& $x^a$\\
$R_0$ & $\OL{c_j}$ 	& \Sc{0} 	& 0 	& 0 	& 0 	& \Sc{0}& 1\\
\cline{3-8}
$R_1$ & $x_2$ 		& \Sc{2} 	& \dt	& 1 	& 0 	&\Sc{-\ut}& \ut \\
$R_2$ & $x^3$ 		& \Sc{0} 	& -\ut	& 0		& 1 	&\Sc{-\ut}& 0 \\
\end{tabular}\caption{Terzo tableau. $A_3$ entra in base al posto di $A_5$. Vertice $\alpha(0,2)$}
}
\label{tab:tab415}
\end{table}
La nuova base e la nuova soluzione sono:
\begin{align*}
\mathcal{B}&=\{A_2,A_3\} \\
x&=(0,2,0,0,0)
\end{align*}
Siamo nel vertice $\alpha(0,2)$ e quindi in una BFS da cui possiamo partire per la \textbf{fase 2} del metodo del simplesso.
\subsubsection{Fase 2 - Simplesso}
Per questa fase useremo come tableau di partenza quello in tabella \vref{tab:tab415} sostituendo la funzione obiettivo fittizia $\psi$ utilizzata in precedenza con la nostra vera funzione obiettivo $z$. Manterremo la variabile artificiale (che si fa notare non cambia in alcun modo il nostro problema in quanto non faremo mai entrare in base) perch�, come vedremo poi, il suo costo relativo finale sar� utile ai fini della soluzione del problema duale.
Il tableau cos� ottenuto � quello in tabella \vref{tab:tab416}
\begin{table}[htbp]
\centering
{
	\newcommand{\ut}{$\frac{1}{3}$}
	\newcommand{\dt}{$\frac{2}{3}$}
\begin{tabular}{rrcccccc}
 	  & 			&$-z$		& $x_1$ & $x_2$ & $x_3$ & $x_4$	& $x^a$\\
$R_0$ & $\OL{c_j}$ 	& \Sc{0} 	& 2 	& 1 	& 0 	& \Sc{0}& 0\\
\cline{3-8}
$R_1$ & $x_2$ 		& \Sc{2} 	& \dt	& 1 	& 0 	&\Sc{-\ut}& \ut \\
$R_2$ & $x^3$ 		& \Sc{0} 	& -\ut	& 0		& 1 	&\Sc{-\ut}& 0 \\
\end{tabular}\caption{Quarto tableau. Vertice $\alpha(0,2)$}
}
\label{tab:tab416}
\end{table}
Per applicare il simplesso, dobbiamo fare in modo che:
$$
y_{ij}=0 \quad\forall i,j:j\in\mathcal{B}, i\neq j
$$
L'elemento $y_{02}$ � l'unico a non essere nullo. Ovviamo al problema con l'operazione di riga:
$$
R_0\leftarrow R_0 - R_1
$$
Otteniamo quindi il tableau in tabella \vref{tab:tab417}.
\begin{table}[htbp]
\centering
{
	\newcommand{\ut}{$\frac{1}{3}$}
	\newcommand{\dt}{$\frac{2}{3}$}
	\newcommand{\qt}{$\frac{4}{3}$}
\begin{tabular}{rrcccccc}
 	  & 			&$-z$		& $x_1$ & $x_2$ & $x_3$ & $x_4$	& $x^a$\\
$R_0$ & $\OL{c_j}$ 	& \Sc{-2} 	& \qt 	& 0 	& 0 	&\Sc{\ut} & -\ut\\
\cline{3-8}
$R_1$ & $x_2$ 		& \Sc{2} 	& \dt	& 1 	& 0 	&\Sc{-\ut}& \ut \\
$R_2$ & $x^3$ 		& \Sc{0} 	& -\ut	& 0		& 1 	&\Sc{-\ut}& 0 \\
\end{tabular}\caption{Quarto tableau. Vertice $\alpha(0,2)$}
}
\label{tab:tab417}
\end{table}
Il nostro lavoro si conclude qui in quanto $\OL{c_j}>0\quad\forall j>0$ e il vertice $\alpha$ � gi� quello della soluzione ottima, come d'altronde previsto durante la soluzione per via grafica. La base e la soluzione sono quelle gi� espresse in precedenza:
\begin{align*}
\mathcal{B}&=\{A_2,A_3\} \\
x&=(0,2,0,0,0)
\end{align*}

\subsection{Soluzione del problema}
La soluzione del problema �:
$$
z(\alpha)=z(0,2)=2
$$

\subsection{Extra - Costruzione del problema duale}
Anche se nessuno ce l'ha chiesto, proviamo a costruire e risolvere il problema duale a quello dato.
Riportiamo ora il problema primale e, per motivi di sintesi, poniamo anche una notazione pi� breve che ci permetter� di costruire il problema duale:
$$
{
	\renewcommand*\arraystretch{.7}
\begin{array}{r| @{}c@{} |c| @{}c@{} |c| @{}c@{}|c| @{}c@{} |c| c|}
\cline{2-2}\cline{4-4}\cline{6-6}\cline{8-8}\cline{10-10}
	 &\qquad\qquad& &\qquad\qquad& &\qquad\qquad& &\qquad\qquad& & \\
\min		& 2x_1 	&+& x_2 	& &   		& &   		& & \CG{\max}\\
	 		&\CG{=}	& &\CG{=}	& &\CG{=}	& &\CG{=}	& & 	\\
\CG{\pi_1}	& x_1	&+& x_2 	&+& x_3 	& &			&=& 2 	\\
	 		&\CG{+}	& &\CG{+}	& &\CG{+}	& &\CG{+}	& & 	\\
\CG{\pi_2}	& 2x_1 	&+&3x_2 	& &			&-& x_4		&=& 6	\\
	 		&  		& &  		& &	  		& &   		& & 	\\
\cline{2-2}\cline{4-4}\cline{6-6}\cline{8-8}\cline{10-10}
	 		&  		& &  		& &	  		& &   		& & 	\\
	 		& x_1	&,& x_2 	&,& x_3 	&,& x_4 	&\geq& 0\\
	 		&  		& &  		& &	  		& &   		& & 	\\
\cline{2-2}\cline{4-4}\cline{6-6}\cline{8-8}\cline{10-10}
\end{array}
}
$$
Ricordando che \textit{ad ogni variabile non negativa corrisponde un vincolo duale di non maggioranza}, il problema duale, perci�, � il seguente:
\begin{alignat*}{6}
&\max \xi = \quad && +2\pi_1 \quad\; && +6\pi_2 \quad\;\; && \\
&\;\st  &&+\pi_1		&&+2\pi_2 		&& \leq 2 \\
&	 	&&+\pi_1		&&+3\pi_2 		&& \leq 1\\
&	 	&&+\pi_1		&&				&& \leq 0 \\
&	 	&&				&&-\pi_2		&& \leq 0 \\
&		&&\quad\;\pi_1,	&&\quad\;\pi_2,	&& \gtreqless 0
\end{alignat*}

\subsection{Extra - Soluzione del problema duale}
Riportiamo i tableau iniziale e finale con i quali abbiamo applicato l'algoritmo del simplesso.
\begin{table}[htbp]
\centering
\begin{tabular}{rrcccccc}
 	  & 			&$-\psi$	& $x_1$ & $x_2$ & $x_3$ & $x_4$	& $x^a$\\
$R_0$ & $\OL{c_j}$ 	& \Sc{0} 	& 2 	& 1 	& 0 	& \Sc{0}& 0\\
\cline{3-8}
$R_1$ & $x_3$ 		& \Sc{2} 	& 1		& 1 	& 1 	& \Sc{0}& 0 \\
$R_2$ & $x^a$ 		& \Sc{6} 	& 2		& 3		& 0 	&\Sc{-1}& 1 \\
\end{tabular}
\caption{Tableau iniziale con la \textit{funzione obiettivo originaria}.}
\label{tab:tab418}
\end{table}
\begin{table}[htbp]
\centering
{
	\newcommand{\ut}{$\frac{1}{3}$}
	\newcommand{\dt}{$\frac{2}{3}$}
	\newcommand{\qt}{$\frac{4}{3}$}
\begin{tabular}{rrcccccc}
 	  & 			&$-z$		& $x_1$ & $x_2$ & $x_3$ & $x_4$	& $x^a$\\
$R_0$ & $\OL{c_j}$ 	& \Sc{-2} 	& \qt 	& 0 	& 0 	&\Sc{\ut} & -\ut\\
\cline{3-8}
$R_1$ & $x_2$ 		& \Sc{2} 	& \dt	& 1 	& 0 	&\Sc{-\ut}& \ut \\
$R_2$ & $x^3$ 		& \Sc{0} 	& -\ut	& 0		& 1 	&\Sc{-\ut}& 0 \\
\end{tabular}\caption{Tableau finale.}
}
\label{tab:tab419}
\end{table}
La base iniziale � $\mathcal{B}_1={A_3,A_5}$. Applicando delle semplici sottrazioni, ricaviamo la soluzione del problema duale:
\begin{align*}
\pi_1&=c_3-\OL{c_3}=0 \\
\pi_3&=c_5-\OL{c_5}=\frac{1}{3}
\end{align*}
Perci�, la soluzione del problema duale � il vettore:
$$
\pi'=
\begin{bmatrix}
0 & \frac{1}{3}
\end{bmatrix}
$$
Per verificare la correttezza dei calcoli, applichiamo la soluzione alla funzione obiettivo del problema duale:
$$
\xi(0,\frac{1}{3})=2(0)+0+6(\frac{1}{3})=2
$$
Il risultato �, come atteso, lo stesso del problema primale.

\section{Esercizio 3}

Sia dato il seguente modello matematico di un problema di LP:
\begin{align*}
\min z	&= -x_1-x_2 \\
\st\;\;	& x_2 \leq 1\\
		& -x_1+x_2\geq 2\\
	  	& x_1,x_2 \geq 0
\end{align*}
\begin{itemize}
\item Si rappresenti accuratamente il problema in forma grafica;
\item Si ricavi la forma standard;
\item Si risolva il problema tramite il metodo del simplesso a due fasi utilizzando il minor numero di variabili artificiali.
\end{itemize}

\subsection{Problema in forma grafica}

In figura \vref{fig:graph6} � rappresentato graficamente il problema presentato. Il gradiente della funzione obiettivo vale:
\begin{equation*}
\nabla(z)=\left(\frac{\partial z}{\partial x},\frac{\partial z}{\partial y}\right) = \left( -1,-1 \right) \\
\end{equation*}
Poich� questa volta il problema � rappresentato sotto forma di minimo, saremo interessati alla limitazione del politopo nella direzione \textbf{opposta} al gradiente. A tal fine utilizziamo una funzione obiettivo ausiliaria $\varphi$ tale che:
\begin{equation*}
\varphi = -z \Rightarrow \nabla\varphi = -\nabla z = (-2,-1)
\end{equation*}
In giallo � rappresentata l'area tra i due vincoli lineari di disuguaglianza. Notiamo che quest'area si estende nel secondo e nel terzo quadrante, perci� non rispetta i vincoli di non minoranza delle singole variabili. Ci aspettiamo una soluzione impossibile dal simplesso.
\begin{figure}[htbp]
\centering
\begin{tikzpicture}
\begin{axis}
[axis lines=middle, axis equal, enlargelimits, xlabel=$x_1$, ylabel=$x_2$,
 every axis x label/.style={
    at={(ticklabel* cs:1.01)},
    anchor=west,
 },
 every axis y label/.style={
    at={(ticklabel* cs:1.01)},
    anchor=south,
 },%xtick={1,2,3}
 ]
    \path[name path=AX] 
        (axis cs:\pgfkeysvalueof{/pgfplots/xmin},0)--
        (axis cs:\pgfkeysvalueof{/pgfplots/xmax},0);
    \path[name path=AY] 
        (axis cs:0,\pgfkeysvalueof{/pgfplots/ymin})--
        (axis cs:0,\pgfkeysvalueof{/pgfplots/ymax});
    \path[name path=UP]
    	(axis cs:\pgfkeysvalueof{/pgfplots/xmin},\pgfkeysvalueof{/pgfplots/ymax})--
    	(axis cs:\pgfkeysvalueof{/pgfplots/xmax},\pgfkeysvalueof{/pgfplots/ymax});

\addplot
[domain=-3.01:3.01, samples=10, thick, blue, name path=yx2]
{x+2} node [pos=0.6,pin={135:{\color{blue}$-x_1+x_2=2$}}, inner sep=0pt] {};
\addplot
[domain=-3.01:3.01, samples=10, thick, red, name path=y1]
{1} node [pos=0.5, pin={75:{\color{red}$x_2=1$}}, inner sep=0pt] {};
\addplot[thick, fill=yellow, fill opacity=0.5] fill between [of=yx2 and y1, soft clip={domain=-3:-1}];
%\addplot[white] fill between [of=xy2 and AX];
%\addplot[pattern=north east lines, pattern color=red!10] fill between [reverse=true, of=AX and UP, soft clip={domain=0:5}];
\addplot[pattern=north west lines, pattern color=blue!10] fill between [of=yx2 and UP];
\addplot[pattern=vertical lines, pattern color=red!10] fill between [of=y1 and AX];
\intne{y1}{yx2}{$\alpha$}{alp};
%\node at (axis cs:1.8,0.5) {$P$};
\addplot[-latex, thick] coordinates
           {(0,0) (1/1.414,1/1.414)} node [pos=.3, anchor=south, label={45:{\small $\nabla\varphi$}}] {};
\end{axis}
\end{tikzpicture}
\caption{Rappresentazione cartesiana del problema di programmazione lineare}
\label{fig:graph6}
\end{figure}

\subsection{Forma standard}

Aggiungendo una variabile slack e una variabile surplus, il problema in forma standard si presenta cos�:
\begin{alignat*}{6}
&\min z = \quad && -x_1 \quad\; && -x_2 \quad\;\; && \qquad\qquad && \qquad\qquad && \\
&\;\st  &&				&&+x_2	 		&&+\pmb{x_3}	&&		 		&&=1\\
&	 	&&-x_1			&&+x_2			&&				&& -\pmb{x_4}	&&=2\\
&		&&\quad\; x_1,	&&\quad\; x_2,	&&\quad\; x_3,	&&\quad\; x_4,	&&\geq 0
\end{alignat*}

\subsection{Risoluzione tramite tableau}

\begin{table}[htbp]
\centering
\begin{tabular}{rccccc}
			&$-z$ & $x_1$ & $x_2$ & $x_3$ & $x_4$ \\
$\OL{c_j}$ 	& \Sc{0} 	& -1 	& -1 	& 0 	& 0 \\
\cline{2-6}
$R_1$ 		& \Sc{1} 	& 0 	& 1 	& 1 	& 0\\
$R_2$		& \Sc{2} 	& -1 	& 1		& 0 	& -1 \\
\end{tabular}
\caption{Tableau iniziale.}
\label{tab:tab420}
\end{table}

In tabella \vref{tab:tab420} il tableau ricavato dal nostro problema. Non abbiamo nessuna sottomatrice identit� a disposizione da utilizzare come base ammissibile quindi ricorriamo alla \textbf{fase 1 del metodo del simplesso} per ottenere una BFS di partenza.

\subsubsection{Fase 1 - aggiunta di variabili artificiali}

Manca solo la seconda colonna della matrice identit� con cui formare la BFS di partenza e la introdurremo con l'\textit{unica} variabile artificiale $x^a$, trasformando il secondo vincolo in:
$$
-x_1 + x_2 - x_4 + x^a = 2
$$
Il nostro scopo, dopo l'introduzione di $x^a$, sar� quello di \textbf{eliminarla} dalla base. Per far ci� bisogna fare in modo che questa valga zero e quindi introduciamo, a tale scopo, una nuova funzione obiettivo da minimizzare $\psi$ tale che:
$$
\psi = \sum_{i=1}^{n'}x_i^a = x^a
$$
Scriviamo il nuovo tableau in tabella \vref{tab:tab421} e applichiamo il simplesso per ottimizzare la nostra funzione $\psi$.
\begin{table}[htbp]
\centering
\begin{tabular}{rrcccccc}
 	  & 			&$-\psi$	& $x_1$ & $x_2$ & $x_3$ & $x_4$	& $x^a$\\
$R_0$ & $\OL{c_j}$ 	& \Sc{0} 	& 0 	& 0 	& 0 	& \Sc{0}& 1\\
\cline{3-8}
$R_1$ & $x_3$ 		& \Sc{1} 	& 0		& 1 	& 1 	& \Sc{0}& 0 \\
$R_2$ & $x^a$ 		& \Sc{2} 	& -1	& 1		& 0 	&\Sc{-1}& 1 \\
\end{tabular}
\caption{Nuovo tableau con la variabile artificiale $x^a$.}
\label{tab:tab421}
\end{table}
Abbiamo una sottomatrice identit� formata dalla base:
$$
\mathcal{B}=\{A_3,A_5\}
$$
Per avere a avere a disposizione i valori delle coordinate della BFS del nuovo problema, � necessario che:
$$
y_{ij}=0 \quad \forall i,j:A_j\in\mathcal{B},i\neq j
$$
Condizione vera per ogni valore tranne $y_{05}$ che provvediamo ad annullare tramite l'operazione elementare di riga:
$$
R_0\leftarrow R_0 - R_2
$$
Nel nuovo tableau in figura \vref{tab:tab422} dobbiamo scegliere su quale colonna fare pivoting. L'unica colonna con $\OL{c_j}<0$ � $A_2$ e cercheremo qui l'elemento pivot.
Per scegliere su quale elemento fare \textbf{pivoting}, dobbiamo ottenere il valore di $y_{\ell 2}$ tale che:
$$
\vartheta_{\max}=\min_{i:y_{i2}>0}\frac{y_{i0}}{y_{i2}}=\frac{y_{i0}}{y_{\ell 2}}
$$
Perci�, operando con gli elementi nel tableau:
$$
\vartheta_{\max}=\min\left(\frac{1}{1},\frac{2}{1}\right)=\frac{1}{1}=\frac{y_{10}}{\pmb{y_{12}}}
$$
Faremo pivoting sull'elemento $y_{12}$ (cerchiato in tabella). Il nostro scopo � ora far comparire uno 0 nella colonna dell'elemento pivot in tutte le righe tranne quella in cui si trova l'elemento pivot e far comparire un 1 in quest'ultima.
\begin{table}[htbp]
\centering
\begin{tabular}{rrcccccc}
 	  & 			&$-\psi$	& $x_1$ & $x_2$ & $x_3$ & $x_4$	& $x^a$\\
$R_0$ & $\OL{c_j}$ 	& \Sc{-2} 	& 1 	& -1 	& 0 	& \Sc{1}& 0\\
\cline{3-8}
$R_1$ & $x_2$ 		& \Sc{1} 	& 0		& \C{1}	& 1 	& \Sc{0}& 0 \\
$R_2$ & $x^a$ 		& \Sc{2} 	& -1	& 1		& 0 	&\Sc{-1}& 1 \\
\end{tabular}
\caption{Pivoting su $y_{12}$. $A_2$ entra in base e $A_3$ esce.}
\label{tab:tab422}
\end{table}
Poich� $y_{12}=1$ non c'� nulla da fare su $R_1$. Applichiamo le operazioni elementari di riga al nostro tableau come segue:
\begin{align*}
R_0&\leftarrow R_0 + R_1; \\
R_2&\leftarrow R_2 - R_1.
\end{align*}
Il nostro nuovo tableau diventa quindi quello in tabella \vref{tab:tab423}.
\begin{table}[htbp]
\centering
\begin{tabular}{rrcccccc}
 	  & 			&$-\psi$	& $x_1$ & $x_2$ & $x_3$ & $x_4$	& $x^a$\\
$R_0$ & $\OL{c_j}$ 	& \Sc{-1} 	& 1 	& 0 	& 1 	& \Sc{1}& 0\\
\cline{3-8}
$R_1$ & $x_2$ 		& \Sc{1} 	& 0		& 1 	& 1 	& \Sc{0}& 0 \\
$R_2$ & $x^a$ 		& \Sc{1} 	& -1	& 0		& -1	&\Sc{-1}& 1 \\
\end{tabular}
\caption{Secondo tableau. $x^a$ ancora in base e $\psi\neq 0$.}
\label{tab:tab423}
\end{table}
Siamo nella soluzione ottima ma:
\begin{itemize}
\item la variabile artificiale $x^a$ � ancora fuori base;
\item la soluzione ottima non � nulla.
\end{itemize}
Non vale la pena sprecare energie per far uscire dalla base la variabile artificiale. Qualunque operazione facessimo, il valore della soluzione ottima non potrebbe migliorare e quindi non potrebbe annullarsi. Ci� significa che non esistono soluzioni ammissibili al nostro problema.
\subsection{Soluzione del problema}
Il problema non ha soluzione in quanto nella fase 1 del metodo del simplesso non siamo riusciti a trovare una BFS che non coinvolga variabili artificiali.

\end{document}
